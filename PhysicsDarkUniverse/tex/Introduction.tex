Dark matter (DM) \footnote{Many theories of physics beyond the Standard Model predict the existence
of stable, neutral, weakly-interacting and massive particles that are
putative DM candidates. In the following, we refer to such
matter as DM, even though the observation of such matter at a collider
could only establish that it is neutral, weakly-interactive, massive and stable
on the distance-scales of tens of meters.} has not yet been observed in particle physics experiments, and
there is not yet any evidence for non-gravitational interactions
between DM and the Standard Model (SM) particles.  If such
interactions exist, particles of DM could be produced
at the LHC. Since DM particles themselves do not produce signals
in the LHC detectors, one way to observe them is when they are produced in association
with a visible SM particle X(=$g, q, \gamma, Z, W$, or $h$).
Such reactions, which are
observed at colliders as particles or jets recoiling against an invisible state, are
called ``mono-X'' or \MET{}+X reactions (see e.g 
Refs.~\cite{Birkedal:2004xn,Feng:2005gj,Petriello:2008pu,Beltran:2010ww,Bai:2010hh}), 
where \MET is the missing transverse momentum observable in the detector.

Early Tevatron and LHC Run-1 searches for \MET{}+X signatures at 
CDF~\cite{Aaltonen:2012jb}, 
ATLAS~\cite{Aad:2015zva,Aad:2014tda,ATLAS:2014wra,Aad:2014vka,Aad:2013oja,Aad:2014wza,Aad:2014vea,ATL-PHYS-PUB-2014-007}
and
CMS~\cite{Khachatryan:2014rra,Khachatryan:2014rwa,Khachatryan:2014tva,Khachatryan:2014uma,Khachatryan:2015nua,CMS-PAS-B2G-13-004,CMS-PAS-EXO-14-004},
employed a basis of contact interaction operators in effective field
theories (EFTs) \cite{Goodman:2010yf,Goodman:2010ku} to calculate the
possible signals. %\Todo{Check these references here}
These EFTs assume that production of DM takes place through a
contact interaction involving a quark-antiquark pair, or two gluons,
and two DM particles.  In this case, the missing energy
distribution of the signal is determined by the nature and the mass of
the DM particles and the Lorentz structure of the
interaction. Only the overall production rate is a free parameter to
be constrained or measured.  Provided that the contact interaction
approximation holds, these EFTs provide a straightforward way to
compare the results from different collider searches with non-collider
searches for DM.  

The EFT describes the case when the mediator of the interaction between SM and DM particles are very heavy; 
if this is not the case, models that explicitly include these mediators are 
needed~\cite{Goodman:2010yf,Shoemaker:2011vi,Bai:2010hh,Kopp:2011eu,Fox:2011fx,Fox:2011pm,Shoemaker:2011vi,Busoni:2013lha}.
Some ``simplified models'' \cite{Alwall:2008ag,Goodman:2011jq,Alves:2011wf}
of DM production were constructed, including particles and interactions beyond the SM.
These models can be used consistently at LHC energies, and provide
an extension to the EFT approach. 
Many proposals for such models have emerged (see, for example
Refs. \cite{An:2012va,An:2012ue,Tait:2013,Buchmueller:2013dya,Bai:2013iqa,Bai:2014osa,An:2013xka,Yavin:14092893,Malik:2014ggr,Harris:2014hga,Buckley:2014fba,Haisch:2015ioa,Bai:2012xg,Carpenter:2012rg,Bell:2012rg,Petrov:2013nia,Carpenter:2013xra}). 
At the LHC, the kinematics of mono-X reactions occurring via a \tev-scale mediator can differ substantially from the prediction of the contact
interaction. The mediator may also produce qualitatively different signals, such as decays back into the SM particles. 
Thus, appropriate simplified models are an important component of the design, optimization, and interpretation of DM searches at ATLAS and CMS.
This has already been recognized in the CDF, ATLAS and CMS searches quoted above, where both EFT and selected simplified model
results are presented. 

\subsection{The ATLAS/CMS Dark Matter Forum}

To understand what signal models should be considered for the upcoming LHC Run-2, 
groups of experimenters from both ATLAS and CMS collaborations have held separate 
meetings with small groups of theorists, and discussed further at the DM@LHC 
workshop~\cite{Malik:2014ggr,Yavin:14092893,DMatLHCProceedings}. 
These discussions identified overlapping sets of simplified models as possible
benchmarks for early LHC Run-2 searches. 
Following the DM@LHC workshop, ATLAS and CMS organized a forum, called the~\textit{ATLAS-CMS Dark
Matter Forum}, to form a consensus on the use of these simplified models
and EFTs for early Run-2 searches with the participation of experts on
theories of DM. This is the final report of the ATLAS-CMS Dark Matter Forum.

One of the guiding principles of this report is to channel the efforts
of the ATLAS and CMS collaborations towards a minimal basis of dark
matter models that should influence the design of the early Run-2
searches. At the same time, a thorough survey of realistic collider
signals of DM is a crucial input to the overall design of the
search program.

The goal of this report is such a survey, though confined within some
broad assumptions and focused on benchmarks for kinematically-distinct
signals which are most urgently needed. As far as time and resources
have allowed, the assumptions have been carefully motivated by
theoretical consensus and comparisons of simulations. But, to achieve such a 
consensus in only a few months before the start of Run-2, it was
important to restrict the scope and timescale to the following:

\begin{enumerate}
\item The forum should propose a prioritized, compact set of benchmark
  simplified models that should be agreed upon by both collaborations for
  Run-2 searches. The values for the scan on the parameters of the models for which
  experimental results are provided should be specified, to facilitate theory reinterpretation 
  beyond the necessary model-independent limits that 
  should be provided by all LHC DM searches. 
\item The forum should recommend the use of the state of the art calculations
  for these benchmark models. Such a recommendation will aid the  
  standardization the event generator implementation
  of the simplified models and the harmonization of other common technical
  details as far as practical for early Run-2 LHC analyses. It
  would be desirable to have a common choice of leading order (LO) and 
  next-to-leading order (NLO) matrix elements corresponding to the state of the art calculations, 
  parton shower (PS) matching and merging, factorization and renormalization
  scales for each of the simplified models. This will also lead to a
  common set of theory uncertainties, which will facilitate the
  comparison of results between the two collaborations.
\item The forum should discuss how to apply the
  EFT formalism and present the results of EFT
  interpretations.
\item The forum should prepare a report summarizing these items,
  suitable both as a reference for the internal ATLAS and CMS
  audiences and as an explanation of early Run-2 LHC benchmark models for theory and non-collider
  readers. This report represents the views of its endorsers, as participants of the forum.
\end{enumerate}

\subsection{Grounding Assumptions}

We assume that interactions exist between the SM hadrons
and the particles that constitute cosmological DM. If this
is not the case, then proton collisions will not directly produce DM
particles, and DM will not scatter off nuclei in direct
detection experiments.

The DM itself is assumed to be a single particle, a Dirac
fermion WIMP, stable on collider timescales and non-interacting with
the detector.  
The former assumption is reductionistic.
The rich particle content of the SM is circumstantial evidence that
the DM sector, which constitutes five times as much of the
mass of the universe, may be more complex than a single particle or a
single interaction. But, as was often the case in the discoveries of
the SM, here only one mediator and one search channel might play a
dominant role in the opening stages of an LHC discovery. The latter
assumption focuses our work on early LHC searches, where small
kinematic differences between models will not matter in a discovery
scenario, and with the imminent re-start of the LHC our report relies
heavily on a large body of existing theoretical work which assumed Dirac fermionic DM. 

Different spins of DM particles will typically
give similar results. Exceptions exist: For example, the choice of Majorana fermions forbids some
processes that are allowed for Dirac fermions~\cite{Goodman:2010yf}.
Aside from these, adjusting the choice of Dirac or Majorana fermions or scalars will produce only minor changes
in the kinematic distributions of the visible particle and is expected to have little effect
on cut-and-count\footnote{Cut-and-count refers to an analysis
that applies a certain event selection and checks the inclusive number of events which pass. 
This is to be contrasted with a shape analysis, which compares the distribution of events.} analysis. Thus the choice of Dirac
fermion DM should be sufficient as benchmarks for the upcoming Run-2 searches. 

One advantage of collider experiments lies in their ability to study
and possibly characterize the mediator. A discovery of an anomalous
\MET signature at the LHC would not uniquely imply discovery of dark
matter, while at the same time e.g. discovery of an anomalous and
annually-modulated signal in a direct-detection experiment would leave
unanswered many questions about the nature of the interaction that
could be resolved by the simultaneous discovery of a new mediator
particle. Collider, direct, and indirect detection searches provide
complementary ways to approach this problem~\cite{Bauer:2013ihz}, and it is in this spirit
that much of our focus is on the mediator.

We systematically explore the basic possibilities for
mediators of various possible spins and couplings.
All models considered are assumed to produce a signature with pairs of DM particles.
Though more varied and
interesting possibilities are added to the literature almost daily,
these basic building blocks account for much of the physics studied at
hadron colliders in the past three decades.

We also assume that Minimal Flavor Violation (MFV) \cite{Chivukula:1987py,Hall:1990ac,Buras:2000dm,D'Ambrosio:2002ex} applies to the
models included in this report. This means that the flavor structure of the
couplings between DM and ordinary particles follows the same
structure as the SM. This choice is simple, since no
additional theory of flavor is required, beyond what is already
present in the SM, and it provides a mechanism to ensure that the
models do not violate flavor constraints.  As a consequence, \spinzero
resonances must have couplings to fermions proportional to the SM Higgs couplings. 
Flavor-safe models can still be constructed beyond the MFV
assumption, for example ~\cite{Agrawal:2014aoa}, and deserve further study.
For a discussion of MFV in the context of the simplified models
included in this report, see Ref.~\cite{DMatLHCProceedings}.

In the parameter scan for the models considered in this report, we make the
assumption of a minimal decay width for the particles mediating the
interaction between SM and DM.  This means that only decays
strictly necessary for the self-consistency of the model (e.g.  to DM
and to quarks) are accounted for in the definition of the mediator
width. We forbid any further decays to other invisible particles of
the Dark Sector that may increase the width or produce striking, visible signatures. 
Studies within this report show that, for cut-and-count analyses, the kinematic distributions of
many models, and therefore the sensitivity of these searches, do not depend
significantly on the mediator width, as long as the width remains smaller
than the mass of the particle and that narrow mediators are sufficiently light.

The particle content of the models chosen as benchmarks is limited to
one single kind of DM whose self-interactions are not relevant for LHC
phenomenology, and to one type of SM/DM interaction at a time. These
assumptions only add a limited number of new particles and new interactions to the
SM. These simplified models, independently explored by different
experimental analyses, can be used as starting points to build more
complete theories. Even though this factorized picture does not always
lead to full theories and leaves out details that are necessary for
the self-consistency of single models (e.g. the mass generation for
mediator particles), it is a starting point to prepare a set of
distinct but complementary collider searches for DM, as
it leads to benchmarks that are easily comparable across channels.

\subsection{Choices of benchmarks considered in this report and parameter scans}

Contact interaction operators have been outlined as basis set of theoretical
building blocks representing possible types of interactions between SM and DM particles
in~\cite{Goodman:2010ku}. The approach followed by LHC searches (see e.g. Refs.~\cite{Khachatryan:2014rra,Aad:2015zva} 
for recent jet+\MET{} Run-1 searches with the 8 TeV dataset) 
so far has been to simulate only a prioritized set of the possible operators with distinct kinematics
for the interpretation of the constraints obtained, and provide results that may be reinterpreted in terms of the other operators.
This report intends to follow this strategy, firstly focusing on simplified models that allow the exploration 
of scenarios where the mediating scale is not as large.  In the limit of large mediator mass, the simplified models map onto
the EFT operators.
Secondly, this report considers specific EFT benchmarks 
whenever neither a simplified model completion 
nor other simplified models yielding similar kinematic distributions are available 
and implemented in one of the event generators used by both collaborations. 
This is the case for dimension-5 or dimension-7 operators with direct 
DM-electroweak boson couplings \footnote{An example of a dimension-5 operator for scalar
DM is described in Appendix~\ref{app:EWSpecificModels_Appendix}. 
Dimension-7 operators of DM coupling to gauge bosons exist in the literature, but they require a larger particle spectrum
with respect to the models studied in this report.}.
Considering these models as separate experimental benchmarks 
will allow to target new signal regions and help validate the 
contact interaction limit of new simplified models 
developed to complete these specific operators. 
Results from these EFT benchmarks should include the condition that
the momentum transfer does not probe the scale of the interaction; whenever there is no model
that allows a direct mapping between these two quantities, various options should be tested to 
ensure a given fraction of events within the range of applicability of the EFT approach.
Experimental searches should in any case deliver 
results that are independent from the specific benchmark tested, such as fiducial cross-sections that
are excluded in a given signal region. 

When choosing the points to be scanned in the parameter space of the models,
this report does not quantitatively consider constraints that are 
external to the MET+X analyses. This is the case also for results from LHC experiments
searching for mediator decays. 
The main reason for not doing so in this report 
is the difficulty of incorporating these constraints in a rigorous quantitative way within
the timescale of the Forum. However, even if the parameter scans
and the searches are not optimized with those constraints in mind, 
we intend to make all information available to the community to exploit
the unique sensitivity of colliders to all possible DM signatures. 

\subsection{Structure of this report and dissemination of results}

The report provides a brief theoretical summary of the models considered, 
starting from the set of simplified models and contact interactions put forward 
in previous discussions and in the literature cited above. 
Its main body documents the studies 
done within this Forum to identify a kinematically distinct set of model parameters
to be simulated and used as benchmarks for early Run-2 searches. The implementation
of these studies according to the state of the art calculations is detailed,
including instructions on how to estimate theoretical uncertainties in the generators used
for these studies. The presentation of results for EFT benchmarks is also covered. 

Chapter \ref{subsec:MonojetLikeModels} of this report is dedicated to simplified
models with radiation of a hard object either from the initial state
or from the mediator. These models produce primarily monojet signatures, 
but should be considered for all \MET{}+X searches.
Chapter~\ref{subsec:EWSpecificModels} contains studies on the benchmark models
for final states specifically containing an electroweak 
boson ($W/Z/\gamma/H$). In this case, both 
simplified models leading to mono-boson signatures
and contact interaction operators are considered. 
Details of the state of the art calculations and on the implementation of the simplified models in
Monte Carlo generators are provided in
Chapter \ref{app:MonojetLikeModels_Appendix}.
Chapter \ref{sec:EFTValidity} is devoted to the treatment of the presentation of results for the benchmark
models from contact interaction operators. 
Chapter \ref{sec:TheoryUncertainties} prescribes how to estimate theoretical uncertainties on the simulation of these models. 
Chapter \ref{chapter:conclusions} concludes the report.

Further models that could be studied
beyond early searches and their implementation are described in Appendix~\ref{app:EWSpecificModels_Appendix}. 
For these models, either the implementation could not be fully developed by the time of this report,
or some of the grounding assumptions were not fully met.  
Some of these models have been used in previous ATLAS and CMS analyses and discussed thoroughly within the Forum. 
They are therefore worth considering for further studies and for Run-2 searches, since they lead to unique \MET{}+X signatures 
that are not shared by any other of the models included in this report. 
Appendix~\ref{app:Presentation_Of_Experimental_Results} contains the necessary elements that
should be included in the results of experimental searches to allow for further reinterpretation. 

It is crucial for the success of the work of this Forum that these studies can be employed as cross-check
and reference to the theoretical and experimental community interested in early Run-2 searches. 
For this reason, model files, parameter cards, and cross-sections for the models considered in these studies 
are publicly available. The SVN repository of the Forum~\cite{ForumSVN} contains the models and parameter files
necessary to reproduce the studies within this report. Details and cross-sections for these models, 
as a function of their parameters, will be published on HEPData~\cite{HEPData}. 

