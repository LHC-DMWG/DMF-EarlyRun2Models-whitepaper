\subsection{\texorpdfstring{Models with a single $top-$quark + \MET}{Models with a single top-quark + MET}}
\label{sec:singletop}
\textbf{[TODO: find a consistent notation for Xnew, M, V and Madgraph model]}

\newcommand{\SUtwoUone}{\ensuremath{\mathrm{SU}(2)_{L} \times \mathrm{U}(1)_{Y}}\xspace}
\newcommand{\SUtwo}{\ensuremath{\mathrm{SU}(2)_{L}}\xspace}
\newcommand{\Lagr}{\ensuremath{\mathcal{L}}\xspace}
\newcommand{\fmet}{\ensuremath{f_{\mathrm{met}}}\xspace}
\newcommand{\met}{MET\xspace}
\newcommand{\ares}{\ensuremath{a_{\mathrm{res}}}\xspace}
\newcommand{\anonres}{\ensuremath{a_{\mathrm{non-res}}}\xspace}
\newcommand{\Xnew}{\ensuremath{X_{\mathrm{new}}}\xspace}
\newcommand{\BR}[2]{\ensuremath{\mathrm{BR}({#1} \to {#2})}\xspace}
\newcommand{\ssll}{\ensuremath{\ell^{+}\ell^{+}}\xspace}
\newcommand{\ttbarV}{\ensuremath{\ttbar V}\xspace}
\def\mfmet{\ensuremath{m(f_{\mathrm{met}})}}
\def\mvmet{\ensuremath{m(v_{\mathrm{met}})}}
\def\vmet{\ensuremath{v_{\mathrm{met}}}}
\def\fmet{\ensuremath{f_{\mathrm{met}}}}


A dark matter candidate $\chi$ and a new particle $M$ (vector or scalar) 
are added to the SM, in an effective theory that respects the $\SUtwoUone$ symmetry 
%%CD: effective theory???
and produces a single top quark in association with either the DM particle or the new particle
(generally called $\Xnew$ when no distinction is made). The full details of 
these models are described in~\cite{AndreaFuksMaltoni,Agram:2013wda,Boucheneb:2014wza}. 

There are two classes of models based on the monotop production mode: 
resonant and non-resonant production, as shown 
in Fig.~\ref{fig:feyn_prod}. 

The following two sections describe the phenomenology leading to these two production mechanisms.
Depending on the nature of $\Xnew$, two main final states might be relevant: 
monotop production or same-sign top quark pair production. 
The interplay of these two signatures can largely probe this class of dark matter model,
but a detailed study of their complementarity is beyond the scope of this Forum report. 

\begin{figure}[!h!tpd]
\centering
\includegraphics[width=0.31\textwidth]{figures/singletop/feyn_diags/Resonant}
\includegraphics[width=0.31\textwidth]{figures/singletop/feyn_diags/NonResonant}
\includegraphics[width=0.31\textwidth]{figures/singletop/feyn_diags/NonResonant2}
%\subfigure[\label{subfig:S1}]{\includegraphics[width=0.46\textwidth]{feyn_diags/S1}}\\
%\subfigure[\label{subfig:S4_s}]{\includegraphics[width=0.46\textwidth]{feyn_diags/S4_s}}
%\subfigure[\label{subfig:S4_t}]{\includegraphics[width=0.46\textwidth]{feyn_diags/S4_t}}
\caption
{
%Feynman diagram of leading order processes leading to monotop events: production of
%a coloured scalar resonance $S$ decaying into a top quark and a spin-$1/2$ fermion $f_{met}$
%in the $\mathrm{S1_R}$ model~\subref{subfig:S1}, and $s$-\subref{subfig:S4_s}
%and $t$-\subref{subfig:S4_t} channel non resonant production of a top quark in association with
%a spin-1 boson $v_{met}$ in the $\mathrm{S4_R}$ model.
Feynman diagram of leading order processes leading to monotop events: resonant production of
$t$ via resonant new particle $M$ decaying into a top quark and $\Xnew$, which is the dark matter fermion $\chi$ (left),
and $s$ and $t$ channel non-resonant production of a top quark in association with $\Xnew$, which is the new particle $M$ (middle and right).
}
\label{fig:feyn_prod}

\end{figure}

\newthought{Resonant production}
\label{sec:ResonantProd}

In this case, the new particle $M$ is a couloured $2/3$-charged scalar $\phi^{\pm}$ decaying into a top quark and a spin-$1/2$ invisible particle, $\chi$ (in this case $\Xnew$ is the dark matter candidate $\chi$). 
The dynamics of the new sector is then described by the following Lagrangian:

\begin{eqnarray}
\label{eq:lagrangianResonant}
\mathcal{L} =  d^{C}_{i} \:  [ (g^{v}_{\phi d})^{ij} +  (g^{a}_{\phi d})^{ij} \gamma^{5} ] \: d_{j} \: \phi^{\pm}  +  u^{C}_{k}  [ (g^{v}_{u\chi})^{k} + (g^{a}_{u\chi})^{k} \gamma^{5} ] \: \chi \: \phi^{\pm}
%%CD: problems with the original typesetting
\end{eqnarray}
where $u$ ($d$) stands for any $up$-quark ($down$-quark), 
the index $v$ ($a$) stands for vectorial (axial), $C$ means charge conjugate and $i,j,k$ 
run over the generations (color indices involved in the $\phi^{\pm}-$quarks interaction are not explicitly written).
The first term leads to the production of the new particle and the last term allows its decay into a $up$-quark 
and a non interacting fermion (in particular to the top quark when $(g^{v/a}_{u\chi})^{k}$ 
is sizable mainly for $k=3$).
This model is then described by the masses of the new particle $m_{\phi}$ and the invisible 
fermion $m_{\chi}$, and the coupling 
$(g^{v/a}_{\phi d})^{ij}$ and $(g^{v/a}_{u\chi})^{k}$.
% 
% \com{Question/comment: in this resonant model, this is not so obvious to interpret $\phi_{\pm}$ as the new particle since there is a vertex $\phi-u-\chi$.
% It is somehow breaking the concept of having a dark sector weakly coupled to ordinary matter via a new particle.}
% 
% 
\newthought{Non-Resonant production}
\label{sec:NonResonantProd}

For the non-resonant production, the top quark is produced in association with the new particle 
($\Xnew$ is then the new particle and not the dark matter candidate). 
The new particle can be either a new scalar, interacting with the SM and the DM candidate, 
or a new vector. For simplicity, we only consider the case of a vector new particle, as the scalar case
would involve a mixing with the SM Higgs boson and therefore a larger parameter space. 

%%CD: Full explanation is below
% First, the new particle can be a scalar field interacting with the SM field and the dark matter
% candidate as described in this lagrangian:
% \begin{equation}
%  \label{eq:lagrangianNonResonantScalar}
% \mathcal{L} =  u^{C}_{i} \:  [ (g^{v}_{\phi u})^{ij} +  (g^{a}_{\phi u})^{ij}  \gamma^{5} ] \: u_{j} \: \phi  
%  +  \chi^{C}  [ g^{v}_{\phi\chi} + g^{a}_{\phi\chi}  \gamma^{5} ] \chi \: \phi
% \end{equation}
% where $u$ stands for any $up$-quark, the index $v$ ($a$) stands for vectorial (axial), 
% $C$ means charge conjugate and $i,j,k$ run over the generations.
% The first term describes the interaction between the new particle and the $up$-quarks while the 
% second term leads to the decay of the new particle into invisible fermions. 
% In this model, there is necessarily a mixing between $\phi$ and  the Higgs boson field. 
% Additional parameters are then required to describe this new sector: in addition to 
% the new particle mass and couplings, the mixing matrix of the two scalar fields
% is needed in order to make predictions. For the sake of simplicity, 
% we do not consider this case were the parameters space would be too large.

The dynamics of a case with a vector new particle follows this Lagrangian:
\begin{equation}
 \label{eq:lagrangianNonResonantVector}
  \mathcal{L}  =  \bar{u}_{i} [ (g^{v}_{Vu})^{ij} \gamma^{\mu} + (g^{a}_{Vu})^{ij} \gamma^{5} ] u_{j} \: V_{\mu}  
  +  \bar{\chi} [ g^{v}_{Vu} \gamma^{\mu} + g^{a}_{V\chi} \gamma^{5} ]   \chi \: V_{\mu}
\end{equation}
where $u$ stands for any $up$-quark, the index $v$ ($a$) stands for vectorial (axial) and $i,j,k$ run over the generations.
The first term describes the interaction between the new particle and the $up$-quarks while the second term leads to the decay the new particle 
into invisible fermions. The new sector can be defined with the couplings $(g^{v/a}_{Vu})^{ij}$, 
$g^{a/v}_{V\chi}$ and the masses $m_V$ and $m_{\chi}$. 
This model can be probed by two different experimental signatures: monotop and same-sign top quark production. 
% 
% \com{Question for theorists: why it cannot mix with $\Zboson$ in case of vectorial new particle ?}

\newthought{Model parameters and assumptions}
 
The models considered as benchmarks for the first LHC searches
contain further assumptions in terms of the flavour and chiral structure of the model
with respect to the full Lagrangians from  equations~\eqref{eq:lagrangianResonant} and~\eqref{eq:lagrangianNonResonantVector}.
These assumptions lead to limitations in LHC constraints of 
the parameter space of these models, qualitatively discussed below. 

\paragraph{Assumptions in the flavour structure of the models}

In order to be visible at the LHC in the monotop final state, 
these models must include a strong coupling between the new particle $\phi$ and $t\chi$.
In the resonant case, the new particle must also couple to light quarks in order
to be produced at the LHC, leading to possible constraints from
dijet searches. 
The same kind of assumption exists for the non-resonant production. 
The new particle $M$ must be produced from a light quark in the initial state, 
in association with a top quark: this signature can mainly probe a high 
coupling $\left(g^{v/a}_{Vu}\right)^{13}_{Vu} \equiv g^{v/a}_{Vtu}$. Therefore,
the sensitivity to other flavour couplings is significantly lower, since $V$ would be 
produced at a lower rate. 

%CD: not sure I understand?
% In addition, the new particle must decay into invisible particles
% to lead to the searched monotop final state. As a consequence, the sensitivity for 
% scenario where $\BR{V}{\chi\chi}\ll 100\%$ can be quite low. 
% To cope with this second limitation, a same-sign top quark final state 
% $gu \to tV(\to t\bar{u})$ is proposed to cover the cases where $V$ would decay
% into visible particles. This case is more likely as the $tV$ production rate increases, 
% and becomes then a key point to constraint this model in a consistent way.

% % \com{Questions for theorists:
% % \begin{itemize}
% %  \item How well these flavour assumptions are allowed by the other HEP data (proton decay life time, flavour physics, etc ...) ?
% %  \item MFV criteria ?
% % \end{itemize}
% % }
% 
% \subsubsection{Chiral structure}
% \label{sec:chiralstructure}
\paragraph{Assumptions in the chiral structure of the models}

We only consider right-handed quark components, in order to simplify the phenomenology. 
The representation of the left-handed components under the $\SUtwo$ symmetry would a 
coupling to $down$-type quarks, since the effective theory is invariant under $\SUtwoUone$ 
gauge symmetry. Having a coupling between the new particle and $down$-type quarks 
complicates the collider phenomenology in terms of decay modes. Typically, including 
the left-handed components of quarks in the lagrangian~\eqref{eq:lagrangianNonResonantVector} 
describing the $Vtu$ vertex would lead to 
\begin{equation}
 \mathcal{L}_{Vtu} \; = \;  g^{R}_{Vtu} \: \bar{t}_{R}\gamma^{\mu}u_R \: V_{\mu} \; + \; g^{L}_{Vtu} 
 (\bar{t}_{L}\gamma^{\mu}u_L \: + \:  \bar{b}_{L}\gamma^{\mu}d_L ) \: V_{\mu}
\end{equation}
where $g^{R/L} \equiv 1/2 \, (g^{v} \pm g^{a})$ couples only to right-handed/left-handed components. 
The second term stems from invariance under $\SUtwo$ rotations, and leads to an additional 
decay mode $V \to b\bar{d} + \bar{b}d$ (on top of $V \to t\bar{u} + \bar{t}u$ and $V \to \chi\chi$). 
\textbf{[Open point: do we just set the 2nd term to zero in this model? Justification?]}

\newthought{Implementation and notation}


This Section describes the notations used in the MadGraph model~\cite{MGmodel} convention, 
in term of the ones introduced in the previous Section.

The Madgraph model corresponds to the Lagrangian from~\cite{AndreaFuksMaltoni}. 
Each coupling constant of this dynamics can be set via the paramater card and 
the blocks which are relevant for the two models used for the experimental searches are described below.

\begin{enumerate}

\item Resonant scalar model described by the Lagrangian~\eqref{eq:lagrangianResonant}
  \begin{itemize}
  \item \texttt{AQS} and \texttt{BQS}: $3\times 3$ matrices (flavour space) fixing the coupling of the scalar $\phi^{\pm}$ ($S$ stands for scalar) and $down$-type 
    quarks ($Q$ stands for quarks), written in this note $g_{\phi u}$ or $a^{q}_{\mathrm{res}}$.
  \item \texttt{A12S} and \texttt{B12S}: $3\times 1$ matrices (flavour space) fixing the coupling of the fermion $\chi$ ($12$ stands for spin-$1/2$ fermion) 
    and $up$-type quarks, written in this note $g_{u \chi}$ or $a^{1/2}_{\mathrm{res}}$.
  \item particle name: the scalar $\phi^{\pm}$ is labelled $S$ and the fermion $\chi$ is $f_{met}$
  \end{itemize}  
  
\item Non-resonant vectorial model described by the Lagrangian~\eqref{eq:lagrangianNonResonantVector}
\begin{itemize}
\item \texttt{A1FC} and \texttt{B1FC}: $3\times 3$ matrices (flavour space) fixing the coupling of the vector $V$ 
  ($1$ stands for vector) and $up$-type quarks, written in this note $g_{Vu}$ or $a_{\mathrm{non-res}}$.
\item particle name: the vector $V$ is labelled $v_{met}$ and the fermion $\chi$ doesn't exist
\item the dark matter candidate $\chi$ is not implemented (this model assumes $\BR{V}{\chi\chi}=100\%$)
\end{itemize}

\end{enumerate}
% 
$A$ means vectorial coupling ($g^{v}$) and $B$ means axial coupling ($g^{a}$) and these two matrices 
are taken to be equal according to the chiral assumptions made above. 
The convention adopted follows \cite{ATLASmonotop} in defining 
a single number $a_{\mathrm{res}}$ ($a_{\mathrm{non-res}}$) 
for the resonant (non resonant) model, such as $(\ares^q)_{\mathrm{12}}=(\ares^q)_{\mathrm{21}}=(\ares^{1/2})_{\mathrm{3}}\equiv \ares$ 
(in order to have $d-s-S$ couplings, and $t-S-f_{met}$ couplings) 
and $(\anonres)_{\mathrm{13}}=(\anonres)_{\mathrm{31}}\equiv \anonres$ (in order to have $v_{met}-t-u$ couplings). 

\newthought{Parameter scan}

\textbf{[Open point - parameter scan studies go here.]}

 Which parameters impact the kinematics (this is the only relevant aspect form the experimental point of view)? 
 Some studies would be nice to put in this documents about:
 \begin{itemize}
  \item mediator mass
  \item mediator width: no effect (or parametrizable effects, plots are ready and need to be included)
  \item \textbf{which parameters} impact our experimental sensitivity? Which plane should be scanned?
 \end{itemize}

 What are the relevant numerical range to explore? First guess would be to follow the mono-top analysis.

\newthought{Parameter choices and cross sections}

\textbf{[Open point: update with new numbers]}

ATLAS has considered two models, a resonant and a non-resonant production, using only right-handed top quarks in the lepton+jets final state. The signal samples were produced with {\sc Madgraph5} v1.5.11 interfaced with {\sc Pythia} 8.175, using the MSTW2008LO Parton Distribution Function (PDF) set (lhapdf ID: 21000).
The mass of the top quark was set at 172.5 GeV. Dynamic renormalisation and factorisation scales were used.
The $\met$ particle mass was varied, and in the case of the resonant model the resonance mass was fixed at 500~GeV:
\begin{itemize}
\item Resonant model,  $\met$ particle mass: [0,100]~GeV in 20~GeV steps
\item Non-resonant model, $\met$ particle mass: [0,150]~GeV in 25~GeV steps, [200,300]~GeV in 50~GeV and [400,1000]~GeV in 100~GeV steps 
\end{itemize}

%\com{How to translate the monotop paper couplings to the notation of this note? }
The couplings $\ares$ and $\anonres$ are set at a fixed value of $0.2$.
In addition, two samples are produced for the resonant model for $\mfmet=100$~GeV,
with coupling strengths fixed at $\ares=0.5$ and $\ares=1.0$,
in order to check the effect of the resonance width on the signal event kinematics. 
The total width of the resonance varies quadratically with the coupling strength,
corresponding to a width of 3.5~GeV, 21.6~GeV, and 86.5~GeV at $\ares=0.2$, $\ares=0.5$, and $\ares=1.0$, respectively.

The number of free parameters is reduced by assuming $(\ares^q)_{\mathrm{12}}=(\ares^q)_{\mathrm{21}}=(\ares^{1/2})_{\mathrm{3}}\equiv \ares$
for the resonant model and $(\anonres)_{\mathrm{13}}=(\anonres)_{\mathrm{31}}\equiv \anonres$ for the non-resonant model,
all other elements of these coupling matrices being equal to 0.
For each model, the coupling parameter $\ares$ or $\anonres$ and the masses of the exotic particles are independent.

The cross-sections as well as the width of the resonance for the resonance model are shown in Table~\ref{tab:S1R_Xsec}.  
The cross-section is slowly decreasing when $m(f_{met})$ increases,
and the values do not differ by larger than 10\%, due to the similarity of the kinematics, in the chosen mass range.
\begin{table}[!htb]\centering
\begin{tabular}{l|r|r|r}
\hline \hline
% $\mathrm{m}(f_{\mathrm{met}})$ [GeV] & $\sigma\times\mathrm{BR(S\rightarrow t f_{met})}\times\mathrm{BR(t\rightarrow l^+\nu b)[pb]}$ & $\sigma\times\mathrm{BR(S\rightarrow t f_{met})}\times\mathrm{BR(t\rightarrow jjj)[pb]}$ & $\Gamma(S)$ [GeV]   \\
$m(f_{met})$ [GeV] & $\sigma_{lep}$~[pb] & $\sigma_{had}$~[pb] & $\Gamma(\Phi)$ [GeV]   \\
\hline \hline
0                        &  1.107              & 2.214               &  3.492  \\
20                       &  1.102              & 2.205               &  3.491  \\
40                       &  1.089              & 2.180               &  3.487  \\
60                       &  1.068              & 2.137               &  3.481 \\	
80                       &  1.039              & 2.078               &  3.472  \\
100                      &  1.001              & 2.003               &  3.461  \\
100 ($\ares=0.5$) &  6.091              &  12.13              & 21.63    \\
100 ($\ares=1.0$  &  21.77              &  43.72              &  86.52   \\
\hline \hline
\end{tabular}
\caption
{
Theoretical predictions for the product of the production cross-section of the
scalar resonance, the branching ratio of its decay into a top quark and the invisible particle,
and of the branching ratio of the top quark decay into a semi-leptonic ($\sigma_{lep}$) or fully-hadronic ($\sigma_{had}$) final state,
in the resonance model.
Values are given for a resonance of mass 500~GeV and for an effective coupling $\ares=0.2$ (except for two masses),
as a function of the mass $m(f_{met})$ of the neutral fermion.
The total widths $\Gamma(\Phi)$ of the resonance are also shown.
}
\label{tab:S1R_Xsec}
\end{table}


For the non-resonant case, the cross-sections are given in Table~\ref{tab:S4R_Xsec} and are calculated with $\anonres=0.2$.
The cross-section diverges when $m(v_{met})$ tends to 0~GeV.  However, when the mass is exactly 0~GeV the cross-section has a finite value,
due to the specificity of the propagator for this massless spin-1 boson.
\begin{table}[!htb]\centering
\begin{tabular}{l|r|r}
\hline \hline
% $m(v_{met})$ [GeV] & $\sigma\times\mathrm{BR(t\rightarrow l^+\nu b)[pb]}$ & $\sigma\times\mathrm{BR(t\rightarrow jjj)[pb]}$ \\
$m(v_{met})$ [GeV] & $\sigma_{lep}$~[pb] & $\sigma_{had}$~[pb] \\
\hline \hline
0                  &  96.03              & 192.4    \\
25                 &  359.0              & 717.9    \\
50                 &  113.4              & 226.9    \\
75                 &  59.86              & 119.5    \\
100                &  37.45              & 74.82    \\
125                &  25.35              & 50.68    \\
150                &  18.00              & 35.96    \\
200                &  9.662              & 19.28    \\
250                &  5.506              & 11.02    \\
300                &  3.328              & 6.656    \\
400                &  1.372              & 2.738    \\
500                &  0.6345             & 1.270    \\
600                &  0.3192             & 0.6354   \\
700                &  0.1698             & 0.3383   \\
800                &  0.09417            & 0.1883   \\
900                &  0.05472            & 0.1091   \\
1000               &  0.03259            & 0.06479  \\
\hline \hline
\end{tabular}
\caption
{
Theoretical predictions for the product of the production cross-section of the invisible vector $v_{met}$ and of a top quark,
and of the branching ratio of the top quark decay into a semi-leptonic ($\sigma_{lep}$) or fully-hadronic ($\sigma_{had}$) final state, 
in the non-resonance model.
Values are given for an effective coupling $\anonres=0.2$, as a function of the mass $m(v_{met})$ of the invisible state.
}
\label{tab:S4R_Xsec}
\end{table}

% \com{I think it might make more sense to have the joboption information in a public web site instead of
% adding all the details into the note. Reference only visible for ATLAS members: \\
% {\small https://svnweb.cern.ch/trac/atlasoff/browser/Generators/MC12JobOptions/trunk/gencontrol/MadGraphControl$\_$Monotop.py}}

\textbf{[Open point: systematic uncertainties]}
% \com{Question for DM forum:
% \begin{itemize}
%  \item Do we want to give more details about the Madgraph implementation, the couplings value in the param\_card, etc ... ?
%  \item I am not aware of any work on systematic variation due to scale, PDF choice, showering (Maybe some was done in the monotop analysis?). Then I am not completely what to put here.
% \end{itemize}
% }




\subsection{\texorpdfstring{Further W+\MET models with possible cross-section enhancements}{Further W+MET models with possible cross-section enhancements}}
\label{app:monoWExtramodel}
%Contributed by Yang Bai

As pointed out in Ref.~\cite{Bell:2015sza}, the mono-$W$ signature can probe the iso-spin violating interactions of dark matter with quarks. The relevant operator after the electroweak symmetry breaking is 
%
\begin{equation}
\frac{1}{\Lambda^2}\overline{\chiDM} \gamma_\mu \chiDM \left( \overline{u}_L \gamma^\mu u_L + \xi \bar{d}_L \gamma^\mu d_L \right) \,.
\end{equation}
%
Here, we only keep the left-handed quarks because the right-handed quarks do not radiate a $W$-gauge boson from the weak interaction. As the LHC constrains the cutoff to higher values, it is also important to know the corresponding operators before the electroweak symmetry. At the dimension-six level, the following operator
%
\begin{equation}
\frac{c_6}{\Lambda^2}\overline{\chiDM} \gamma_\mu \chiDM \,\overline{Q}_L \gamma^\mu Q_L 
\end{equation}
%
conserves iso-spin and provides us $\xi=1$~\cite{Bell:2015sza}. At the dimension-eight level, new operators appear to induce iso-spin violation and can be
%
\begin{equation}
\frac{c^d_8}{\Lambda^4}\overline{\chiDM} \gamma_\mu \chiDM \,(H\overline{Q}_L) \gamma^\mu (Q_L H^\dagger) 
+ \frac{c^u_8}{\Lambda^4}\overline{\chiDM} \gamma_\mu \chiDM \,(\tilde{H}\overline{Q}_L) \gamma^\mu (Q_L \tilde{H}^\dagger)  \,.
\end{equation}
% 
After inputting the vacuum expectation value of the Higgs field, we have 
\begin{equation}
\xi = \frac{c_6 \,+\, c_8^d\,v_{\rm EW}^2/2\Lambda^2}{c_6 \,+\, c_8^u \,v_{\rm EW}^2/2\Lambda^2} \,.
\end{equation}
% 
For a nonzero $c_6$ and $v_{\rm EW} \ll \Lambda$, the iso-spin violation effects are suppressed. On the other hand, the values of $c_6$, $c^d_8$ and $c^u_8$ depend on the UV-models. 

There is one possible UV-model to obtain a zero value for $c_6$ and non-zero values for $c^d_8$ and $c^u_8$. One can have the dark matter and the SM Higgs field charged under a new $U(1)^\prime$ symmetry. There is a small mass mixing between SM $Z$-boson and the new \Zprime with a mixing angle of ${\cal O}(v_{\rm EW}^2/M^2_{\Zprime})$. After integrating out \Zprime, one has different effective dark matter couplings to $u_L$ and $d_L$ fields, which are proportional to their couplings to the $Z$ boson. For this model, we have $c_6=0$ and 
 %
\begin{equation}
\xi = \frac{-\frac{1}{2} + \frac{1}{3} \sin^2{\theta_W} }{ \frac{1}{2} - \frac{2}{3} \sin^2{\theta_W}} \approx  -2.7 
\end{equation}
%
and order of unity. 

\subsection{Simplified model corresponding to dimension-5 EFT operator}

% +++++++++++++++++++++++++++++++++++++++++++++++++++++++++++++++++++++++++++++++++++++
%   Linda 11/5/15
% +++++++++++++++++++++++++++++++++++++++++++++++++++++++++++++++++++++++++++++++++++++

As an example of a simplified model corresponding to the dimension-5 EFT operator 
described in Section~\ref{sec:EFT_models_with_direct_DM_boson_couplings}, 
we consider a Higgs portal with a scalar mediator. Models of this kind
are among the most concise versions of simplified models that produce 
couplings of Dark Matter to pairs of gauge-bosons.  Scalar fields may couple directly to pairs of electroweak gauge bosons, 
but must carry part of the electroweak vacuum expectation value.  One may thus consider a simple model where Dark Matter couples to a a scalar 
singlet mediator, which mixes with the fields in the Higgs sector.
\begin{equation}
L\subset \frac{1}{2} m_s S^2 + \lambda S^2|H|^2 +\lambda^{'} S |H|^2 + y S \chiDM \overline{\chiDM}
\end{equation}
Where H is a field in the Higgs sector that contains part of the electroweak vacuum expectation value, 
S is a heavy scalar singlet and $\chiDM$ is a Dark Matter field. 
There is then an \schannel diagram where DM pairs couple to the singlet field S, 
which then mixes with a Higgs-sector field, and couples to W and Z bosons. 
This diagram contains 2 insertions of EW symmetry breaking fields, 
corresponding in form to the effective dimension-5 operator in Section~\ref{sub:EW_EFT_Dim5}.

\subsection{Inert two-Higgs Doublet Model (IDM)}\label{sec:i2HDM}

For most of the simplified models included in this report, the mass of
the mediator and couplings/width are non-trivial parameters of the
model. In these scenarios, we remain agnostic about the theory behind
the dark matter sector and try to parameterize it in simple terms.

We have not addressed how to extend the simplified models to realistic
and viable models which are consistent with the symmetries of the
Standard Model. Simplified models often violate gauge invariance which
is a crucial principle for building a consistent BSM model which
incorporates SM together with new physics. For example, with a new
heavy gauge vector boson mediating DM interactions, one needs not just
the dark matter and its mediator, but also a mechanism which provides
mass to this mediator in a gauge invariant way.

Considering both the simplified model and other elements necessary for a consistent theory is a next logical step. The authors of \cite{Belyaev:2015tap} term these Minimal
Consistent Dark Matter (MCDM) models. MCDM models are at the same time still toy models that can be 
easily incorporated into a bigger BSM model and explored via
complementary constraints from collider and direct/indirect DM search
experiments as well as relic density constraints. 

%We discuss this model here both on its own merits and as an example of the flexibility of the simplified model approach.

%The idea of an Inert two-Higgs Doublet Model (IDM) was introduced more than 30 years ago in \cite{IDMfirst}. The IDM was first proposed as a dark matter model in \cite{IDMnaturalness} and its DM phenomenology was further studied in \cite{IDMarchetype,
%ScalarMultiplet,IDMnewviable,IDMgammalines,}. 


The idea of an inert Two-Higgs Doublet Model (IDM) was introduced
more than 30 years ago in Ref~\cite{Deshpande:1977rw}. The IDM was
first proposed as a Dark Matter model in Ref.~\cite{IDMnaturalness} and its
phenomenology further studied in Refs.~\cite{LopezHonorez:2006gr,ScalarMultiplet,IDMnewviable,IDMgammalines,Dolle:2009fn,ATL-PHYS-PUB-2014-007,IDMnu1,IDMnu2,IDMpos,IDMVIB,Goudelis:2013uca,Belyaev:2015tap}. It is an extension of the SM with a second scalar 
doublet $\phi_2$ with no direct coupling to fermions.  This doublet has a discrete $Z_2$ symmetry, 
under which $\phi_2$ is odd and all the other fields are even. 
The Lagrangian of the odd sector is,
\begin{equation}
  \mathcal{L} = \frac{1}{2}(D_{\mu}\phi_2)^2 -V(\phi_1,\phi_2)
\end{equation}
with the  potential $V$  containing mass terms and $\phi_1 - \phi_2$
interactions:
  \begin{eqnarray}
    V &=& -m_1^2 (\phi_1^{\dagger}\phi_1) - m_2^2 (\phi_2^{\dagger}\phi_2) + \lambda_1 (\phi_1^{\dagger}\phi_1)^2 + \lambda_2 (\phi_2^{\dagger}\phi_2)^2    \nonumber  \\
      &+&  \lambda_3(\phi_2^{\dagger}\phi_2)(\phi_1^{\dagger}\phi_1)  + \lambda_4(\phi_2^{\dagger}\phi_1)(\phi_1^{\dagger}\phi_2) + 
          \frac{\lambda_5}{2}\left[(\phi_1^{\dagger}\phi_2)^2 + (\phi_2^{\dagger}\phi_1)^2 \right],
  \end{eqnarray}

where $\phi_1$ and  $\phi_2$ are SM and inert Higgs doublets respectively carrying the same hypercharge. These doublets can be parameterized as
\begin{equation}
  \phi_1=\frac{1}{\sqrt{2}}
  \begin{pmatrix}
    0\\
    v+H 
  \end{pmatrix}
  \qquad
  \phi_2= \frac{1}{\sqrt{2}}
  \begin{pmatrix}
    \sqrt{2}{h^+} \\
    h_1 + ih_2
  \end{pmatrix}
\end{equation}

In addition to the SM, the IDM introduces four more degrees of freedom coming from the inert doublet in the form of a $Z_2$-odd charged scalar $h^\pm$ and two neutral $Z_2$-odd scalars $h_1$ and $h_2$. The lightest neutral scalar, $h_1$ is identified as the dark matter candidate. Aspects of the IDM collider phenomenology have been studied in \cite{Burgess:2000yq, Andreas:2008xy, Arhrib:2013ela, Belyaev:2015tap,IDMnaturalness,IDMLEPII,IDMLHChinvfirst,IDMdileptons1,IDMtrileptons,IDMmultileptons,IDMhgaga1,IDMhgaga2,IDMposthiggs,IDMdileptonsII}. Its LHC signatures include dileptons \cite{IDMdileptons1,IDMdileptonsII}, trileptons \cite{IDMtrileptons} and multileptons \cite{IDMmultileptons} along with missing transverse energy, modifications of the Higgs branching ratios \cite{IDMhgaga1,IDMhgaga2,Goudelis:2013uca}, as well as $\slashed{E}_T + \rm{jet}$, Z, and Higgs and $\slashed{E}_T + \rm{VBF}$ signals (see Figs.~\ref{fig:fdmonojet1}--\ref{fig:fdvbf}).

\begin{figure}
\includegraphics[width=\textwidth]{figures/EW/i2HDM/fd-mono-j1.pdf} 
\caption{Feynman diagrams for $gg\to h_1 h_1+g$ process
contributing to mono-jet signature, adapted from \cite{Belyaev:2015tap}.}
\label{fig:fdmonojet1}
\end{figure}
\begin{figure}[htb]
\includegraphics[width=\textwidth]{figures/EW/i2HDM/fd-mono-j2.pdf} 
\caption{Feynman diagrams for $q\bar{q}\to h_1 h_2+g$ ($gq\to h_1 h_2+q$) process 
contributing to mono-jet signature, adapted from \cite{Belyaev:2015tap}.}
\label{fig:fdmonojet2}
\end{figure}
\begin{figure}[htb]
\includegraphics[width=\textwidth]{figures/EW/i2HDM/fd-mono-z.pdf} 
\caption{Feynman diagrams for $q\bar{q}\to h_1 h_1+Z$  process 
contributing to mono-Z signature, adapted from \cite{Belyaev:2015tap}.}
\label{fig:fdmonoZ}
\end{figure}
\begin{figure}[htb]
\includegraphics[width=\textwidth]{figures/EW/i2HDM/fd-mono-h1.pdf} 
\caption{Feynman diagrams for $gg\to h_1 h_1+H$  process 
contributing to mono-Higgs signature, adapted from \cite{Belyaev:2015tap}.}
\label{fig:fdmonoH1}
\end{figure}
\begin{figure}[htb]
\includegraphics[width=\textwidth]{figures/EW/i2HDM/fd-mono-h2.pdf} 
\caption{Feynman diagrams for $q\bar{q}\to h_1 h_2+H$  process 
contributing to mono-Higgs signature, adapted from \cite{Belyaev:2015tap}.}
\label{fig:fdmonoH2}
\end{figure}
\begin{figure}[htb]
\includegraphics[width=\textwidth]{figures/EW/i2HDM/fd-vbf.pdf} 
\caption{Diagrams for $qq\to qq h_1 h_1$ DM production in vector boson
fusion process, adapted from \cite{Belyaev:2015tap}.}
\label{fig:fdvbf}
\end{figure}

Based on the various LHC search channels, DM phenomenology issues and theoretical considerations, numerous works have proposed benchmark scenarios for the IDM, see e.g. \cite{IDMmultileptons,Goudelis:2013uca} while a FeynRules implementation (including MadGraph, CalcHEP and micrOMEGAs model files) was provided in \cite{IDMmultileptons}. An updated analysis of the parameter space has recently been performed in Ref.~\cite{Belyaev:2015tap}. 

The authors suggested to study mono-X  signatures that are relevant to model-independent collider DM searches,
and evaluated their rates presented below. They have implemented and cross-checked the IDM model into CalcHEP and micrOMEGAs,
with an implementation publicly available on the  ~\href{http://hepmdb.soton.ac.uk/hepmdb:0615.0189}{HEPMDB database}, including loop-induced $HHG$ and $\gamma\gamma H$ models. They propose an additional set of benchmark points, mostly inspired by mono-$X$ and VBF searches (Table.~\ref{tab:IDMbenchMarks}). Though the overall parameter space of IDM is 5-dimensional, once all relavant constraints are applied the parameter space relevant to a specific LHC signature typically reduces to 1-2 dimensional. In the mono-jet case, one can use two separate simplified models, a $gg \rightarrow h_1 h_1 + g$ process (via Higgs mediator) and a $qq \rightarrow h_1 h_2 + g (gq \rightarrow h_1 h_2 + q)$ process (through a Z-boson mediator) to capture the physics relevant to the search.
The cross sections for the various mono-$X$ and VBF signatures produced by this model are displayed in Fig.~\ref{fig:IDM_xsecs}.



\begin{table}[htb]
	\centering
	\begin{tabular}{|c||c|c|c|c|c|c|}
		\hline
		{\bf BM}                       &  {\bf 1}  & {\bf 2}  & {\bf 3}  & {\bf 4}  &  {\bf 5}  \\
		\hline\hline 
		$M_{h_{1}}$ (GeV)     & 48      	& 53 		& 70 		& 82 	&120 \\
		\hline
		$M_{h_{2}}$ (GeV)     & 55      	& 189 		& 77  		&  89  & 140 \\
		\hline
		$M_{h_{\pm}}$ (GeV)   & 130     	& 182 		& 200  	&  150  &  200 \\
		\hline
		$\lambda_{2}$         &  0.8    	& 1.0 		& 1.1 		& 0.9 	& 1.0 \\ 
		\hline
		$\lambda_{345}$       & $-0.010$ 	& $-0.024$  	& $+0.022$ 	& $-0.090$  & $-0.100$      \\
		\hline
		$\Omega h^2$          & $3.4 \times 10^{-2}$ & $8.1 \times 10^{-2}$  & $9.63 \times 10^{-2}$  & $1.5 \times 10^{-2}$  &  $2.1 \times 10^{-3}$ \\
		\hline 
		$\sigma_{SI}$ (pb)   & $2.3 \times 10^{-10}$ &  $7.9 \times 10^{-10}$  & $5.1 \times 10^{-10}$  & $4.5 \times 10^{-10}$  &  $2.6 \times 10^{-9}$ \\
		\hline 
		$\sigma_{LHC}$ (fb)     & $1.7 \times 10^{2}$ &  $7.7 \times 10^{2}$  & $4.3 \times 10^{-2}$  & $1.2 \times 10^{-1}$  &  $2.3 \times 10^{-2}$ \\
		\hline\hline
	\end{tabular}
	\caption{Five benchmarks for IDM in  ($M_{h_{1}},M_{h_{2}},M_{h_{\pm}},\lambda_{2},\lambda_{345}$) parameter space.
		We also present the corresponding relic density ($\Omega h^2$), the spin-independent cross section for DM scattering on the proton ($\sigma_{SI}$),
		and the LHC cross section at 13 TeV for mono-jet process $pp\to h_1,h_1+jet$ for $p_T^{jet}>100$~GeV cut ($\sigma_{LHC}$).}
	\label{tab:IDMbenchMarks}
\end{table}


\begin{figure}[htb]
	\includegraphics[width=\textwidth]{figures/EW/i2HDM/i2HDM_crossSections.pdf} 
	\caption{LHC cross section at 13~\tev for various signatures, from \cite{Belyaev:2015tap}.}
	\label{fig:IDM_xsecs}
\end{figure}

\clearpage