%%%ATLAS prefers the following 5 lines (comment CMS line out)
\documentclass[a4,debug,notitlepage,nobib]{tufte-handout}
\usepackage[svgnames]{xcolor}
\usepackage{hyperref}
\usepackage{lineno}
\usepackage{titling}
\linenumbers

%%%CMS prefers the following line (comment ATLAS lines out)
%\documentclass[notitlepage,11pt,twoside,a4paper,debug]{dmfm}
%Also, uncomment abstract lines

\usepackage{rotating}
\usepackage{xspace}
\usepackage{graphicx}
\usepackage{epstopdf}

\usepackage{slashed}
\usepackage{amsfonts}
\usepackage{amsmath,amssymb}
\usepackage{slashed}
\usepackage{feynmp}
\usepackage{subfig}
\DeclareGraphicsRule{*}{mps}{*}{}

\setkeys{Gin}{width=\linewidth,totalheight=\textheight,keepaspectratio}
\usepackage{mathtools} % extended mathematics
\usepackage{booktabs} % book-quality tables
\usepackage{units}    % non-stacked fractions and better unit spacing
\usepackage{multicol} % multiple column layout facilities
\usepackage{fancyvrb} % extended verbatim environments
\usepackage{fancyhdr}
\usepackage{refcount}
\usepackage{calc}
\usepackage{lastpage}

% \usepackage{natbib} % for \citep
% create a dummy file (shell command "touch moderntex") to turn on some features that don't work on lxplus
\IfFileExists{moderntex}{
  \usepackage[protrusion=true,expansion=true,tracking=true,kerning=true,spacing=true]{microtype}
}{}
\fvset{fontsize=\normalsize} % default font size for fancy-verbatim environments


% Prints an asterisk that takes up no horizontal space.
% Useful in tabular environments.
\newcommand{\hangstar}{\makebox[0pt][l]{*}}
\newcommand{\openepigraph}[2]{%
  %\sffamily\fontsize{14}{16}\selectfont
  \begin{fullwidth}
  \sffamily\large
  \begin{doublespace}
  \noindent\allcaps{#1}\\% epigraph
  \noindent\allcaps{#2}% author
  \end{doublespace}
  \end{fullwidth}
}
\newcommand{\blankpage}{\newpage\hbox{}\thispagestyle{empty}\newpage}

\usepackage{xhfill}% http://ctan.org/pkg/xhfill
\newcommand{\ditto}[1][.4pt]{\xrfill{#1}~''~\xrfill{#1}}
\newcommand{\mdm}{\ensuremath{m_{DM}}\xspace}
\newcommand{\mmed}{\ensuremath{m_{med}}\xspace}
\newcommand{\gq}{\ensuremath{g_{q}}\xspace}
\newcommand{\gdm}{\ensuremath{g_{DM}}\xspace}

\makeatletter

\newif\ifATLAS
\newif\ifCMS
%%Uncomment this in the case of ATLAS-only details
\ATLAStrue
%%%Uncomment this in the case of CMS-only details
% \CMStrue

\title{Short summary of ATLAS+CMS Dark Matter Forum Recommendations For Signal MC} 
\author{ATLAS+CMS Dark Matter Forum}
\date{\today}

\begin{document}

%%\morefloats
\setcounter{secnumdepth}{3} % turn on section numbering, for now (3=subsubsection)

\maketitle

% \blankpage
% \abstract{This document outlines the choices made for the MC production and parameter scans
% for the simplified models recommended by the ATLAS/CMS Dark Matter Forum. 
% This summary is aimed at defining the models and parameters
% for the upcoming signal Monte Carlo production by the two collaborations. 
% Motivations for these choices, reinterpretation details, and a full bibliography 
% will be provided in an upcoming document concluding the works of the Forum.}

This document outlines the choices made for the MC production and parameter scans
for the simplified models recommended by the ATLAS/CMS Dark Matter Forum. 
This summary is aimed at defining the models and parameters
for the upcoming signal Monte Carlo production by the two collaborations. 
Motivations for these choices, reinterpretation details, and a full bibliography 
will be provided in an upcoming document concluding the works of the Forum.

\section{Prioritized lists of simplified models for MET+X searches}

\subsection{Recommended models for all MET+X analyses}
\label{sec:RecommendedModelsAllAnalyses}

For all MET+X analyses, simplified models should replace EFTs as the
highest-priority dark matter interpretation. The consensus of the
Forum is that all of these analyses should consider a set of simplified models involving 
a single, mediating particle with mass \mmed. 
The two underlying choices for the model recommendations by the ATLAS/CMS DM Forum are:
\begin{enumerate}
 \item the dark matter particle is a Dirac fermion of mass \mdm;
 \item only interactions consistent with Minimal Flavor Violation are allowed.
\end{enumerate}

The highest priority models discussed within the Forum are:
\begin{itemize}
\item[a.] spin-1 $s-$channel mediators with vector or 
  axial-vector coupling to SM quarks \gq
  and vector or axial-vector coupling to a Dirac fermion WIMP \gdm;
\item[b.] spin-0 $s-$channel mediators with scalar or 
  pseudo-scalar coupling to SM quarks \gq and scalar or pseudo-scalar
  coupling to a Dirac fermion WIMP \gdm;
\item[c.] and colored spin-0 $t-$channel mediators with 
  coupling $g$ at the vertex between quark, WIMP 
  and mediator. 
\end{itemize}

In all three cases, the coupling of the mediator to quarks \gq is
identical for all three generations. 
In the case of the scalar and pseudo-scalar models, we assume a separate 
Yukawa coupling ($y_q$ = $m_q/v$) between the quarks and the mediators
multiplying \gq; the specific case of scalar mediator decaying
into two top quarks is discussed in Section~\ref{sub:SPttbar}. The
coupling between mediator and WIMP does not have any Yukawa structure.

There are thus four parameters in the first two
models (\mdm, \mmed, \gq, and \gdm) and three parameters
in the last model (\mdm, \mmed, $g$). We assume that no
additional visible or invisible decays contribute to the width of the
mediator. We provide formulae for calculation of the ``minimal
width'' for any given choice of the four parameters. Early run 2 MET+X
analyses are not sensitive to changes in the kinematic distributions that 
result from increasing the width up to $\Gamma \approx \mmed$. 
The exceptional case of a narrow, off-shell mediator is addressed later.

All types of $s-$channel mediators are important, and ideally an analysis 
would have to include all cases. On the other hand, the vector and axial-vector models 
(including those with mixed couplings) produce nearly indistinguishable kinematic distributions 
for the MET+X analyses, and the scalar and pseudo-scalar models are also very similar. 
If one cannot generate all models, one could generate the full grid of mass points for only one of each 
type (for example, the pure axial vector and pure pseudo-scalar models, motivated
respectively by complementarity with direct and indirect detection and galactic center excess 
interpretations). 

%TODO: CD: can one have S/PS mixed couplings?
% Pure axial-vector (plus vector-axial and axial-vector combinations) 
% and pseudo-scalar versions of the $s-$channel models are
% also available and have been studied in the same detail. These studies
% show that the respective differences with the vector and scalar versions
% are small but in some cases non-negligible. 
% Therefore it is desirable to include these models as well, 
% but the pure axial-vector (as it's a spin-independent operator)
% and scalar versions should take priority. 
% [AB: switch the priority to favor
% spin-dependent models, since this is a strength? CD: Yes, but
% we should be careful not to make the mistake for D7 that is outlined 
% in http://arxiv.org/abs/1402.1173 (unfortunately no work for simplified models 
% is available yet). Received an answer, D8 (pure axial vector) is fine
% because it's at loop level already.]

Studies of the $s-$channel models include comparisons of the kinematic
distributions across a scan of each of the four parameters. From these
studies, we recommend generation of a minimal set of a total of approximately 30 points 
per simplified model for the monojet analysis, as shown in Table~\ref{tab:parameterScanGeneralModels},
combinations of parameter choices in order to capture the full variety
of shapes possible in the models.

%%Combined table
% \begin{sidewaystable}[!h]
\begin{table}[!h]
\centering
\begin{tabular}{| l |r r r r r r r r r r|}
\hline
\multicolumn{1}{|c|}{\mdm (${\rm GeV}$)} & \multicolumn{10}{c|}{\mmed (${\rm GeV}$)} \\
\hline
 %this row is for a cros$s-$check of all the points at 1 GeV, where the LHC is more sensitive than DD/ID
 1             &         10  & 20 & 50 & 100 & 200 & 300 & 500 &         1000  & \textbf{2000}          &         5000  \\
 %the following rows have only the diagonal + one point above/below

 10   	       &         10  & 15 & 50 & 100 &     &     &     &               &                        &         5000  \\
 50            &         10  &    & 50 &  95 & 200 & 300 &     &               &                        &         5000  \\
 150           &         10  &    &    &     & 200 & 295 & 500 &               &                        &         5000  \\
 500           &         10  &    &    &     &     &     & 500 &          995  & \textbf{\textit{2000}} &         5000  \\
 \textit{1000} & \textit{10} &    &    &     &     &     &     & \textit{1000} & \textbf{\textit{1995}} & \textit{5000}  \\
\hline
\end{tabular}

\caption{Simplified model benchmarks for all $s-$channel simplified models (spin-1 and spin-0 mediators 
decaying to Dirac DM fermions taking the minimum width for \gq = \gdm = 1).
Points in \textbf{bold} are only generated for the vector/axial vector cases, while points in 
\textit{italics} are generated for the monojet analysis 
but not for the search including heavy quarks. This table corresponds to 29 points for monojet vector/axial vector models,
26 points for monojet scalar/pseudoscalar models and 24 points for $t \bar{t}$+MET scalar/pseudoscalar models.}

\label{tab:parameterScanGeneralModels}
% \end{sidewaystable}
\end{table}

Namely the bulk of the grid scan for
the $s-$channel models will cover from 10 GeV to 2 TeV (10 GeV to 1 TeV)
in mediator mass for the vector and axial vector (scalar and pseudoscalar) models,
and from 1 GeV to 1 TeV in WIMP mass. A high-mediator-mass point is added for 
each of the WIMP masses in order to provide an EFT limit for theory reinterpretation
and comparisons with previous results. 

For very heavy mediators, changing
the minimal width (by changing the coupling) also affects the signal
kinematic distributions. It is unclear whether the MET+jet analysis
would be able to observe these signals during Run 2; therefore we
recommend a more limited scan which still allows further studies.
For the highest mediator mass points for the vector and axial vector models, 
signals should be generated with couplings in a grid of $g_{DM} = g_{q} = $ 1.0, 0.3, 0.5, 1.45,
while for the scalar and pseudoscalar signals the recommended grid is
$g_{DM} = g_{q} = $ 1.0, 0.5, 1.0, 2.0, 3.0.

% \textbf{[Point being finalized: What values of couplings should we suggest to scan? 
% We could have a limited scan of the Bristol workshop:
% weak, semi-weak, strong and maximum coupling based on width~mass for 1 TeV mediator,
% leading to a scan on $g_{DM} = g_{q} = $ 1.0, 0.3, 0.5, 1.45, 
% but we may need a finer granularity.}

% \begin{itemize}
%  \item 32 grid points in the DM/mediator mass plane
%  \item 6 grid points in the quark-mediator coupling \gq, 
%  to allow testing widths larger than the minimal width. 
% \end{itemize}

% > L54:  How about a table, and we were thinking 32.
% > M_DM    M_MED (in units of GeV)
% > 1            10, 50, 100, 150, 200, 300, 500, 1000, 1500
% > 10                50, ....
% > 100                     100, ...
% > 200                                    200, ...
% > 500                                                    500, ....
% > 

Results for other choices of parameters can be related to
these through simple rescalings of the total rate, using cross-section
formulae that will be provided by the forum.

The parameter space of the $t-$channel model 
has not yet been studied in the same detail as the
$s-$channel models, but a proposal will be available for the full Forum report. 
%[AB: Comment on the relationship to SUSY parameter space]. 
This model is parallel to, and partially motivated by, the
squark of the MSSM. Since the $t$-channel model can produce signatures 
similar to those arising from squark production in the MSSM, investigations 
are needed of what ranges of $g$, \mmed, and \mdm are already excluded.
Since the relevant coupling is arbitrary in the $t$-channel model, while
it is weak-scale in the MSSM, additional production topologies need to
be considered. The coordination of a generic $t$-channel scan with
MSSM scans would need more thought.

%CD this was text answering Rizzo. 
% This model is parallel to, and partially motivated by, the
% squark of the MSSM, with the difference that in the MSSM the 
% coupling of quark-squark-WIMP is small. 
% Since the relevant coupling is arbitrary in the $t$-channel model, while
% it is weak-scale in the MSSM, additional production topologies need to
% be considered.   The coordination of a generic $t$-channel scan with
% MSSM scans needs more thought.
% Since the $t-$channel model can produce signatures 
% similar to those arising from squark production in the MSSM, investigations 
% may be needed of what ranges of $g$, \mmed, and \mdm are already probed
% by other searches. The mediator can radiate a SM object, such as a jet or gauge boson, 
% thus providing three separate MET+X diagrams which must be considered together in
% calculations. This model can also give a signal in the di-jet + MET
% channel when, for example, a WIMP is exchanged in the $t-$channel to
% produce a pair of mediators, each of which decays to a jet and a DM particle.

\subsubsection{Technical implementation of the $s-$channel models} 

There are several matrix element implementations of the $s-$channel
vector and scalar mediated DM production.
They are available in POWHEG, MadGraph
and also MCFM. 
We recommend the POWHEG implementation of DM pair production +1 parton at NLO (the DMV model), 
and the associated theoretical uncertainties 
computed by POWHEG. 
% CD: After a discussion with SarahM, SM and SL, removed the line
%``For scalar models, the MCFM implementation
% is also acceptable.'', mostly because it would not be
% possible to synchronize the theoretical and PDF uncertainties
% as easily between the two collaborations]
POWHEG provides a set
of weights in the LHE v3 format, which will be propagated through the
ATLAS and CMS software to allow a determination of signal uncertainties
without generating additional sets of samples. A prescription for
determining these uncertainties from the weights is being provided
through the forum by the POWHEG authors.
For ATLAS and CMS, the necessary version of POWHEG is centrally available 
(V2.0 in both cases, revision 3049 for CMS and revision 3033 for ATLAS).

\textbf{[Point being finalized: The recommendations for assessing systematic uncertainties has to be approved by 
both collaborations. Once ready, ATLAS and CMS should discuss whether this can be harmonised.]}

\subsubsection{Technical implementation of the $t-$channel model} 

The $t-$channel model is implemented in MadGraph. 
It is recommended
to compute at LO and apply matching up to two additional partons
beyond the Born level
process. Theoretical uncertainties are to be calculated using
the MadGraph SysCalc package, which performs scale and PDF variations
in a similar manner as POWHEG.
The relevant parameter card can be found on the Forum
SVN repository~\cite{ForumSVN_TChannel}. 
% \textbf{[Point being finalized: put MadGraph card file on SVN]}

\subsection{Specific models for analyses of MET+b quark(s)}

\subsubsection{Single $b$+MET}

We consider a simplified model for single-$b$-jet signatures, 
with a colored scalar mediator coupling to Dark Matter and a $b-$quark. 
This model is described in Ref.~\cite{Agrawal:2014una}. 
The model is similar to the MSSM
with a light bottom squark and neutralino, and is thus a flavor-specific
example of a $t$-channel model. 
Preliminary studies 
show that generation of a grid of roughly 10 points in DM mass, 15 points in mediator
mass and two coupling values is sufficient to cover the parameter
space probed by the LHC with early searches. 

This model is available in MadGraph, and the relevant card files are 
available in the Forum SVN repository~\cite{ForumSVN_DMSingleB}.

\subsubsection{$b \bar{b}$+MET}

As noted earlier, the $t$-channel model produces both mono-jet and
di-jet +MET signatures after applying cuts.  In the case when
only a $b$-flavored mediator is accessible, both mono-$b$ and
$b\bar b$+MET signatures are possible. 
The details for the implementation of these
models is still under investigation and will be provided for the full Forum report.

\subsection{Specific models for analyses of MET+top(s)}
\label{sub:SPttbar}

\subsubsection{$t \bar{t}$+MET}

The pair production of top-flavored resonances that decay to
a top quark and DM is the subject of intensive study by the
experimental collaborations, and are not in the purview of this Forum.

Instead, we focus on the implications of the scalar/pseudo-scalar
model introduced earlier, with the mediator produced in association
with a $t\bar t$ pair.

As mentioned in Sec.~\ref{sec:RecommendedModelsAllAnalyses}, we assume 
mass dependent Yukawa couplings to quarks for spin-$0$ scalar
and pseudoscalar mediators. Couplings to top quarks will dominate, 
producing a tree-level signature of a $t \bar{t}$ pair
in association with missing transverse momentum from the two DM particles.
The parameter scan for this model is based on three different kinematic
regimes: the case of a very heavy mediator with \mmed $> 2 m_{top}$, 
the case of an on-shell mediator below the $t\bar{t}$ threshold
($\mmed >  \mdm, \mmed < 2 m_{top}$) and the case of an off-shell mediator below
the $t\bar{t}$ threshold (\mmed $< 2 m_{DM}$, \mmed $< 2 m_{top}$). 
Large kinematic differences are found between on-shell and off-shell mediator
regimes, while the total width changes slope depending on whether one
is above or below the ttbar threshold. 
Scalar and pseudoscalar models show kinematic differences as the mediator mass 
is reduced, due to the difference in coupling of the top pair to the mediator. 
These considerations lead to the choice of the grid scan shown in 
Table~\ref{tab:parameterScanGeneralModels}, harmonised with the choices for
the generic models and signatures, to be generated for both scalar and pseudoscalar
cases separately.

The effect of varying the couplings (and therefore the mediator width) 
is small in the on-shell regime, but is significant in the case of
very small or very large couplings and large mediator masses, 
with the latter effect due to parton distribution function suppression. 
LHC searches will be sensitive to those masses after roughly 30 fb$^{-1}$ of data. 
A table of cross-sections will be provided, for several values of coupling strengths,
to aid reinterpretation and validate the suggested scaling of the cross-sections
that substitutes the scan in couplings for early searches. 

\paragraph{Technical implementation of models for $t \bar{t}$+MET}

The implementation of this model is available in POWHEG, MadGraph
and MCFM. In the case of searches with heavy quarks in the final state
that are sensitive only to the tree-level mediator decays into DM particles
in association with two top quarks, the MadGraph implementation is suggested
for Run-2 searches. The relevant parameter cards are available in the Forum 
SVN repository~\cite{ForumSVN_DMTTBar}.

% \textbf{[Point being finalized (write-up): do we have comparisons between 
% POWHEG (monojet) and MadGraph (ttbar)?]}

\subsubsection{Single top+MET}

Models of single top+MET production are different from the other models under consideration, 
since they truly involve a single top quark in the final state.

Two simplified models, described in Refs.~\cite{Andrea:2011ws,Boucheneb:2014wza}, are recommended 
for the signature of a single top quark+MET~\footnote{The non-resonant production 
contributes to both both MET+t and MET+tt final states.}:

\begin{itemize}
\item $s-$channel production of a colored scalar resonance, decaying into a top quark and a fermion
that can decay in two DM candidates;
\item $s$- and $t$- channel non-resonant production of a top quark and a color singlet, vector mediator which then decays invisibly.
\end{itemize}

The parameters for the $s-$channel resonant production are:
\begin{itemize}
 \item the couplings of the scalar resonance to down-type quarks, all considered equal and denoted with $a_{res}^q$;
 \item the couplings of the DM particle to up-type quarks and to the mediator, denoted with $a_{res}^{1/2}$;
\end{itemize}
The resonance width is computed according to the couplings, assuming no other decays in addition
to those in DM and quarks. Fixing the branching fraction to DM particles to 100\% would lead 
to a single value for the coupling of the resonance to DM particles. $a_{res}^{1/2}$ 
may however become a parameter to be scanned, after further studies on whether changing 
the resonance width leads to significant changes in the kinematics of the model. 

The only relevant parameters considered for the $s-$ and $t-$channel non-resonant production 
are the coupling of the vector mediator to up-type quarks, all considered equal and denoted with $a_{non-res}$. 
The width of the mediator does not affect the model kinematics. 
A proposal for the scan in the parameters of these models will be prepared after further study 
from the ATLAS and CMS analysers, possibly beyond the timescale of the Forum. 

\paragraph{Technical implementation of the single top+MET models} 

Card files for MadGraph are provided on the Forum SVN repository~\cite{ForumSVN_EWMonoTop}.

% \textbf{[Point being finalized: need to put this model in SVN.]}

% \ifATLAS 
% The ATLAS implementation of this model using on-the-fly MadGraph
% will follow the pilot request for the monophoton D5 operator~\cite{ATLAS_PowhegPythiaMC15Test}. 
% \fi
% 
% \ifCMS
% \textbf{[Point being finalized: Add here implementation details for CMS..]}
% \fi



\subsection{Specific models for MET+EW boson searches}

Searches with an electroweak boson (photon, Z, Higgs or W) in the
final state should include the general models recommended in
Section~\ref{sec:RecommendedModelsAllAnalyses}. In addition to these,
we recommend models where an electroweak boson participates in
the dark matter interaction itself, rather than appearing as final
state radition.

Though the requisite set simplified models have not yet been fully
developed by the theoretical community, it is here especially that
searches in MET+photon, W, Z, or H may play a more important role than
in the general models considered above. Until all necessary simplified models
are developed, we must recommend use of EFT contact operators
in some cases, provided the region of validity is carefully studied
following the relevant recommendations in Section~\ref{sec:eft}.

% \begin{description}
%  \item Models with direct couplings of new mediators to bosons, e.g. by coupling 
%    the Higgs boson to a new scalar. 
%  \item Models including an EFT contact operator, where the boson is
%    directly coupled to DM, leading to a three-body final state (the
%    boson and two DM particles).
% \end{description}

\subsubsection{MET+photon,Z,W} 

These searches should consider an
effective $VV\chi\chi$ vertex, with two electroweak bosons and two WIMPs,
via an EFT contact operator. Such operators are available for scalar
and fermion WIMPs and $VV=ZZ,Z \gamma,WW,\gamma \gamma$ (so-called
dimension 4, 5, and 7 operators~\cite{Carpenter:2012rg,
Crivellin:2015wva}). We prioritize the Dirac fermion WIMP (dimension
7) operators, as those have been studied more widely and yield
kinematic distributions that are distinct from any of the simplified
models described above.  Correlations between the different
MET+photon, Z and W signatures are accounted for by the use of two
model parameters encapsulating the various couplings ($k_1$ and $k_2$). 
Since only the cross section, and not the kinematic
distributions, depend on the value of those parameters, we recommend
the generation of these benchmarks using only a single value of these
parameters. No difference in the kinematics is seen between the scalar
and pseudoscalar models considered.  This leads to the parameter scan
for this benchmark to correspond to 8 different DM masses: 
1, 10, 50, 100, 200, 400, 800 and 1300 GeV. The scale of the EFT
for the generation is fixed at 3 TeV, and the couplings are
set to $k_1 = k_2$ = 1. 

\paragraph{MET+W enhancements} The $s-$channel model mentioned in
Subsection~\ref{sec:RecommendedModelsAllAnalyses} has previously been
simulated using different coupling parameters between the mediator and
the up and down quarks, leading to an increase in cross-section for W
boson production. It has recently been shown \cite{Bell:2015sza} that
this is due to a violation of gauge invariance. For this reason, we do
not recommend this practice anymore, and restrict the grid scan to
equal couplings between the quarks. Simplified models overcoming this
issue are under development, beyond the timescale of the Forum.
Meanwhile, the $t-$channel model of the previous subsection may be
used to obtain different couplings and interference, but those studies
are left for a longer timescale.

\paragraph{Technical implementation of models with EW bosons in the final state} 

These models are generated at leading
order with MadGraph 2.2.2, using Pythia8 for the parton shower.
Parameter cards can be found on the Forum SVN repository~\cite{ForumSVN_EWEFTD7}.

\subsubsection{MET+Higgs} 

These searches should also consider models with an
effective $VV\chi\chi$ vertex, with two Higgs bosons and two WIMPs coupled
via an EFT contact operator. Such operators are available for scalar
and fermion WIMPs at dimension 5~\cite{Carpenter:2013xra}. The
recommended parameter scan and implementation follows that of the $VV=ZZ,Z \gamma,WW,\gamma \gamma$
cases, and the model can be found on the Forum SVN repository~\cite{ForumSVN_monoHEFTD5}.

Two benchmark simplified models \cite{Carpenter:2013xra} 
are recommended for MET+Higgs searches:
\begin{itemize}
 \item A model where a vector mediator ($Z'$) is exchanged in the $s-$channel, 
 radiates a Higgs or a $Z$ boson and decays into two DM particles. 
 \item A model where a scalar mediator couples to the SM only 
 through the SM Higgs and decays to two DM particles. 
\item A model where a vector $Z^\prime$ is produced resonantly and decays into a Higgs boson
plus an intermediate heavy pseudoscalar particle $A^0$, in turn decaying into two DM particles. 
 	
\end{itemize}

Preliminary studies of these models show that the parameter scan for 
these models (on DM mass, mediator mass, mediator couplings and
width, mixing of the mediator with the Z boson) can be reduced to 
a scan in DM mass and mediator mass that follows Table~\ref{tab:parameterScanGeneralModels} while keeping 
the other model parameters fixed to the following values: 
\begin{itemize}
 \item coupling between mediator and SM Higgs boson $g_{hZ'Z'}/ \mmed$ = 1
 \item mediator-DM coupling \gdm = 1
 \item mediator-quark $g_{q}$ = 1/3
 \item mixing angle between the new baryonic Higgs in this model and the SM Higgs $sin(\theta)$ = 0.3
\end{itemize}
for the vector mediator model, the following values: 
\begin{itemize}
 \item Yukawa coupling to DM $y_{DM}$ = 1
 \item new physics coupling between scalar and SM Higgs $b$ = 3
 \item mixing angle $sin(\theta)$ = 0.3
\end{itemize}
for the scalar model. 
For the scalar model, the parameter space scanned spans 10 points in $M_{Z^\prime}$, and $M_{A^0}$, with $M_{A^0} < M_{Z^\prime}-m_h$,
comprising 24 points, as no dependence is observed on the other model parameters.  

% \textbf{[Point being finalized: Update after wider discussion within the Forum, 
% maybe mention correlations with diagrams yielding a MET+jet signature.]}

\paragraph{Technical implementation of models with Higgs bosons in the final state} 

These models are generated at leading
order with MadGraph 2.2.2, using Pythia8 for the parton shower.
The MadGraph implementations of those models can be found on the Forum SVN 
repository~\cite{ForumSVN_EWMonoHiggs}.


% \section{Considerations on presentation of collider results}\label{sec:presentation}

% We suggest the following to collider searches, when presenting results 
% from the recommended benchmarks: 
% \begin{itemize}
% \item Provide limits in collider language, on fundamental parameters of
% interaction: couplings and masses of particles in simplified model.
% \item Translate limits to non-collider language, for a range of
% assumptions in order to convey a rough idea of the range of
% possibilities.
% \item Provide all necessary material for theorists to reinterpret simplified
% model results as building blocks for more complete models (e.g. signal cutflows,
% acceptances, etc). 
% \end{itemize}

% A full discussion of the presentation of collider results is left to 
% work beyond this Forum. This discussion should involve ATLAS and CMS, the theory community
% and the Direct and Indirect Detection communities. 


\subsection{Prioritization of parameter scans and external constraints}

The prioritization of parameter scans according to existing constraints 
has been widely discussed. It is generally difficult to map
collider results to results of other searches for dark matter, such as
direct detection cross-section. This mapping is highly
model-dependent, and in order to do this with certainty one must have
a fully specified theory of dark matter interactions. 
% At present,
% using a single benchmark at a time leads to many unwarranted guesses
% and does not arrive at such a theory.  More than one flavor of dark
% matter may exist, multiple or different interactions may be relevant,
% etc.

The problems that arise when mapping collider results on to the
results of other experiments, such as the direct detection scattering
cross-section, arise with equal difficult in the reverse direction,
when attempting to apply results from non-collider searches to
collider models. One should consider non-collider and relic density
constraints only with a complete theory in mind; for this reason the
experiments should provide information on the widest sensible range of
parameters so that others who are interested in specific models can
apply the constraints within their model of choice.

Another type of constraint 'external' to the MET+X analyses comes from
the LHC experiments themselves. Resonant signals of dark matter
mediators are a key difference between simplified models and the EFT
scenario. Direct searches for these mediators, for example in dijet,
ttbar, and dilepton final states, are thus a crucial part of the LHC
search program. These are discussed in the forum report but specific
recommendations for them are beyond the scope of the document.  It is
tempting to apply the constraints from these mediator searches to
reduce the parameter space studied in the MET+X searches, but one
should remember that the proposed simplified models are intended as
representative benchmarks. There is a wide consensus amongst the
participants of the forum that these constraints cannot be applied to
the parameter scans in any rigorous quantitative way, without a more
complete theory of dark matter. However, even if the parameter scans
and the searches are not optimized with those constraints in mind, 
we intend to make all information available to the community to exploit
the unique sensitivity of colliders to all possible 
Dark Matter signatures. 

\section{Recommendation for contact operator interpretation}
\label{sec:eft}

Except in the cases explicitly mentioned above, EFT contact operators
should no longer be a target for optimization: in general there is a
poor relationship between the kinematic distributions predicted by the
EFTs and the types of signals to which Run 2 analyses will be
sensitive. On the other hand, many theorists participating in the
forum insist that the information conveyed by EFT limits is
valuable. %consult roni et al%
For the V/A and S/P simplified models, the highest mediator mass has
been chosen to provide an equivalent of the D5/D8 and D1~\footnote{Following 
the notation of Ref.~\cite{Goodman:2010ku}} EFT interpretations, 
which are the limiting-cases of these models. 
We recommend to cross-check the kinematics of the highest-mass simplified model point 
with the corresponding contact interaction benchmark limit, after having ensured 
the robustness of the contact interaction benchmark point tested (e.g. with the ratio 
of valid events $R$ defined in~\cite{Aad:2015zva}). 

In case limits on other EFT operators are requested by theorists, and they
cannot be obtained by suitable combinations of simplified models, the
forum proposes two truncation procedures to ensure the EFT prediction
is defensible:

\begin{itemize}
\item for cut and count analyses (all of the searches done so far), a very conservative truncation for EFT 
limits from Ref.~\cite{Racco:2015dxa} is proposed as a modification of the approach 
ATLAS has used for 8 TeV papers~\cite{Aad:2015zva}
\item for shape-based analyses, the references above may be used to obtain the criteria that 
have to be applied to discard generated events, rather than rescaling the limit as done
for cut and count analyses. 
\end{itemize}

With Run 1 analyses, both experiments are very familiar with
simulation of the EFT contact operators. The forum has focused on
validity procedures that may be applied at analysis-level after sample
generation. Thus, so long as samples of contact operators are actually
produced where required, the forum recommendations here have no direct
bearing on MC requests.

Whenever a UV completion is not available, EFT results can still
be a source of useful information as in the case of the contact operators 
for boson+MET studies. 
However, in this case we can only naively control the validity of the EFT operator. 
Given our ignorance of
the actual kinematics, 
the truncation procedure suggested for this purpose
is the one described in ~\cite{Racco:2015dxa},
as it is independent from any UV completion details. 

% Still, at a lower priority than the simplified models, an EFT interpretation may be 
%included for purposes of comparison with prior results.

\bibliographystyle{alpha}
\bibliography{summary}

\end{document}
