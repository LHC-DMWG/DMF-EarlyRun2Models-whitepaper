\section{Scalar s-channel Mediator}
\label{sec:scalar}
 
\hto{One of the most simple UV complete extension of the effective field theory approach is the addition of a scalar mediator between DM and SM.}
\sout{A scalar particle mediator can be a very simple addition to the SM}.  {\sout If it is chosen} \hto{In case this mediator is as a gauge singlet} it can have tree-level interactions with a singlet DM particle that is either a Dirac or Majorana fermion, or DM that  is a scalar itself. The spin-$0$ mediator \sout{could  still be chosen as} \hto{can  either be} a real or complex scalar, which are distinguished by the fact that a complex scalar contains both scalar and pseudoscalar particles, whereas the real \sout{option contains} \hto{field can be} only a scalar field. We will consider two choices for DM simplified models: one where the interaction with the SM is mediated by the real scalar, and the second where we consider only a light pseudoscalar (assuming that the associated scalar is decoupled from the low-energy spectrum). 
 
Couplings to the SM fermions can be arranged by mixing with the SM Higgs.  Such models have intriguing connections with Higgs physics, and can be viewed as generalizations of the Higgs portal to DM. The most general scalar mediator models will \sout{of course} have renormalizable interactions between the SM Higgs and the new scalar $\phi$ or pseudoscalar $a$, as well as $\phi/a$ interactions with electroweak gauge bosons. Such interactions are model dependent, often subject to constraints from electroweak precision tests, and would suggest specialized searches which cannot be generalized to a broad class of models (unlike, for instance, the \hto{$\missET+\textrm{jets}$} searches). As a result, for this class of \hto{minimal} simplified models with spin-$0$ mediators, we \sout{suggest to} \hto{will} focus primarily on \sout{the} couplings to fermions and \sout{the} loop-induced couplings to gluons. \sout{The possibility that the couplings} Possible couplings to the electroweak sector \hto{may} also lead to interesting DM phenomenology \sout{should however be kept in mind} \hto{BP: Broken sentence} and can be studied in the context of Higgs \hto{P}ortal DM.

\subsection{Fermionic DM}

Minimal Flavor Violation (MFV) \sout{dictates} \hto{implies} \sout{that the coupling of a scalar to the SM fermions} \hto{scalar couplings to fermions are} \sout{will be} proportional to the fermion mass\sout{es}.  However, \sout{it allows these couplings to be scaled by separate factors} \hto{can differ} for \sout{the up-type quarks, down-type quarks, and the} \hto{for up- and down-type quarks and} charged leptons. Assuming that DM is a fermion $\chi$, which couples to the SM only through a scalar $\phi$ or pseudoscalar $a$, the most general tree-level Lagrangians compatible with the MFV assumption are~\cite{Cotta:2013jna,Abdullah:2014lla}:
 \begin{eqnarray}
{\cal L}_{{\rm fermion},\phi} & = & {\cal L}_{\rm SM}+i\bar{\chi} \slashed{\partial} \chi + m_\chi \bar{\chi}\chi + \left| \partial_\mu \phi \right|^2+\frac{1}{2}m_\phi^2 \phi^2 + \nonumber \\ 
& &  g_\chi \phi \bar{\chi}\chi+ \frac{\phi}{\sqrt{2}} \sum_i \left(g_u y_i^u \bar{u}_i u_i+g_d y_i^d \bar{d}_i d_i+g_\ell y_i^\ell \bar{\ell}_i \ell_i\right)\, , \label{eq:scalarlag} \\
{\cal L}_{{\rm fermion},a} & = & {\cal L}_{\rm SM}+i\bar{\chi} \slashed{\partial} \chi + m_\chi \bar{\chi}\chi + \left| \partial_\mu a \right|^2+\frac{1}{2}m_a^2 a^2 + \nonumber \\
& &  ig_\chi a \bar{\chi}\gamma_5\chi+ \frac{i a}{\sqrt{2}}\sum_i  \left(g_u y_i^u \bar{u}_i \gamma_5 u_i+g_d y_i^d \bar{d}_i \gamma_5 d_i+g_\ell y_i^\ell   \bar{\ell}_i \gamma_5 \ell_i\right) \,. \label{eq:pseudoscalarlag}
\end{eqnarray}
Here the sums run over the \sout{there} \hto{all} SM \sout{families} \hto{generations} \sout{and we are using} \hto{the} Yukawa couplings $y_i^f$ \hto{are} normalized to $y_i^f = \sqrt{2}m_i^f/v$ \sout{with} \hto{where} $v \simeq 246 \, {\rm GeV}$ \hto{represents} the Higgs vacuum expectation value (VEV). \sout{Since the DM particle $\chi$ receives most of its mass most likely not from electroweak symmetry breaking but from another (unknown) mechanism, we parametrize the DM-mediator coupling by $g_\chi$ rather than by a Yukawa coupling $y_\chi = \sqrt{2}m_\chi/v$.}

\hto{BP: Previous sentence sounds speculating and I'd remove that, at least backup with reference. Prefer to just say: We parametrise the DM-mediator coupling as $g_\chi$.}

The most general Lagrangians including new scalars or pseudoscalars will have a potential  containing interactions with the SM Higgs field $h$. As \hto{already} stated \sout{above,} we choose \sout{to take a more} a minimal set of \sout{possible} interactions and leave the discussions of the couplings in the Higgs sector to \sout{the} section \sout{on} \hto{about}  Higgs portal DM. Given this simplification, the minimal set of parameters under consideration is
 \bea
  \left\{ m_\chi,~ m_{\phi/a},~ g_\chi,~ g_u,~ g_d,~ g_\ell \right\} \,.
 \eea
The simplest choice of couplings  \hto{known as Minimal Simplified Dark Matter model (MSDM) (following here notation used Buchm\"uller et al., Buckley et al. etc. which makes a lot of sense)}  $g_u = g_d = g_\ell$, which is realized in singlet scalar extensions of the SM. 

If one extends the SM Higgs sector to a two Higgs doublet model, one \sout{can} obtain\hto{s} \sout{other coupling patterns} \hto{more complex couplings} such as  $g_u = \cot \beta$ and $g_d = g_e = \tan \beta$ \sout{with} \hto{where} $\tan \beta$ denoting the ratio of VEVs of the two Higgs doublets.  The case $g_u \neq  g_d \neq g_\ell$ requires \sout{more} additional scalars \sout{, whose masses could be rather heavy} \hto{with potentially large masses}. 

\sout{For simplicity, we will use}\hto{We assume} universal SM-mediator couplings $g_v = g_u = g_d = g_\ell$ in the remainder of this work \sout{, though one should bear in mind that finding ways to experimentally  test this assumption would be very useful.} \hto{BP: That sounds like we don't know what we are doing. If there is no indication otherwise the easiest assumptions is always the most scientific to chose (also listening to some phenomenologists this seems not always to be the case ;-) )  }
 
The \sout{signal strength} \hto{The expected signal of DM pair production depends on the production rate defined by the dark matter mass $m_\chi$, mediator $m_{\phi/a}$ and the couplings $g_i$ and also the branching ration defined by the} total decay width of the mediator $\phi/a$. \sout{In the minimal model} \hto{While we cannot specify a width only from the couplings, we can calculate the minimum
possible width (assuming only decays into the dark matter and the Standard Model fermions) that is consistent with a given value of $g_\chi g_\textrm{SM}$. These are given by Eq.~\myeq{eq:width}}\cite{Buckley:2014fba}. 

\hto{BP: Removed the initial two equations as they are identical except of an index and used only one with proper indices and also two two line instead of extending beyond the text width.}


\begin{equation} \label{eq:width}
\begin{split}
\Gamma_{\phi,a}  = & \sum_f N_c \frac{y_f^2 g_v^2 m_{\phi,a}}{16 \pi} \left(1-\frac{4 m_f^2}{m_{\phi,a}^2}\right)^{3/2}
                   + \frac{g_\chi^2 m_{\phi,a}}{8 \pi} \left(1-\frac{4 m_\chi^2}{m_{\phi,a}^2}\right)^{3/2}\\
                 & + \frac{\alpha_s^2 y_t^2 g_v^2 m_{\phi,a}^3}{32 \pi^3 v^2} \left| f_{\phi,a}\left(\tfrac{4m_t^2}{m_{\phi,a}^2} \right)\right|^2
\end{split}
\end{equation}

%\begin{eqnarray}
%\Gamma_\phi & = & \sum_f N_c \frac{y_f^2 g_v^2 m_\phi}{16 \pi} \left(1-\frac{4 m_f^2}{m_\phi^2}\right)^{3/2} + \frac{g_\chi^2 m_\phi}{8 \pi} \left(1-\frac{4 m_\chi^2}{m_\phi^2}\right)^{3/2} + \frac{\alpha_s^2 y_t^2 g_v^2 m_\phi^3}{32 \pi^3 v^2} \left| f_\phi\left(\tfrac{4m_t^2}{m_\phi^2} \right)\right|^2 \,,  \label{eq:scalarwidth}  \hspace{8mm} \\
%\Gamma_a & = & \sum_f N_c \frac{y_f^2 g_v^2 m_a}{16 \pi} \left(1-\frac{4 m_f^2}{m_a^2}\right)^{1/2} + \frac{g_\chi^2 m_a}{8 \pi} \left(1-\frac{4 m_\chi^2}{m_a^2}\right)^{1/2}  
%+ \frac{\alpha_s^2 y_t^2 g_v^2 m_a^3}{32 \pi^3 v^2} \left| f_a\left(\tfrac{4m_t^2}{m_\phi^2} \right)\right|^2 \,, \label{eq:pseudoscalarwidth} 
%\end{eqnarray}


whereas

\bea \label{eq:fphifa}
f_\phi (\tau) = \tau \left [ 1+ (1-\tau) \arctan^2 \left ( \frac{1}{\sqrt{\tau-1}} \right ) \right ]  \,, \qquad 
f_a (\tau) =  \tau \arctan^2 \left ( \frac{1}{\sqrt{\tau-1}} \right ) \,. 
\eea

The first term in each width corresponds to the decay into SM fermions (the sum runs over all kinematically available fermions, $N_c = 3$ for quarks and $N_c = 1$ for leptons). The second term is the decay into DM (assuming that this decay is kinematically allowed). The factor of \sout{2} two between the decay into SM  fermions and into DM  is a result of our choice of normalization of the Yukawa couplings \hto{BP: Is this the best way to express that? I'd add:... due to spin dependencies}. The last two terms correspond to decay into gluons.  Since we have assumed that $g_v = g_u = g_d = g_\ell$, we have included in the partial decay widths $\Gamma (\phi/a \to gg)$ only the contributions stemming from top loops, which provide the by far largest corrections given that $y_t \gg y_b$~etc. At the loop level the mediators can decay not only to gluons but also to pairs of photons and other final states if \sout{these are} kinematical accessible. \hto{However} the decay rates $\Gamma (\phi/a \to gg)$ are \sout{however} always larger than the other loop-induced partial widths, and in consequence the total decay widths $\Gamma_{\phi/a}$ are well approximated by the corresponding sum of the individual partial decay widths involving DM, fermion or gluon pairs. \sout{Notice finally} \hto{It should be noted} that if  $m_{\phi/a} > 2m_t$ the total widths of $\phi/a$ will typically be dominated by the partial widths to top quarks.
