
%\section{DM production with $t\bar{t}$}
\section{DM+$t\bar{t}$ production}
\label{sec:ttdm}

\sout{As discussed in Section~\ref{sec:scalar}, the MFV assumption entails a spin-$0$ mediator that couples most strongly to top quarks on account of Yukawa couplings}.
\hto{As discussed in Sec.~\ref{sec:scalar}, the MFV assumption leads for spin-$0$ mediators to quark mass dependent Yukawa couplings and therefore dominant couplings to top quarks.}

\sout{ For establishing benchmarks, DM tt models with a scalar phi or a pseudoscalar a are the most compelling.}
\hto{[BP: Physics first:] Therefore DM+}$t\bar{t}$\hto{searches are strongly physically motivated and it is important to establish benchmarks for collider searches following the assumptions described in Sec.~\ref{sec:scalar}, in particular a Dirac fermion DM particle, universal couplings and minimum width.}
\sout{The simplifications mentioned in Section~\ref{sec:scalar} are followed; namely, coupling of the mediator to SM Higgs are not considered and universal SM-mediator couplings are assumed. Another simplification is to assume minimal width, so that }$\Gamma_{\phi,a}$\sout{is considered a dependent variable in the models. The DM particle is taken to be a Dirac fermion.}

\hto{Benchmark points have been selected to ensure that} \sout{The primary consideration for selecting the set of benchmark points is to ensure} the kinematic features of the parameter space are sufficiently represented. Detailed studies were performed to identify points in the $m_{\chi}$, $m_{\phi,a}$, $g_{\chi}$, $g_{v}$ (and $\Gamma_{\phi,a}$), parameter space that differ significantly from each other in terms of expected detector acceptance. Because missing \hto transverse momentum \sout{energy} is the key observable for searches, the mediator $p_{T}$ spectra is taken to represent the \hto{main} kinematics of a model. Another consideration in determining the set of benchmarks is to \sout{restrict to cases  where we have sensitivity} \hto{to focus on phase space where we expected to gain sensitivity during the LHC run in 2015}. Based on the \sout{estimated} \hto{projected} integrated luminosity of $30\,{\rm fb}^{-1}$ expected \hto{for} 2015, we disregard models \hto{with} cross section times branching ratio smaller than $0.1\,{\rm fb}$.

\subsection{Parameter scan}
The kinematics \sout{show the strongest dependence} \hto{is most dependent} on the masses $m_{\chi}$ and $m_{\phi,a}$. \hto{Figure~\ref{fig:scanPhi} and~\ref{fig:scanPhiPseudo} show typical depenencies
for for scalar, pseudoscalar couplings respectively.}
\sout{Examples of the dependence on mediator mass are given in Figs.\sout{ures}~\ref{fig:scanPhi} and~\ref{fig:scanPhiPseudo} for the scalar and pseudoscalar respectively}.\sout{ There are} \hto{The} two relevant thresholds \hto{are}: $m_{\phi,a} = 2m_{\chi}$ and $m_{\phi,a} = 2m_t$. When the mediator \hto{mass exceeds} both \sout{these} thresholds \hto{then} the $p_{T}$ spectra broadens with larger $m_{\phi,a}$ and the kinematics for $\phi$ and $a$ are \sout{quite similar} \sout{comparable}. The mediator $p_{T}$ spectra changes significantly when crossing these thresholds. \hto{For instance} \hto{In particular} kinematics are different for an on-shell mediator compared to an off-shell mediator ($m_{\phi,a}<2m_{\chi}$). Furthermore, the scalar \hto{case} differs from the pseudoscalar \hto{one} when $m_{\phi}<2m_t$. Therefore, it is important to \sout{have benchmark points } cover both sides of these thresholds \hto{with sufficient benchmark points}.

\begin{figure}[!ht]
  \begin{center}
    \includegraphics[scale=0.45]{figures/MEt_chi1.pdf}
    \vspace{2mm}
    \caption{\label{fig:scanPhi} \htg{Example of the dependence of the kinematics on the scalar mediator mass. The Dark Matter mass is fixed to be $1 {\rm GeV}$.}
    }
\end{center}
\end{figure}


\begin{figure}[!ht]
  \begin{center}
    \includegraphics[scale=0.45]{figures/MEt_chi1_pseudo.pdf}
    \vspace{2mm}
    \caption{\label{fig:scanPhiPseudo} \htg{Example of the dependence of the kinematics on the pseudoscalar mediator mass. The Dark Matter mass is fixed to be $1 {\rm GeV}$.}
    }
\end{center}
\end{figure}

\begin{figure}[!ht]
  \begin{center}
    \includegraphics[scale=0.45]{figures/MEt_diagonal_scan.pdf}
    \vspace{2mm}
    \caption{\label{fig:scanPhidiag} \htg{Example of the dependence of the kinematic for points of the grid proposed in Tab.~\ref{tab:ttdm_benchmarks} close to the $m_{\phi,a} \sim 2m_\chi$ limit.}
    }
\end{center}
\end{figure}


%{\color{red} Add plot: $m_{\chi}$ scan}
%{\color{red} Add plot: on-shell vs off-shell}
%{\color{red} Add plot: S vs PS for $m_{\phi} < 2m_t$}

Typically, \sout{the kinematics show little dependence on the width -- or equivalently on the couplings} \hto{typically only weak dependencies on width or equivalently couplings are observed}  (see Fig.\sout{ure}~\ref{fig:widthsmallscan}), except \sout{at} \hto{for} large mediator masses of $\sim 1.5\,{\rm TeV}$ or \hto{for} very small couplings of $\sim 10^{-2}$. These regimes where width effects are significant have \sout{signal strengths} \hto{production cross section} that are too \sout{weak} \hto{small} to be relevant for $30\,{\rm fb}^{-1}$ and are not considered here. However, with the full dataset of Run-2, such models may be within reach. The generally weak dependence on typical width values can be understood as the parton distribution function \sout{being} \hto{are} the dominant effect on mediator production. In other words, for couplings $\sim O(1)$ the width is \sout{broad} \hto{large} enough that the $p_T$ of the mediator is determined mainly by the PDF.

\begin{figure}[!ht]
  \begin{center}
    \includegraphics[scale=0.45]{figures/MEt_smallwidth.pdf}
    \vspace{2mm}
    \caption{\label{fig:widthsmallscan} \htg{Study of the dependence of kinematics on the width of a scalar mediator. The width is increased up to four times the minimal width for each mediator and dark matter mass combination. }
    }
\end{center}
\end{figure}

Another case where the width can impact the kinematics is when $m_{\phi,a}$ is slightly larger than $2m_\chi$. Here, the width determines the relative contribution between on-shell and off-shell production. An example is given in Fig.\sout{ure}~\ref{fig:widthlargescan}. \htg{In our reccomendations we propose to use for semplicity the minimal width, as this is represents the most conservative choice to interpret the LHC results}

\begin{figure}[!ht]
  \begin{center}
    \includegraphics[scale=0.45]{figures/ScalarWidth.pdf}
    \vspace{2mm}
    \caption{\label{fig:widthlargescan} \htg{Dependence of the dependence of kinematics on the width of a scalar mediator. The width is increased up to the mediator mass. Choices of mediator and dark matter masses such that $m_{\phi,a}$ is slightly larger than $2m_\chi$ is the only case that shows a sizeable variation of the kinematics as a function of the width.  }
    }
\end{center}
\end{figure}

%{\color{red} Add plot: scan of couplings at $m_{\phi}=1.5\,{\rm TeV}$ to show kinematic differences}

Given that the kinematics are similar for all couplings $\sim O(1)$, we \sout{make a simplification to generate benchmark models with} \hto{we generate only samples with} $g_{\chi} = g_{v} = 1$. It follows that the model should be a good approximation for non-unity couplings and $g_{\chi} \neq g_{v}$ provided that the sample is normalized to the appropriate cross section times branching ratio. While a simple scaling function can be found that works for a limited range of coupling values (see Fig.\sout{ure}~\ref{fig:xsec_scaling} for example), we \sout{opt} \hto{chose} to provide instead a table of cross section times branching ratio values over a large range of couplings to support interpretation of search results (see Sec.tion~\ref{subsec:xsectable}). The table \hto{covers} lists couplings from $g=0.1$ to $g=3.5$, where the upper limit is chosen to \sout{lie near but below the} \hto{close to the} perturbative limit.

\begin{figure}[!ht]
\begin{center}
\includegraphics[scale=0.45]{figures/xVSwom_mphi_400_mchi_1_proc_S.png}
\vspace{2mm}
\caption{\label{fig:xsec_scaling} An example comparing a simple cross section scaling versus the computation from the generator, for a scalar model with $m_{\phi}=400\,{\rm GeV}$ and $m_{\chi}=1\,{\rm GeV}$. In this example, the scaling relationship holds for $\Gamma_{\phi}/m_{\phi}$ below $0.2$, beyond which finite width effects become important and the simple scaling breaks down.}
\end{center}
\end{figure}





\subsection{Benchmarks}
The benchmark points are listed in Table~\ref{tab:ttdm_benchmarks}. In addition to the considerations discussed in the preceding subsections, very light DM fermions are included ($m_{\chi}=1, 10\,{\rm GeV}$) as these are beyond the reach of current direct detection experiments, and a few points are chosen to overlap with monojet benchmarks to facilitate comparison with searches in different final states.


\begin{table}[!ht]
\centering
\begin{tabular}{| l | r |}
\hline
\multicolumn{1}{|c|}{$m_{\chi}$ (${\rm GeV}$)} & \multicolumn{1}{c|}{$m_{\phi,a}$ (${\rm GeV}$)} \\
\hline
 $1$    & $10$, $20$, $50$, $100$, $150$, $200$, $300$, $500$, $1000$, $1500$  \\
 $10$   & $10$, $20$, $50$, $100$, $150$, $200$, $300$, $500$, $1000$, $1500$  \\
 $50$   &             $50$, $100$, $150$, $200$, $300$, $500$, $1000$, $1500$  \\
 $150$  &                          $150$, $200$, $300$, $500$, $1000$, $1500$  \\
 $500$  &                                               $500$, $1000$, $1500$  \\
\hline
\end{tabular}
\caption{Simplified model benchmarks for $t\bar{t}$+DM production via spin-0 mediators decaying to Dirac DM fermions taking the minimum width presciption for $g_v = g_{\chi} = 1$.}
\label{tab:ttdm_benchmarks}
\end{table}

\newpage
\subsection{Table of Cross Sections}
\label{subsec:xsectable}
\begin{longtable}{lllll}
\toprule
Coupling (g) & $m_{Phi}$ [GeV] & $m_{\chi}$ [GeV] & $\Gamma_{min}$ [GeV] & $\sigma$\\
\midrule
\endfirsthead

\multicolumn{5}{c}{\tablename\ \thetable{} -- continued from previous page} \\
Coupling (g) & $m_{Phi}$ [GeV] & $m_{\chi}$ [GeV] & $\Gamma_{min}$ [GeV] & $\sigma$\\
\midrule
\endhead

\midrule
\multicolumn{5}{r}{{Continued on next page}} \\ 
\endfoot

\bottomrule
\endlastfoot

 0.1 &    10 &     1 & 0.00374318 &    0.207 $\pm$ 0.0006846 \\
 0.1 &    20 &     1 & 0.00784569 &   0.1121 $\pm$ 0.0003285 \\
 0.1 &    50 &     1 &  0.01987 &  0.03211 $\pm$ 0.0001005 \\
 0.1 &   100 &     1 & 0.0398141 & 0.007325 $\pm$ 2.416e-05 \\
 0.1 &   150 &     1 & 0.0597437 & 0.002396 $\pm$ 7.419e-06 \\
 0.1 &   200 &     1 & 0.0796724 & 0.001018 $\pm$ 3.398e-06 \\
 0.1 &   300 &     1 & 0.119549 & 0.0003394 $\pm$ 1.234e-06 \\
 0.1 &   500 &     1 & 0.310863 & 6.802e-05 $\pm$ 2.343e-07 \\
 0.1 &  1000 &     1 & 0.881329 & 5.817e-06 $\pm$ 2.356e-08 \\
 0.1 &  1500 &     1 &  1.40417 & 8.942e-07 $\pm$ 3.832e-09 \\
 0.1 &    10 &    10 & 0.000100 & 1.007e-05 $\pm$ 3.761e-08 \\
 0.1 &    20 &    10 & 0.000100  & 3.491e-05 $\pm$ 1.012e-07 \\
 0.1 &    50 &    10 & 0.0153395 &  0.03212 $\pm$ 0.0001037 \\
 0.1 &   100 &    10 & 0.0374747 & 0.007343 $\pm$ 2.011e-05 \\
 0.1 &   150 &    10 & 0.0581752 & 0.002389 $\pm$ 7.654e-06 \\
 0.1 &   200 &    10 & 0.0784937 & 0.001018 $\pm$ 6.258e-06 \\
 0.1 &   300 &    10 & 0.118762 & 0.0003373 $\pm$ 1.448e-06 \\
 0.1 &   500 &    10 & 0.310391 & 6.773e-05 $\pm$ 2.326e-07 \\
 0.1 &  1000 &    10 & 0.881093 & 5.81e-06 $\pm$ 2.245e-08 \\
 0.1 &  1500 &    10 &  1.40401 & 8.937e-07 $\pm$ 4.013e-09 \\
 0.1 &    50 &    50 & 0.0000233555 & 2.581e-07 $\pm$ 1.214e-09 \\
 0.1 &   100 &    50 & 0.0000492402 & 1.526e-06 $\pm$ 7.038e-09 \\
 0.1 &   150 &    50 & 0.0247905 & 0.002387 $\pm$ 8.272e-06 \\
 0.1 &   200 &    50 & 0.051794 &  0.00102 $\pm$ 3.216e-06 \\
 0.1 &   300 &    50 & 0.100226 & 0.0003366 $\pm$ 1.393e-06 \\
 0.1 &   500 &    50 & 0.299052 & 6.679e-05 $\pm$ 2.406e-07 \\
 0.1 &  1000 &    50 & 0.875378 & 5.764e-06 $\pm$ 2.472e-08 \\
 0.1 &  1500 &    50 &   1.4002 & 8.866e-07 $\pm$ 3.257e-09 \\
 0.1 &   100 &   150 & 0.0000492402 & 1.246e-08 $\pm$ 5.121e-11 \\
 0.1 &   150 &   150 & 0.0000765167 & 1.393e-08 $\pm$ 6.653e-11 \\
 0.1 &   200 &   150 & 0.000106902 & 1.693e-08 $\pm$ 8.493e-11 \\
 0.1 &   300 &   150 & 0.000190543 & 7.557e-08 $\pm$ 2.171e-10 \\
 0.1 &   500 &   150 & 0.213784 & 5.063e-05 $\pm$ 1.724e-07 \\
 0.1 &  1000 &   150 & 0.828844 & 5.365e-06 $\pm$ 2.028e-08 \\
 0.1 &  1500 &   150 &  1.36872 & 8.603e-07 $\pm$ 3.769e-09 \\
 0.1 &   200 &   300 & 0.000106902 & 1.415e-09 $\pm$ 5.97e-12 \\
 0.1 &   300 &   300 & 0.000190543 & 1.64e-09 $\pm$ 7.878e-12 \\
 0.1 &   500 &   300 & 0.111924 & 3.078e-09 $\pm$ 1.482e-11 \\
 0.1 &  1000 &   300 & 0.687162 & 3.828e-06 $\pm$ 1.416e-08 \\
 0.1 &  1500 &   300 &  1.26683 & 7.579e-07 $\pm$ 3.041e-09 \\
 0.1 &   500 &   500 & 0.111924 & 1.784e-10 $\pm$ 1.105e-12 \\
 0.1 &  1000 &   500 & 0.483444 & 1.98e-09 $\pm$ 9.199e-12 \\
 0.1 &  1500 &   500 &  1.05448 & 4.92e-07 $\pm$ 2.14e-09 \\
 0.3 &    10 &     1 & 0.0336886 &    1.876 $\pm$ 0.006611 \\
 0.3 &    20 &     1 & 0.0706112 &    1.006 $\pm$ 0.003894 \\
 0.3 &    50 &     1 &  0.17883 &   0.2886 $\pm$ 0.0009285 \\
 0.3 &   100 &     1 & 0.358327 &  0.06598 $\pm$ 0.000182 \\
 0.3 &   150 &     1 & 0.537693 &   0.0214 $\pm$ 6.701e-05 \\
 0.3 &   200 &     1 & 0.717052 & 0.009216 $\pm$ 3.533e-05 \\
 0.3 &   300 &     1 &  1.07594 & 0.003044 $\pm$ 1.194e-05 \\
 0.3 &   500 &     1 &  2.79777 & 0.0006105 $\pm$ 2.187e-06 \\
 0.3 &  1000 &     1 &  7.93196 & 5.256e-05 $\pm$ 2.165e-07 \\
 0.3 &  1500 &     1 &  12.6376 & 8.048e-06 $\pm$ 3.473e-08 \\
 0.3 &    10 &    10 &  5.69808 & 0.0008143 $\pm$ 3.272e-06 \\
 0.3 &    20 &    10 & 0.0000630938 & 0.002836 $\pm$ 9.724e-06 \\
 0.3 &    50 &    10 & 0.138055 &   0.2869 $\pm$ 0.0008971 \\
 0.3 &   100 &    10 & 0.337272 &  0.06606 $\pm$ 0.0002407 \\
 0.3 &   150 &    10 & 0.523576 &  0.02145 $\pm$ 8.01e-05 \\
 0.3 &   200 &    10 & 0.706443 & 0.009222 $\pm$ 2.807e-05 \\
 0.3 &   300 &    10 &  1.06886 & 0.003051 $\pm$ 1.001e-05 \\
 0.3 &   500 &    10 &  2.79352 & 0.0006115 $\pm$ 2.268e-06 \\
 0.3 &  1000 &    10 &  7.92983 & 5.24e-05 $\pm$ 1.964e-07 \\
 0.3 &  1500 &    10 &  12.6361 & 8.053e-06 $\pm$ 3.203e-08 \\
 0.3 &    10 &    50 &  5.69808 & 1.704e-05 $\pm$ 7.077e-08 \\
 0.3 &    20 &    50 & 0.0000630938 & 1.746e-05 $\pm$ 7.383e-08 \\
 0.3 &    50 &    50 & 0.000210199 & 2.071e-05 $\pm$ 8.162e-08 \\
 0.3 &   100 &    50 & 0.000443162 & 0.0001245 $\pm$ 3.888e-07 \\
 0.3 &   150 &    50 & 0.223114 &  0.02138 $\pm$ 6.22e-05 \\
 0.3 &   200 &    50 & 0.466146 & 0.009186 $\pm$ 3.168e-05 \\
 0.3 &   300 &    50 & 0.902031 & 0.003039 $\pm$ 1.09e-05 \\
 0.3 &   500 &    50 &  2.69146 & 0.0005971 $\pm$ 2.181e-06 \\
 0.3 &  1000 &    50 &   7.8784 & 5.222e-05 $\pm$ 1.907e-07 \\
 0.3 &  1500 &    50 &  12.6018 & 7.947e-06 $\pm$ 2.996e-08 \\
 0.3 &   100 &   150 & 0.000443162 & 1.004e-06 $\pm$ 4.682e-09 \\
 0.3 &   150 &   150 & 0.00068865 & 1.132e-06 $\pm$ 4.644e-09 \\
 0.3 &   200 &   150 & 0.000962116 & 1.349e-06 $\pm$ 6.834e-09 \\
 0.3 &   300 &   150 & 0.00171489 & 6.08e-06 $\pm$ 2.289e-08 \\
 0.3 &   500 &   150 &  1.92405 & 0.000456 $\pm$ 2.064e-06 \\
 0.3 &  1000 &   150 &  7.45959 & 4.818e-05 $\pm$ 1.84e-07 \\
 0.3 &  1500 &   150 &  12.3185 & 7.796e-06 $\pm$ 2.802e-08 \\
 0.3 &   200 &   300 & 0.000962116 & 1.144e-07 $\pm$ 4.635e-10 \\
 0.3 &   300 &   300 & 0.00171489 & 1.324e-07 $\pm$ 6.534e-10 \\
 0.3 &   500 &   300 &  1.00732 &  2.5e-07 $\pm$ 1.113e-09 \\
 0.3 &  1000 &   300 &  6.18446 & 3.439e-05 $\pm$ 1.376e-07 \\
 0.3 &  1500 &   300 &  11.4014 & 6.834e-06 $\pm$ 2.623e-08 \\
 0.3 &   500 &   500 &  1.00732 & 1.449e-08 $\pm$ 5.536e-11 \\
 0.3 &  1000 &   500 &  4.35099 & 1.487e-07 $\pm$ 6.617e-10 \\
 0.3 &  1500 &   500 &  9.49035 & 4.374e-06 $\pm$ 1.739e-08 \\
 0.7 &    10 &     1 & 0.183416 &     10.2 $\pm$  0.03649 \\
 0.7 &    20 &     1 & 0.384439 &    5.462 $\pm$  0.02022 \\
 0.7 &    50 &     1 &  0.97363 &    1.558 $\pm$ 0.004491 \\
 0.7 &   100 &     1 &  1.95089 &   0.3568 $\pm$ 0.001143 \\
 0.7 &   150 &     1 &  2.92744 &   0.1161 $\pm$ 0.0003685 \\
 0.7 &   200 &     1 &  3.90395 &  0.04995 $\pm$ 0.0001494 \\
 0.7 &   300 &     1 &  5.85789 &  0.01649 $\pm$ 5.579e-05 \\
 0.7 &   500 &     1 &  15.2323 & 0.003313 $\pm$ 1.464e-05 \\
 0.7 &  1000 &     1 &  43.1851 & 0.0002823 $\pm$ 1.233e-06 \\
 0.7 &  1500 &     1 &  68.8045 & 4.481e-05 $\pm$ 1.885e-07 \\
 0.7 &    10 &    10 & 0.0000310229 &  0.02403 $\pm$ 0.0001038 \\
 0.7 &    20 &    10 & 0.000343511 &  0.08347 $\pm$ 0.0004742 \\
 0.7 &    50 &    10 & 0.751635 &    1.553 $\pm$ 0.004764 \\
 0.7 &   100 &    10 &  1.83626 &   0.3569 $\pm$ 0.0009501 \\
 0.7 &   150 &    10 &  2.85058 &   0.1165 $\pm$ 0.0004139 \\
 0.7 &   200 &    10 &  3.84619 &  0.04984 $\pm$ 0.0001855 \\
 0.7 &   300 &    10 &  5.81933 &  0.01649 $\pm$ 6.843e-05 \\
 0.7 &   500 &    10 &  15.2092 & 0.003301 $\pm$ 1.289e-05 \\
 0.7 &  1000 &    10 &  43.1735 & 0.0002815 $\pm$ 1.129e-06 \\
 0.7 &  1500 &    10 &  68.7967 & 4.491e-05 $\pm$ 2.108e-07 \\
 0.7 &    10 &    50 & 0.0000310229 & 0.000511 $\pm$ 1.977e-06 \\
 0.7 &    20 &    50 & 0.000343511 & 0.0005184 $\pm$ 2.146e-06 \\
 0.7 &    50 &    50 & 0.00114442 & 0.0006176 $\pm$ 3.053e-06 \\
 0.7 &   100 &    50 & 0.00241277 & 0.003681 $\pm$ 1.333e-05 \\
 0.7 &   150 &    50 &  1.21473 &   0.1156 $\pm$ 0.0003755 \\
 0.7 &   200 &    50 &  2.53791 &  0.04988 $\pm$ 0.0001824 \\
 0.7 &   300 &    50 &  4.91106 &  0.01651 $\pm$ 6.317e-05 \\
 0.7 &   500 &    50 &  14.6535 & 0.003218 $\pm$ 1.523e-05 \\
 0.7 &  1000 &    50 &  42.8935 & 0.0002794 $\pm$ 1.049e-06 \\
 0.7 &  1500 &    50 &  68.6098 & 4.46e-05 $\pm$ 1.989e-07 \\
 0.7 &   100 &   150 & 0.00241277 & 2.968e-05 $\pm$ 1.364e-07 \\
 0.7 &   150 &   150 & 0.00374932 & 3.327e-05 $\pm$ 1.594e-07 \\
 0.7 &   200 &   150 & 0.00523819 & 4.04e-05 $\pm$ 1.861e-07 \\
 0.7 &   300 &   150 & 0.00933663 & 0.0001787 $\pm$ 7.694e-07 \\
 0.7 &   500 &   150 &  10.4754 &  0.00243 $\pm$ 1.128e-05 \\
 0.7 &  1000 &   150 &  40.6133 & 0.0002573 $\pm$ 1.014e-06 \\
 0.7 &  1500 &   150 &  67.0675 & 4.239e-05 $\pm$ 1.707e-07 \\
 0.7 &   100 &   300 & 0.00241277 & 3.132e-06 $\pm$ 1.547e-08 \\
 0.7 &   150 &   300 & 0.00374932 & 3.227e-06 $\pm$ 1.433e-08 \\
 0.7 &   200 &   300 & 0.00523819 & 3.393e-06 $\pm$ 1.437e-08 \\
 0.7 &   300 &   300 & 0.00933663 & 3.918e-06 $\pm$ 1.628e-08 \\
 0.7 &   500 &   300 &   5.4843 & 7.383e-06 $\pm$ 2.87e-08 \\
 0.7 &  1000 &   300 &  33.6709 & 0.0001801 $\pm$ 7.992e-07 \\
 0.7 &  1500 &   300 &  62.0745 & 3.644e-05 $\pm$ 1.473e-07 \\
 0.7 &   500 &   500 &   5.4843 & 4.301e-07 $\pm$ 1.836e-09 \\
 0.7 &  1000 &   500 &  23.6887 & 3.684e-06 $\pm$ 2.358e-08 \\
 0.7 &  1500 &   500 &  51.6697 & 2.291e-05 $\pm$ 9.843e-08 \\
  1. &    10 &     1 & 0.374318 &    20.79 $\pm$  0.08102 \\
  1. &    20 &     1 & 0.784569 &    11.08 $\pm$   0.0396 \\
  1. &    50 &     1 &    1.987 &    3.146 $\pm$  0.01331 \\
  1. &   100 &     1 &  3.98141 &   0.7199 $\pm$ 0.002775 \\
  1. &   150 &     1 &  5.97437 &   0.2354 $\pm$ 0.0008189 \\
  1. &   200 &     1 &  7.96724 &   0.1009 $\pm$ 0.0003854 \\
  1. &   300 &     1 &  11.9549 &  0.03369 $\pm$ 0.0001155 \\
  1. &   500 &     1 &  31.0863 & 0.006652 $\pm$ 2.898e-05 \\
  1. &  1000 &     1 &  88.1329 & 0.0005705 $\pm$ 2.817e-06 \\
  1. &  1500 &     1 &  140.417 & 9.244e-05 $\pm$ 4.273e-07 \\
  1. &    10 &    10 & 0.000063312 &   0.1009 $\pm$  0.00035 \\
  1. &    20 &    10 & 0.000701043 &   0.3475 $\pm$ 0.002265 \\
  1. &    50 &    10 &  1.53395 &    3.139 $\pm$  0.01028 \\
  1. &   100 &    10 &  3.74747 &   0.7158 $\pm$ 0.002486 \\
  1. &   150 &    10 &  5.81752 &    0.236 $\pm$ 0.0007591 \\
  1. &   200 &    10 &  7.84937 &   0.1013 $\pm$ 0.0003668 \\
  1. &   300 &    10 &  11.8762 &  0.03374 $\pm$ 0.0001403 \\
  1. &   500 &    10 &  31.0391 & 0.006631 $\pm$ 2.585e-05 \\
  1. &  1000 &    10 &  88.1093 & 0.0005663 $\pm$ 2.515e-06 \\
  1. &  1500 &    10 &  140.401 & 9.408e-05 $\pm$ 4.698e-07 \\
  1. &    10 &    50 & 0.000063312 &  0.00212 $\pm$ 8.815e-06 \\
  1. &    20 &    50 & 0.000701043 & 0.002149 $\pm$ 9.604e-06 \\
  1. &    50 &    50 & 0.00233555 & 0.002568 $\pm$ 1.017e-05 \\
  1. &   100 &    50 & 0.00492402 &  0.01523 $\pm$ 5.043e-05 \\
  1. &   150 &    50 &  2.47905 &   0.2351 $\pm$ 0.0008404 \\
  1. &   200 &    50 &   5.1794 &  0.09993 $\pm$ 0.0003164 \\
  1. &   300 &    50 &  10.0226 &  0.03349 $\pm$ 0.0001351 \\
  1. &   500 &    50 &  29.9052 & 0.006402 $\pm$ 2.604e-05 \\
  1. &  1000 &    50 &  87.5378 & 0.0005634 $\pm$ 2.601e-06 \\
  1. &  1500 &    50 &   140.02 & 9.211e-05 $\pm$ 4.909e-07 \\
  1. &   100 &   150 & 0.00492402 & 0.0001247 $\pm$ 5.899e-07 \\
  1. &   150 &   150 & 0.00765167 & 0.0001387 $\pm$ 5.889e-07 \\
  1. &   200 &   150 & 0.0106902 & 0.000168 $\pm$ 7.656e-07 \\
  1. &   300 &   150 & 0.0190543 & 0.0007464 $\pm$ 2.977e-06 \\
  1. &   500 &   150 &  21.3784 & 0.004856 $\pm$ 1.95e-05 \\
  1. &  1000 &   150 &  82.8844 & 0.0005122 $\pm$ 1.98e-06 \\
  1. &  1500 &   150 &  136.872 & 8.662e-05 $\pm$ 3.821e-07 \\
  1. &   200 &   300 & 0.0106902 & 1.422e-05 $\pm$ 6.147e-08 \\
  1. &   300 &   300 & 0.0190543 & 1.626e-05 $\pm$ 6.865e-08 \\
  1. &   500 &   300 &  11.1924 & 3.081e-05 $\pm$ 1.244e-07 \\
  1. &  1000 &   300 &  68.7162 & 0.0003534 $\pm$ 1.392e-06 \\
  1. &  1500 &   300 &  126.683 & 7.258e-05 $\pm$ 3.651e-07 \\
  1. &   500 &   500 &  11.1924 & 1.777e-06 $\pm$ 9.67e-09 \\
  1. &  1000 &   500 &  48.3444 & 1.331e-05 $\pm$ 6.551e-08 \\
  1. &  1500 &   500 &  105.448 & 4.443e-05 $\pm$ 1.988e-07 \\
 1.5 &    10 &     1 & 0.842215 &    46.59 $\pm$   0.1797 \\
 1.5 &    20 &     1 &  1.76528 &    24.52 $\pm$  0.08387 \\
 1.5 &    50 &     1 &  4.47075 &    6.903 $\pm$  0.02244 \\
 1.5 &   100 &     1 &  8.95817 &    1.577 $\pm$ 0.005493 \\
 1.5 &   150 &     1 &  13.4423 &   0.5224 $\pm$ 0.002309 \\
 1.5 &   200 &     1 &  17.9263 &   0.2259 $\pm$ 0.0008625 \\
 1.5 &   300 &     1 &  26.8985 &  0.07529 $\pm$ 0.0003407 \\
 1.5 &   500 &     1 &  69.9442 &  0.01445 $\pm$ 6.469e-05 \\
 1.5 &  1000 &     1 &  198.299 & 0.001234 $\pm$ 5.694e-06 \\
 1.5 &  1500 &     1 &  315.939 & 0.0002179 $\pm$ 1.024e-06 \\
 1.5 &    10 &    10 & 0.000142452 &   0.5117 $\pm$ 0.002037 \\
 1.5 &    20 &    10 & 0.00157735 &    1.763 $\pm$  0.01031 \\
 1.5 &    50 &    10 &  3.45138 &    6.906 $\pm$  0.02283 \\
 1.5 &   100 &    10 &   8.4318 &    1.568 $\pm$ 0.006489 \\
 1.5 &   150 &    10 &  13.0894 &   0.5162 $\pm$ 0.001934 \\
 1.5 &   200 &    10 &  17.6611 &   0.2249 $\pm$ 0.0008153 \\
 1.5 &   300 &    10 &  26.7214 &  0.07541 $\pm$ 0.0002941 \\
 1.5 &   500 &    10 &  69.8379 &  0.01447 $\pm$ 6.923e-05 \\
 1.5 &  1000 &    10 &  198.246 & 0.001242 $\pm$ 6.739e-06 \\
 1.5 &  1500 &    10 &  315.903 & 0.0002157 $\pm$ 8.805e-07 \\
 1.5 &    10 &    50 & 0.000142452 &  0.01068 $\pm$ 4.527e-05 \\
 1.5 &    20 &    50 & 0.00157735 &  0.01093 $\pm$ 6.079e-05 \\
 1.5 &    50 &    50 & 0.00525498 &  0.01302 $\pm$ 6.649e-05 \\
 1.5 &   100 &    50 & 0.011079 &  0.07677 $\pm$ 0.0002445 \\
 1.5 &   150 &    50 &  5.57786 &   0.5195 $\pm$ 0.001577 \\
 1.5 &   200 &    50 &  11.6536 &   0.2195 $\pm$ 0.0006711 \\
 1.5 &   300 &    50 &  22.5508 &  0.07353 $\pm$ 0.0003291 \\
 1.5 &   500 &    50 &  67.2866 &   0.0139 $\pm$ 6.13e-05 \\
 1.5 &  1000 &    50 &   196.96 & 0.001209 $\pm$ 7.038e-06 \\
 1.5 &  1500 &    50 &  315.045 & 0.0002109 $\pm$ 8.631e-07 \\
 1.5 &   100 &   150 & 0.011079 & 0.0006295 $\pm$ 3.008e-06 \\
 1.5 &   150 &   150 & 0.0172162 & 0.000706 $\pm$ 3.661e-06 \\
 1.5 &   200 &   150 & 0.0240529 &  0.00086 $\pm$ 3.608e-06 \\
 1.5 &   300 &   150 & 0.0428723 & 0.003751 $\pm$ 1.304e-05 \\
 1.5 &   500 &   150 &  48.1013 &  0.01046 $\pm$ 4.013e-05 \\
 1.5 &  1000 &   150 &   186.49 & 0.001072 $\pm$ 4.469e-06 \\
 1.5 &  1500 &   150 &  307.963 & 0.0001931 $\pm$ 1.022e-06 \\
 1.5 &   200 &   300 & 0.0240529 & 7.176e-05 $\pm$ 3.641e-07 \\
 1.5 &   300 &   300 & 0.0428723 &  8.3e-05 $\pm$ 3.627e-07 \\
 1.5 &   500 &   300 &   25.183 & 0.000155 $\pm$ 6.658e-07 \\
 1.5 &  1000 &   300 &  154.611 & 0.0007234 $\pm$ 2.773e-06 \\
 1.5 &  1500 &   300 &  285.036 & 0.0001529 $\pm$ 7.694e-07 \\
 1.5 &   500 &   500 &   25.183 & 9.099e-06 $\pm$ 4.301e-08 \\
 1.5 &  1000 &   500 &  108.775 & 5.335e-05 $\pm$ 2.699e-07 \\
 1.5 &  1500 &   500 &  237.259 & 8.736e-05 $\pm$ 4.268e-07 \\
  2. &    10 &     1 &  1.49727 &    82.65 $\pm$   0.3408 \\
  2. &    20 &     1 &  3.13828 &     43.1 $\pm$   0.1487 \\
  2. &    50 &     1 &    7.948 &    11.84 $\pm$  0.04278 \\
  2. &   100 &     1 &  15.9256 &    2.712 $\pm$  0.01209 \\
  2. &   150 &     1 &  23.8975 &   0.9056 $\pm$ 0.004237 \\
  2. &   200 &     1 &   31.869 &   0.3952 $\pm$ 0.001653 \\
  2. &   300 &     1 &  47.8195 &    0.132 $\pm$ 0.0004713 \\
  2. &   500 &     1 &  124.345 &  0.02461 $\pm$ 0.0001101 \\
  2. &  1000 &     1 &  352.532 & 0.002071 $\pm$ 1.061e-05 \\
  2. &  1500 &     1 &  561.669 & 0.0003815 $\pm$  1.4e-06 \\
  2. &    10 &    10 & 0.000253248 &    1.627 $\pm$ 0.005672 \\
  2. &    20 &    10 & 0.00280417 &    5.528 $\pm$  0.03152 \\
  2. &    50 &    10 &  6.13579 &    11.98 $\pm$  0.04005 \\
  2. &   100 &    10 &  14.9899 &    2.696 $\pm$  0.01091 \\
  2. &   150 &    10 &  23.2701 &   0.8981 $\pm$ 0.004067 \\
  2. &   200 &    10 &  31.3975 &   0.3921 $\pm$ 0.001675 \\
  2. &   300 &    10 &  47.5047 &   0.1312 $\pm$ 0.0005524 \\
  2. &   500 &    10 &  124.156 &  0.02454 $\pm$ 0.0001302 \\
  2. &  1000 &    10 &  352.437 & 0.002051 $\pm$ 9.73e-06 \\
  2. &  1500 &    10 &  561.606 & 0.0003797 $\pm$ 1.522e-06 \\
  2. &    10 &    50 & 0.000253248 &  0.03397 $\pm$ 0.0001354 \\
  2. &    20 &    50 & 0.00280417 &  0.03452 $\pm$ 0.0001623 \\
  2. &    50 &    50 & 0.00934219 &  0.04088 $\pm$ 0.0001623 \\
  2. &   100 &    50 & 0.0196961 &     0.24 $\pm$ 0.0008579 \\
  2. &   150 &    50 &   9.9162 &   0.8991 $\pm$ 0.002903 \\
  2. &   200 &    50 &  20.7176 &    0.382 $\pm$ 0.001411 \\
  2. &   300 &    50 &  40.0903 &   0.1287 $\pm$ 0.0005596 \\
  2. &   500 &    50 &  119.621 &  0.02328 $\pm$ 0.0001255 \\
  2. &  1000 &    50 &  350.151 & 0.001995 $\pm$ 1.184e-05 \\
  2. &  1500 &    50 &   560.08 & 0.0003671 $\pm$ 1.741e-06 \\
  2. &    10 &   150 & 0.000253248 & 0.001822 $\pm$ 7.946e-06 \\
  2. &    20 &   150 & 0.00280417 & 0.001842 $\pm$ 8.453e-06 \\
  2. &    50 &   150 & 0.00934219 &  0.00187 $\pm$ 8.818e-06 \\
  2. &   100 &   150 & 0.0196961 & 0.001985 $\pm$ 8.101e-06 \\
  2. &   150 &   150 & 0.0306067 & 0.002231 $\pm$ 1.131e-05 \\
  2. &   200 &   150 & 0.0427607 & 0.002694 $\pm$ 1.215e-05 \\
  2. &   300 &   150 & 0.0762174 &  0.01186 $\pm$ 4.862e-05 \\
  2. &   500 &   150 &  85.5134 &  0.01769 $\pm$ 8.02e-05 \\
  2. &  1000 &   150 &  331.538 & 0.001716 $\pm$ 7.617e-06 \\
  2. &  1500 &   150 &   547.49 & 0.0003242 $\pm$ 1.537e-06 \\
  2. &   100 &   300 & 0.0196961 & 0.0002092 $\pm$ 8.197e-07 \\
  2. &   150 &   300 & 0.0306067 & 0.0002152 $\pm$ 8.37e-07 \\
  2. &   200 &   300 & 0.0427607 & 0.0002275 $\pm$ 8.607e-07 \\
  2. &   300 &   300 & 0.0762174 & 0.0002609 $\pm$ 1.05e-06 \\
  2. &   500 &   300 &  44.7698 & 0.0004931 $\pm$ 2.01e-06 \\
  2. &  1000 &   300 &  274.865 & 0.001119 $\pm$ 5.167e-06 \\
  2. &  1500 &   300 &  506.731 & 0.0002432 $\pm$ 1.053e-06 \\
  2. &   300 &   500 & 0.0762174 & 2.367e-05 $\pm$ 1.206e-07 \\
  2. &   500 &   500 &  44.7698 & 2.871e-05 $\pm$ 1.09e-07 \\
  2. &  1000 &   500 &  193.378 & 0.000131 $\pm$ 5.569e-07 \\
  2. &  1500 &   500 &  421.793 & 0.0001323 $\pm$ 5.222e-07 \\
 2.5 &    10 &     1 &  2.33949 &    128.4 $\pm$   0.4393 \\
 2.5 &    20 &     1 &  4.90356 &    65.92 $\pm$   0.2248 \\
 2.5 &    50 &     1 &  12.4187 &    17.77 $\pm$   0.0663 \\
 2.5 &   100 &     1 &  24.8838 &    4.051 $\pm$  0.01562 \\
 2.5 &   150 &     1 &  37.3398 &    1.364 $\pm$ 0.004927 \\
 2.5 &   200 &     1 &  49.7953 &   0.6008 $\pm$ 0.002928 \\
 2.5 &   300 &     1 &   74.718 &   0.2036 $\pm$ 0.0008994 \\
 2.5 &   500 &     1 &   194.29 &  0.03629 $\pm$ 0.0001865 \\
 2.5 &  1000 &     1 &  550.831 & 0.002918 $\pm$ 1.235e-05 \\
 2.5 &  1500 &     1 &  877.608 & 0.0005639 $\pm$ 2.327e-06 \\
 2.5 &    10 &    10 & 0.0003957 &    3.918 $\pm$   0.0159 \\
 2.5 &    20 &    10 & 0.00438152 &    13.54 $\pm$  0.05349 \\
 2.5 &    50 &    10 &  9.58718 &    18.03 $\pm$  0.06068 \\
 2.5 &   100 &    10 &  23.4217 &    4.025 $\pm$  0.01458 \\
 2.5 &   150 &    10 &  36.3595 &     1.36 $\pm$  0.00698 \\
 2.5 &   200 &    10 &  49.0586 &   0.5979 $\pm$ 0.002445 \\
 2.5 &   300 &    10 &  74.2262 &   0.2016 $\pm$ 0.0006995 \\
 2.5 &   500 &    10 &  193.994 &  0.03579 $\pm$ 0.0001738 \\
 2.5 &  1000 &    10 &  550.683 & 0.002902 $\pm$ 1.515e-05 \\
 2.5 &  1500 &    10 &  877.509 & 0.0005651 $\pm$ 2.275e-06 \\
 2.5 &    10 &    50 & 0.0003957 &  0.08298 $\pm$ 0.000365 \\
 2.5 &    20 &    50 & 0.00438152 &  0.08474 $\pm$ 0.0003631 \\
 2.5 &    50 &    50 & 0.0145972 &  0.09986 $\pm$ 0.000455 \\
 2.5 &   100 &    50 & 0.0307751 &   0.5855 $\pm$ 0.001667 \\
 2.5 &   150 &    50 &  15.4941 &    1.359 $\pm$ 0.005802 \\
 2.5 &   200 &    50 &  32.3712 &   0.5728 $\pm$ 0.002188 \\
 2.5 &   300 &    50 &  62.6411 &   0.1938 $\pm$ 0.0008665 \\
 2.5 &   500 &    50 &  186.907 &  0.03384 $\pm$ 0.0001589 \\
 2.5 &  1000 &    50 &  547.111 & 0.002773 $\pm$ 1.645e-05 \\
 2.5 &  1500 &    50 &  875.125 & 0.0005349 $\pm$ 3.534e-06 \\
 2.5 &    10 &   150 & 0.0003957 & 0.004461 $\pm$ 1.951e-05 \\
 2.5 &    20 &   150 & 0.00438152 & 0.004473 $\pm$ 2.159e-05 \\
 2.5 &    50 &   150 & 0.0145972 &  0.00451 $\pm$ 1.808e-05 \\
 2.5 &   100 &   150 & 0.0307751 &  0.00486 $\pm$ 1.984e-05 \\
 2.5 &   150 &   150 & 0.0478229 &  0.00548 $\pm$ 2.35e-05 \\
 2.5 &   200 &   150 & 0.0668136 & 0.006545 $\pm$ 2.81e-05 \\
 2.5 &   300 &   150 &  0.11909 &  0.02878 $\pm$ 0.0001168 \\
 2.5 &   500 &   150 &  133.615 &  0.02572 $\pm$  0.00011 \\
 2.5 &  1000 &   150 &  518.027 & 0.002339 $\pm$ 1.101e-05 \\
 2.5 &  1500 &   150 &  855.453 & 0.0004622 $\pm$ 2.297e-06 \\
 2.5 &   100 &   300 & 0.0307751 & 0.0005104 $\pm$ 2.62e-06 \\
 2.5 &   150 &   300 & 0.0478229 & 0.000526 $\pm$ 2.091e-06 \\
 2.5 &   200 &   300 & 0.0668136 & 0.0005503 $\pm$ 2.402e-06 \\
 2.5 &   300 &   300 &  0.11909 & 0.0006368 $\pm$ 2.911e-06 \\
 2.5 &   500 &   300 &  69.9528 & 0.001197 $\pm$ 4.697e-06 \\
 2.5 &  1000 &   300 &  429.476 & 0.001499 $\pm$ 6.445e-06 \\
 2.5 &  1500 &   300 &  791.767 & 0.0003277 $\pm$ 1.439e-06 \\
 2.5 &   300 &   500 &  0.11909 & 5.773e-05 $\pm$ 2.645e-07 \\
 2.5 &   500 &   500 &  69.9528 & 6.973e-05 $\pm$ 3.037e-07 \\
 2.5 &  1000 &   500 &  302.152 & 0.0002498 $\pm$ 1.042e-06 \\
 2.5 &  1500 &   500 &  659.052 & 0.000172 $\pm$ 8.531e-07 \\
  3. &    10 &     1 &  3.36886 &    185.9 $\pm$   0.8608 \\
  3. &    20 &     1 &  7.06112 &    92.49 $\pm$   0.3581 \\
  3. &    50 &     1 &   17.883 &    24.38 $\pm$  0.08507 \\
  3. &   100 &     1 &  35.8327 &    5.551 $\pm$  0.02275 \\
  3. &   150 &     1 &  53.7693 &    1.878 $\pm$ 0.008801 \\
  3. &   200 &     1 &  71.7052 &   0.8398 $\pm$ 0.004651 \\
  3. &   300 &     1 &  107.594 &   0.2856 $\pm$ 0.001301 \\
  3. &   500 &     1 &  279.777 &  0.04861 $\pm$ 0.0002143 \\
  3. &  1000 &     1 &  793.196 & 0.003716 $\pm$ 1.874e-05 \\
  3. &  1500 &     1 &  1263.76 & 0.0007294 $\pm$ 3.217e-06 \\
  3. &    10 &    10 & 0.000569808 &    8.181 $\pm$  0.03184 \\
  3. &    20 &    10 & 0.00630938 &    28.05 $\pm$  0.09412 \\
  3. &    50 &    10 &  13.8055 &    24.97 $\pm$  0.07128 \\
  3. &   100 &    10 &  33.7272 &    5.485 $\pm$  0.01916 \\
  3. &   150 &    10 &  52.3576 &    1.858 $\pm$ 0.007406 \\
  3. &   200 &    10 &  70.6443 &   0.8336 $\pm$ 0.003435 \\
  3. &   300 &    10 &  106.886 &   0.2832 $\pm$ 0.001293 \\
  3. &   500 &    10 &  279.352 &  0.04802 $\pm$ 0.0003129 \\
  3. &  1000 &    10 &  792.983 & 0.003669 $\pm$ 1.542e-05 \\
  3. &  1500 &    10 &  1263.61 & 0.0007221 $\pm$ 3.036e-06 \\
  3. &    10 &    50 & 0.000569808 &   0.1714 $\pm$ 0.0007653 \\
  3. &    20 &    50 & 0.00630938 &   0.1751 $\pm$ 0.000689 \\
  3. &    50 &    50 & 0.0210199 &   0.2073 $\pm$ 0.001019 \\
  3. &   100 &    50 & 0.0443162 &     1.21 $\pm$ 0.003153 \\
  3. &   150 &    50 &  22.3114 &    1.896 $\pm$ 0.007571 \\
  3. &   200 &    50 &  46.6146 &    0.787 $\pm$ 0.002939 \\
  3. &   300 &    50 &  90.2031 &   0.2685 $\pm$ 0.001344 \\
  3. &   500 &    50 &  269.146 &  0.04468 $\pm$ 0.0002221 \\
  3. &  1000 &    50 &   787.84 & 0.003505 $\pm$ 1.861e-05 \\
  3. &  1500 &    50 &  1260.18 & 0.0006823 $\pm$ 3.857e-06 \\
  3. &    10 &   150 & 0.000569808 & 0.009285 $\pm$ 4.234e-05 \\
  3. &    20 &   150 & 0.00630938 &  0.00924 $\pm$ 4.234e-05 \\
  3. &    50 &   150 & 0.0210199 & 0.009462 $\pm$ 3.85e-05 \\
  3. &   100 &   150 & 0.0443162 &  0.01017 $\pm$ 4.443e-05 \\
  3. &   150 &   150 & 0.068865 &  0.01124 $\pm$ 5.221e-05 \\
  3. &   200 &   150 & 0.0962116 &  0.01366 $\pm$ 6.834e-05 \\
  3. &   300 &   150 & 0.171489 &  0.05937 $\pm$ 0.0002495 \\
  3. &   500 &   150 &  192.405 &  0.03448 $\pm$ 0.0001467 \\
  3. &  1000 &   150 &  745.959 &  0.00288 $\pm$ 1.359e-05 \\
  3. &  1500 &   150 &  1231.85 & 0.0005735 $\pm$ 3.925e-06 \\
  3. &    50 &   300 & 0.0210199 & 0.001039 $\pm$ 3.982e-06 \\
  3. &   100 &   300 & 0.0443162 & 0.001056 $\pm$ 4.834e-06 \\
  3. &   150 &   300 & 0.068865 & 0.001096 $\pm$ 4.922e-06 \\
  3. &   200 &   300 & 0.0962116 & 0.001147 $\pm$ 5.869e-06 \\
  3. &   300 &   300 & 0.171489 & 0.001327 $\pm$ 6.728e-06 \\
  3. &   500 &   300 &  100.732 &  0.00245 $\pm$ 9.636e-06 \\
  3. &  1000 &   300 &  618.446 & 0.001853 $\pm$ 7.863e-06 \\
  3. &  1500 &   300 &  1140.14 & 0.0003934 $\pm$ 2.083e-06 \\
  3. &   150 &   500 & 0.068865 & 0.0001123 $\pm$ 4.327e-07 \\
  3. &   200 &   500 & 0.0962116 & 0.000114 $\pm$ 5.127e-07 \\
  3. &   300 &   500 & 0.171489 & 0.0001206 $\pm$ 5.124e-07 \\
  3. &   500 &   500 &  100.732 & 0.0001447 $\pm$ 6.102e-07 \\
  3. &  1000 &   500 &  435.099 & 0.0004016 $\pm$ 1.656e-06 \\
  3. &  1500 &   500 &  949.035 & 0.0002061 $\pm$ 8.548e-07 \\
 3.5 &    10 &     1 &  4.58539 &    257.5 $\pm$   0.9241 \\
 3.5 &    20 &     1 &  9.61097 &    123.8 $\pm$   0.4645 \\
 3.5 &    50 &     1 &  24.3407 &    31.59 $\pm$  0.09614 \\
 3.5 &   100 &     1 &  48.7723 &     7.04 $\pm$  0.02954 \\
 3.5 &   150 &     1 &   73.186 &    2.417 $\pm$  0.01038 \\
 3.5 &   200 &     1 &  97.5987 &    1.089 $\pm$ 0.004308 \\
 3.5 &   300 &     1 &  146.447 &   0.3709 $\pm$ 0.001616 \\
 3.5 &   500 &     1 &  380.808 &  0.06035 $\pm$ 0.0003762 \\
 3.5 &  1000 &     1 &  1079.63 & 0.004345 $\pm$ 2.711e-05 \\
 3.5 &  1500 &     1 &  1720.11 & 0.0008581 $\pm$ 3.653e-06 \\
 3.5 &    10 &    10 & 0.000775572 &    15.08 $\pm$   0.0569 \\
 3.5 &    20 &    10 & 0.00858777 &    51.42 $\pm$   0.1478 \\
 3.5 &    50 &    10 &  18.7909 &    32.56 $\pm$   0.1113 \\
 3.5 &   100 &    10 &  45.9065 &    6.963 $\pm$  0.03199 \\
 3.5 &   150 &    10 &  71.2646 &     2.38 $\pm$ 0.009493 \\
 3.5 &   200 &    10 &  96.1548 &    1.079 $\pm$ 0.004244 \\
 3.5 &   300 &    10 &  145.483 &    0.369 $\pm$ 0.001602 \\
 3.5 &   500 &    10 &  380.229 &  0.05978 $\pm$ 0.0003017 \\
 3.5 &  1000 &    10 &  1079.34 & 0.004302 $\pm$ 2.412e-05 \\
 3.5 &  1500 &    10 &  1719.92 & 0.0008525 $\pm$ 3.878e-06 \\
 3.5 &    10 &    50 & 0.000775572 &   0.3176 $\pm$ 0.001314 \\
 3.5 &    20 &    50 & 0.00858777 &   0.3229 $\pm$ 0.001215 \\
 3.5 &    50 &    50 & 0.0286105 &   0.3857 $\pm$ 0.001618 \\
 3.5 &   100 &    50 & 0.0603192 &    2.228 $\pm$  0.00751 \\
 3.5 &   150 &    50 &  30.3684 &    2.477 $\pm$ 0.008787 \\
 3.5 &   200 &    50 &  63.4476 &    1.025 $\pm$ 0.003864 \\
 3.5 &   300 &    50 &  122.776 &   0.3483 $\pm$ 0.001614 \\
 3.5 &   500 &    50 &  366.338 &  0.05534 $\pm$ 0.0003035 \\
 3.5 &  1000 &    50 &  1072.34 & 0.004076 $\pm$ 2.371e-05 \\
 3.5 &  1500 &    50 &  1715.24 & 0.0008077 $\pm$ 4.889e-06 \\
 3.5 &    10 &   150 & 0.000775572 &  0.01719 $\pm$ 9.115e-05 \\
 3.5 &    20 &   150 & 0.00858777 &  0.01719 $\pm$ 8.334e-05 \\
 3.5 &    50 &   150 & 0.0286105 &  0.01754 $\pm$ 8.239e-05 \\
 3.5 &   100 &   150 & 0.0603192 &  0.01855 $\pm$ 8.371e-05 \\
 3.5 &   150 &   150 & 0.0937329 &  0.02099 $\pm$ 0.0001038 \\
 3.5 &   200 &   150 & 0.130955 &   0.0252 $\pm$ 0.0001138 \\
 3.5 &   300 &   150 & 0.233416 &   0.1096 $\pm$ 0.0006465 \\
 3.5 &   500 &   150 &  261.885 &  0.04374 $\pm$ 0.0002091 \\
 3.5 &  1000 &   150 &  1015.33 &  0.00334 $\pm$ 1.751e-05 \\
 3.5 &  1500 &   150 &  1676.69 & 0.0006583 $\pm$ 3.614e-06 \\
 3.5 &    10 &   300 & 0.000775572 & 0.001925 $\pm$ 9.279e-06 \\
 3.5 &    20 &   300 & 0.00858777 & 0.001916 $\pm$ 1.026e-05 \\
 3.5 &    50 &   300 & 0.0286105 & 0.001918 $\pm$ 8.166e-06 \\
 3.5 &   100 &   300 & 0.0603192 & 0.001958 $\pm$ 7.426e-06 \\
 3.5 &   150 &   300 & 0.0937329 & 0.002036 $\pm$ 8.81e-06 \\
 3.5 &   200 &   300 & 0.130955 & 0.002123 $\pm$ 8.379e-06 \\
 3.5 &   300 &   300 & 0.233416 & 0.002448 $\pm$ 9.259e-06 \\
 3.5 &   500 &   300 &  137.107 & 0.004413 $\pm$ 2.588e-05 \\
 3.5 &  1000 &   300 &  841.774 & 0.002184 $\pm$ 1.014e-05 \\
 3.5 &  1500 &   300 &  1551.86 & 0.0004471 $\pm$ 2.349e-06 \\
 3.5 &    10 &   500 & 0.000775572 & 0.0002016 $\pm$ 7.906e-07 \\
 3.5 &    20 &   500 & 0.00858777 & 0.0002011 $\pm$ 9.138e-07 \\
 3.5 &    50 &   500 & 0.0286105 & 0.0002018 $\pm$ 9.929e-07 \\
 3.5 &   100 &   500 & 0.0603192 & 0.0002033 $\pm$ 8.104e-07 \\
 3.5 &   150 &   500 & 0.0937329 & 0.0002067 $\pm$ 8.026e-07 \\
 3.5 &   200 &   500 & 0.130955 & 0.0002106 $\pm$ 8.439e-07 \\
 3.5 &   300 &   500 & 0.233416 & 0.0002225 $\pm$ 9.256e-07 \\
 3.5 &   500 &   500 &  137.107 & 0.0002686 $\pm$ 1.162e-06 \\
 3.5 &  1000 &   500 &  592.219 & 0.0005877 $\pm$ 2.823e-06 \\
 3.5 &  1500 &   500 &  1291.74 & 0.0002318 $\pm$ 1.11e-06 \\

\end{longtable}

