
ATLAS has considered two models, a resonant and a non-resonant production, using only right-handed top quarks in the lepton+jets final state. 
The signal samples were produced with {\sc Madgraph5} v1.5.11 interfaced with {\sc Pythia} 8.175, using the MSTW2008LO Parton Distribution Function (PDF) set (lhapdf ID: 21000).
The mass of the top quark was set at 172.5 GeV. Dynamic renormalisation and factorisation scales were used.
The $\met$ particle mass was varied, and in the case of the resonant model the resonance mass was fixed at 500~GeV:
\begin{itemize}
\item Resonant model,  $\met$ particle mass: [0,100]~GeV in 20~GeV steps
\item Non-resonant model, $\met$ particle mass: [0,150]~GeV in 25~GeV steps, [200,300]~GeV in 50~GeV and [400,1000]~GeV in 100~GeV steps 
\end{itemize}

%\com{How to translate the monotop paper couplings to the notation of this note? }
The couplings $\ares$ and $\anonres$ are set at a fixed value of $0.2$.
In addition, two samples are produced for the resonant model for $\mfmet=100$~GeV,
with coupling strengths fixed at $\ares=0.5$ and $\ares=1.0$,
in order to check the effect of the resonance width on the signal event kinematics. 
The total width of the resonance varies quadratically with the coupling strength,
corresponding to a width of 3.5~GeV, 21.6~GeV, and 86.5~GeV at $\ares=0.2$, $\ares=0.5$, and $\ares=1.0$, respectively.

The number of free parameters is reduced by assuming $(\ares^q)_{\mathrm{12}}=(\ares^q)_{\mathrm{21}}=(\ares^{1/2})_{\mathrm{3}}\equiv \ares$
for the resonant model and $(\anonres)_{\mathrm{13}}=(\anonres)_{\mathrm{31}}\equiv \anonres$ for the non-resonant model,
all other elements of these coupling matrices being equal to 0.
For each model, the coupling parameter $\ares$ or $\anonres$ and the masses of the exotic particles are independent.

The cross-sections as well as the width of the resonance for the resonance model are shown in Table~\ref{tab:S1R_Xsec}.  
The cross-section is slowly decreasing when $m(f_{met})$ increases,
and the values do not differ by larger than 10\%, due to the similarity of the kinematics, in the chosen mass range.
\begin{table}[!htb]\centering
\begin{tabular}{l|r|r|r}
\hline \hline
% $\mathrm{m}(f_{\mathrm{met}})$ [GeV] & $\sigma\times\mathrm{BR(S\rightarrow t f_{met})}\times\mathrm{BR(t\rightarrow l^+\nu b)[pb]}$ & $\sigma\times\mathrm{BR(S\rightarrow t f_{met})}\times\mathrm{BR(t\rightarrow jjj)[pb]}$ & $\Gamma(S)$ [GeV]   \\
$m(f_{met})$ [GeV] & $\sigma_{lep}$~[pb] & $\sigma_{had}$~[pb] & $\Gamma(\Phi)$ [GeV]   \\
\hline \hline
0                        &  1.107              & 2.214               &  3.492  \\
20                       &  1.102              & 2.205               &  3.491  \\
40                       &  1.089              & 2.180               &  3.487  \\
60                       &  1.068              & 2.137               &  3.481 \\	
80                       &  1.039              & 2.078               &  3.472  \\
100                      &  1.001              & 2.003               &  3.461  \\
100 ($\ares=0.5$) &  6.091              &  12.13              & 21.63    \\
100 ($\ares=1.0$  &  21.77              &  43.72              &  86.52   \\
\hline \hline
\end{tabular}
\caption
{
Theoretical predictions for the product of the production cross-section of the
scalar resonance, the branching ratio of its decay into a top quark and the invisible particle,
and of the branching ratio of the top quark decay into a semi-leptonic ($\sigma_{lep}$) or fully-hadronic ($\sigma_{had}$) final state,
in the resonance model.
Values are given for a resonance of mass 500~GeV and for an effective coupling $\ares=0.2$ (except for two masses),
as a function of the mass $m(f_{met})$ of the neutral fermion.
The total widths $\Gamma(\Phi)$ of the resonance are also shown.
}
\label{tab:S1R_Xsec}
\end{table}


For the non-resonant case, the cross-sections are given in Table~\ref{tab:S4R_Xsec} and are calculated with $\anonres=0.2$.
The cross-section diverges when $m(v_{met})$ tends to 0~GeV.  However, when the mass is exactly 0~GeV the cross-section has a finite value,
due to the specificity of the propagator for this massless spin-1 boson.
\begin{table}[!htb]\centering
\begin{tabular}{l|r|r}
\hline \hline
% $m(v_{met})$ [GeV] & $\sigma\times\mathrm{BR(t\rightarrow l^+\nu b)[pb]}$ & $\sigma\times\mathrm{BR(t\rightarrow jjj)[pb]}$ \\
$m(v_{met})$ [GeV] & $\sigma_{lep}$~[pb] & $\sigma_{had}$~[pb] \\
\hline \hline
0                  &  96.03              & 192.4    \\
25                 &  359.0              & 717.9    \\
50                 &  113.4              & 226.9    \\
75                 &  59.86              & 119.5    \\
100                &  37.45              & 74.82    \\
125                &  25.35              & 50.68    \\
150                &  18.00              & 35.96    \\
200                &  9.662              & 19.28    \\
250                &  5.506              & 11.02    \\
300                &  3.328              & 6.656    \\
400                &  1.372              & 2.738    \\
500                &  0.6345             & 1.270    \\
600                &  0.3192             & 0.6354   \\
700                &  0.1698             & 0.3383   \\
800                &  0.09417            & 0.1883   \\
900                &  0.05472            & 0.1091   \\
1000               &  0.03259            & 0.06479  \\
\hline \hline
\end{tabular}
\caption
{
Theoretical predictions for the product of the production cross-section of the invisible vector $v_{met}$ and of a top quark,
and of the branching ratio of the top quark decay into a semi-leptonic ($\sigma_{lep}$) or fully-hadronic ($\sigma_{had}$) final state, 
in the non-resonance model.
Values are given for an effective coupling $\anonres=0.2$, as a function of the mass $m(v_{met})$ of the invisible state.
}
\label{tab:S4R_Xsec}
\end{table}

\com{I think it might make more sense to have the joboption information in a public web site instead of
adding all the details into the note. Reference only visible for ATLAS members: \\
{\small https://svnweb.cern.ch/trac/atlasoff/browser/Generators/MC12JobOptions/trunk/gencontrol/MadGraphControl$\_$Monotop.py}}


\com{Question for DM forum:
\begin{itemize}
 \item Do we want to give more details about the Madgraph implementation, the couplings value in the param\_card, etc ... ?
 \item I am not aware of any work on systematic variation due to scale, PDF choice, showering (Maybe some was done in the monotop analysis?). Then I am not completely what to put here.
\end{itemize}
}

