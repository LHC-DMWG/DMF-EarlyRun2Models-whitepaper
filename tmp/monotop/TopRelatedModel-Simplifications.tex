\subsection{Simplifications}
 
The lagrangians from equations~\eqref{eq:lagrangianResonant} and~\eqref{eq:lagrangianNonResonantVector} contains too many degrees of freedom, 
which makes the LHC phenomenology difficult to predict. In addition, only a certain region of the parameter space can actually be probed with a monotop final state.
For these reasons, further simplifications are performed in particular in term of flavour and chiral structure of the model. 
These simplifications leads to some limitations in the way ATLAS and CMS can constrain the model parameter space and these limitations are also qualitatively discussed below.

\subsubsection{Flavour structure}

The flavour structure is simplified in order to have a reasonable signal production rate in proton-proton collisions.
In case of a scalar mediator, it has to be sufficiently produced so it has to couple with proton content, namely lightest quark which 
are allowed in equation~\eqref{eq:lagrangianResonant}. The monotop final state is sensitive to the scenario where $\phi$ strongly couples to $t\chi$.
\com{Correct and/or complete with the monotop paper}. In term of parameter space, it means that the monotop final state is not sensitive
to some parameters like coupling between the mediator and heavy quarks or the scenario in which $\BR{\phi}{t\chi} \ll 100 \%$. For the latest, 
there is a way to recover the sensitivity looking at $u_i d_j \to \phi \to u_i d_j$. Since $\phi$ must be produced, it has to coupled to quarks 
and must decay in the same final state. Experimentally, this would correspond to a di-jet resonance search.

The same kind of simplification is performed for the non-resonant production. The equation~\eqref{eq:lagrangianNonResonantVector} is simplified in the parameter
space where a monotop final state can be sufficiently produced to be detected at the LHC. The mediator $V$ must be produced from light quark initial state, 
in association with a top quark: this signature can mainly probe a high coupling $\left(g^{v/a}_{Vu}\right)^{13}_{Vu} \equiv g^{v/a}_{Vtu}$. Therefore,
the sensitivity to other flavour couplings is significantly lower since $V$ is less importantly produced. In addition, the mediator must decay into invisible particles
to lead to the searched monotop final state. As a consequence, the sensitivity for scenario where $\BR{V}{\chi\chi}\ll 100\%$ can be quite low. 
To cope with this second limitation, a same-sign top quark final state $gu \to tV(\to t\bar{u})$ is proposed to cover the cases where $V$ would decay
into visible particles. This case is more likely as the $tV$ production rate increases, and becomes then a key point to constraint this model in a consistent way.

\com{Questions for theorists:
\begin{itemize}
 \item How well these flavour assumptions are allowed by the other HEP data (proton decay life time, flavour physics, etc ...) ?
 \item MFV criteria ?
\end{itemize}
}

\subsubsection{Chiral structure}
\label{sec:chiralstructure}

The main point here is to consider only right handed quark components in order to not simplify the phenomenology. In fact, the representation of the left-handed 
components under the $\SUtwo$ symmetry imposes a coupling to $down$-type quarks, since the effective theory is invariant under $\SUtwoUone$ gauge symmetry. Having a coupling between
the mediator and $down$-type quarks fairly complicates the collider phenomenology in term of decay mode. Typically, including 
the left-handed components of quarks in the lagrangian~\eqref{eq:lagrangianNonResonantVector} describing the $Vtu$ vertex would lead to 
\begin{equation}
 \Lagr_{Vtu} \; = \;  g^{R}_{Vtu} \: \bar{t}_{R}\gam^{\mu}u_R \: V_{\mu} \; + \; g^{L}_{Vtu} \left(   \bar{t}_{L}\gam^{\mu}u_L \: + \:  \bar{b}_{L}\gam^{\mu}d_L \right) \: V_{\mu}
\end{equation}
where $g^{R/L} \equiv 1/2 \, (g^{v} \pm g^{a})$ couples only to right-handed/left-handed components. The second term ensure the invariance under $\SUtwo$ rotations, and lead to an additional decay mode $V \to b\bar{d} + \bar{b}d$ (on top of $V \to t\bar{u} + \bar{t}u$ and $V \to \chi\chi$). 
 
\subsection{Notations}

In section~\ref{sec:ColliderSignature}, the collider phenomenology and benchmark definition is discussed with notations which are 
a bit different\footnote{This difference is due to two things: the historical developpement on the monotop analysis and having a 
  common and simple set of notations for equations~\eqref{eq:lagrangianResonant},~\eqref{eq:lagrangianNonResonantScalar} and~\eqref{eq:lagrangianNonResonantVector}.} 
from section~\ref{sec:MotivModelDescription}. This section describes the notations used in section~\ref{sec:ColliderSignature} as well as 
the MadGraph model~\cite{MGmodel} convention, in term of the ones introduced in section~\ref{sec:MotivModelDescription}.

The Madgraph model corresponds to the Lagrangian from~\cite{AndreaFuksMaltoni}. Each coupling constant of this dynamics can be set via the paramater card and 
the blocks which are relevant for the two models used in section~\ref{sec:ColliderSignature} are described below.
\begin{enumerate}

\item Resonant scalar model described by the Lagrangian~\eqref{eq:lagrangianResonant}
  \begin{itemize}
  \item \texttt{AQS} and \texttt{BQS}: $3\times 3$ matrices (flavour space) fixing the coupling of the scalar $\phi^{\pm}$ ($S$ stands for scalar) and $down$-type 
    quarks ($Q$ stands for quarks), written in this note $g_{\phi u}$ or $a^{q}_{\mathrm{res}}$.
  \item \texttt{A12S} and \texttt{B12S}: $3\times 1$ matrices (flavour space) fixing the coupling of the fermion $\chi$ ($12$ stands for spin-$1/2$ fermion) 
    and $up$-type quarks, written in this note $g_{u \chi}$ or $a^{1/2}_{\mathrm{res}}$.
  \item particle name: the scalar $\phi^{\pm}$ is labelled $S$ and the fermion $\chi$ is $f_{met}$
  \end{itemize}  
  
\item Non-resonant vectorial model described by the Lagrangian~\eqref{eq:lagrangianNonResonantVector}
\begin{itemize}
\item \texttt{A1FC} and \texttt{B1FC}: $3\times 3$ matrices (flavour space) fixing the coupling of the vector $V$ 
  ($1$ stands for vector) and $up$-type quarks, written in this note $g_{Vu}$ or $a_{\mathrm{non-res}}$.
\item particle name: the vector $V$ is labelled $v_{met}$ and the fermion $\chi$ doesn't exist
\item the dark matter candidate $\chi$ is not implemented (this model assumes $\BR{V}{\chi\chi}=100\%$)
\end{itemize}

\end{enumerate}

$A$ means vectorial coupling ($g^{v}$) and $B$ means axial coupling ($g^{a}$) and these two matrices 
are taken to be equal according to the chiral simplification (see section~\ref{sec:chiralstructure}).
The convention used in~\cite{ATLASmonotop} and in section~\ref{sec:monotop} is to define a single number $a_{\mathrm{res}}$ ($a_{\mathrm{non-res}}$) 
for the resonant (non resonant) model, such as $(\ares^q)_{\mathrm{12}}=(\ares^q)_{\mathrm{21}}=(\ares^{1/2})_{\mathrm{3}}\equiv \ares$ 
(in order to have $d-s-S$ couplings, and $t-S-f_{met}$ couplings) 
and $(\anonres)_{\mathrm{13}}=(\anonres)_{\mathrm{31}}\equiv \anonres$ (in order to have $v_{met}-t-u$ couplings). 
