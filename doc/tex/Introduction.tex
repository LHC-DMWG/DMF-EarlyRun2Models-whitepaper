\svnidlong
{$HeadURL$}
{$LastChangedDate$}
{$LastChangedRevision$}
{$LastChangedBy$}
\svnid{$Id$}
\pagestyle{fancy}
\fancyhead{}
\fancyhead[ol]{\svnrev;\svndate;\svnauthor}

Dark matter has not yet been seen in particle physics experiments, and
there is not yet any evidence for non-gravitational interactions
between dark matter and Standard Model particles.  If such
interactions exist, pairs of dark matter particles could be produced
at the LHC. Dark matter particles do not produce signals in the LHC
detectors, so simple pair-production of dark matter particles would
not be visible. However, there are many ways for a quark or gluon to
emit a visible particle (X = g, q, gamma, Z, W, or h) before
annihilating to form a dark matter pair. Such reactions, which are
seen as particles or jets recoiling against an invisible state, are
called ``mono-X'' or \MET{}+X reactions.

Early Run 1 searches for \MET{}+X signatures at ATLAS and CMS employed
a set of effective field theories (EFTs) \cite{Goodman:2010ku} to
model the possible signals. These EFTs assume that production of dark
matter takes place through a contact interaction involving two quarks
or gluons and two dark matter particles. In this case, the missing
energy distribution of the signal depends only on the mass of the dark
matter particles and the Lorentz structure of the interaction. Only the
overall rate is a free parameter. Both experiments studied a variety
EFTs with different spin structures. Provided that the contact
interaction approximation holds, these EFTs also provide a
straightforward way to compare results amongst different collider
searches and with non-collider searches for dark matter.

Nevertheless, it has become clear \cite{Bai:2010hh,Kopp:2011eu,Fox:2011fx,Fox:2011pm,Busoni:2013lha} that a contact
interaction is often not the correct description for the signals to
which the LHC is sensitive. Limits from the LHC on particles that do
not couple to QCD and do not decay to hard leptons are relatively
weak. Such particles can mediate reactions that produce dark matter
particles. If the mass of the mediator is light, the kinematics of the
pair-production reaction will differ from that due to a contact
interaction, requiring new search strategies. While the EFT integrates
out the degrees of freedom of the (heavy) intermediate particle,
``simplified models'' \cite{Alwall:2008ag,Alves:2011wf} with
directly-accessible mediators describe this richer
phenomenology. Appropriate simplified models can be used both to
interpret mono-X searches and to guide the design of complementary
searches for additional signatures.

Many proposals for such models began to emerge, for example
Refs. \cite{An:2012va,An:2012ue,Tait:2013,Buchmueller:2013dya,Bai:2013iqa,Bai:2014osa,An:201489115014,Yavin:14092893,Malik:2014ggr,Harris:2014hga,Buckley:2014fba,Haisch:2015ioa}. At
the end of 2014, ATLAS and CMS organized a forum, the ATLAS-CMS Dark
Matter Forum, to form a consensus with the participation of experts on
theories of dark matter. This is the final report of that forum.

One of the guiding principles of this report is to channel the efforts
of the ATLAS and CMS Collaboration towards a minimal set of dark
matter models that should influence the design of the early Run-2
searches. At the same time, a thorough survey of realistic collider
signals of dark matter is a crucial input to the overall design of the
search program.

The goal of this report is such a survey, though confined within some
broad assumptions and focused on benchmarks for kinematically-distinct
signals which are most urgently needed. As far as time and resources
have allowed, the assumptions have been carefully motivated by
theoretical consensus and comparisons of simulations. But to achieve a
true consensus with only a few months before Run 2 would begin, it was
important to narrow the scope and timescale to the following:

\begin{enumerate}
\item The forum should propose a prioritized, small set of benchmark
  simplified models should be agreed upon by both collaborations for
  Run-2 searches.
\item The forum should standardize the matrix element implementation
  of the simplified models and harmonize other common technical
  details (order of the calculation, showering) as far as practical. It
  would be desirable to have a common choice of LO/NLO, ME-parton
  shower matching and merging, factorization and renormalization
  scales for each of the simplified models. This will also lead to a
  single set of theory uncertainties, which will be easier to deal
  with when comparing results from the two collaborations.
\item The forum could also discuss the conditions under which the EFT
  interpretation may still be desirable.
\item The forum should prepare a report summarizing these items,
  suitable both as a reference for the internal ATLAS and CMS
  audiences and as an explanation for theory and non-collider
  readers. This report represents the views of the participants of the
  forum.
\end{enumerate}

\section{Grounding Assumptions}

The first choice made for the Run-2 benchmarks is on the nature of the
Dark Matter itself: it is assumed to be a single particle, stable on
collider timescales and non-interacting with the detector, and a Dirac
fermion.  The former assumption is reductionistic. The rich particle
content of the Standard Model is circumstantial evidence that the dark
matter sector, which constitutes five times as much of the mass of the
universe, may be more complex than a single particle or a single
interaction. But, as was often the case in the discoveries of the SM,
here only one mediator and one search channel might play a dominant
role in the opening stages of an LHC discovery. The latter assumption
focuses our work on early LHC searches, where small kinematic
differences between models will not matter in a discovery scenario,
and with the imminent re-start of the LHC our report relies heavily on
a large body of existing theoretical work which made this
assumption. Different types of dark matter particles will typically
give similar results. The choice of Dirac fermions permits some
processes forbidden for Majorana fermions. Aside from this, the cases
of Dirac or Majorana fermions or scalars produces only minor changes
in the kinematic distributions of the visible particle, especially
when considering cut-and-count analysis. Thus the choice of Dirac
fermion dark matter is deemed sufficient for the benchmarks aiding the
design of the upcoming Run-2 searches. Nevertheless, a more complete
set of models will certainly be required upon a discovery; see
e.g.~\cite{Cotta:2012nj,Haisch:2013fla,Crivellin:2015wva,} for some
studies of observables that may distinguish amongst these models.

We also assume that Minimal Flavor Violation (MFV) \cite{Chivukula:1987py,Hall:1990ac,Buras:2000dm,D'Ambrosio:2002ex} applies to the
recommended models. By this, we mean that the flavor structure of the
couplings between dark matter and ordinary particles follows the same
structure as the Standard Model. This choice is simple, since no
additional theory of flavor is required, beyond what is already
present in the SM, and it provides a mechanism to ensure that the
models do not violate flavor constraints.  As a consequence, spin 0
resonances exchanged in the \schannel must have couplings to fermions proportional to the SM Higgs couplings. Flavor-safe models can still be constructed beyond the MFV
assumption, for example ~\cite{Agrawal:2014aoa}, and deserve worth further study.
%cite Dolan:2014ska? Also DM@LHC report, if out in time. 

In the parameter scan for the recommended models, we make the
assumption of minimal width for the particles mediating the
interaction between SM and DM.  By this, we mean that only decays
strictly necessary for the self-consistency of the model (e.g.  to DM
and to quarks) are accounted for in the definition of the mediator
width. We forbid any further decays to other invisible particles of
the Dark Sector that may increase the width. Studies within the Forum
show that, for cut and count analyses, the kinematic distributions of
many models and therefore the sensitivity of the search do not depend
significantly on the mediator width, as long as the width remains less
than the mass of the particle.

The particle content of the models chosen as benchmarks is limited to
one single kind of DM whose self-interactions are not relevant for LHC
phenomenology, and to one type of SM/DM interaction at a time. These
assumptions only add a limited number of new particles to the
SM. These simplified models, independently sought by different
experimental analyses, can be used as starting points to build more
complete theories. Even though this factorized picture does not always
lead to full theories and leaves out details that are necessary for
the self-consistency of single models (e.g. the mass generation for
mediator particles), it is a starting point to prepare a set of
distinct but yet complementary collider searches for Dark Matter, as
it leads to benchmarks that are easily comparable across channels.

Contact interaction operators (EFTs) are also part of this report, 
to be considered whenever neither a simplified model completion nor other simplified models 
yielding similar kinematic distributions are available. 
This is the case for dimension 5-7 operators with direct 
Dark Matter-electroweak boson couplings. 
Model-independent results should be emphasized 
in the signal regions that are unique to these models, 
and results should be considered with the necessary caution due to EFT 
validity issues covered in the last chapter in this report. 
Results from these models may nevertheless help validating the 
contact interaction limit of new simplified models developed to complete 
these specific operators. 

\section{Structure of this document}

The first chapter of the report is dedicated to simplified
models with radiation of a hard object either from the initial state
or from the mediator. These models produce primarily monojet signatures, 
but are recommended for all \MET{}+X searches.
Details of the implementation of these models in
Monte Carlo generators are provided in
Appendix \ref{app:MonojetLikeModels_Appendix}.
Chapter~\ref{subsec:DMHFModels} describes the specific details of the general models
for final states including heavy flavor quarks, leaving the implementation
details to appendix~\ref{app:TTBar_Xsecs}.  
Chapter~\ref{subsec:EWSpecificModels} contains the benchmark model
recommendations and choices for final states containing an electroweak 
boson ($W/Z/gamma/H$). 
Simplified models from chapter~\ref{subsec:MonojetLikeModels}
are considered, together with specific simplified models leading to mono-boson signatures
and contact interaction operators wherever simplified models are not available. 
The final chapter is devoted to the treatment of the validity of benchmark
models from contact interaction operators.
The technical implementation and further models that can be studied
beyond early searches are in appendix~\ref{app:EWSpecificModels_Appendix}. 
Appendix~\ref{app:Presentation_Of_Experimental_Results} contains the necessary elements that
should be included in the results of experimental searches to allow for further reinterpretation. 

%%This could be saved for monotop
%Appendix~\ref{app:ExtraModels} contains a series of models that produce interesting \MET{}+X signatures. 
%These models are not specifically recommended by the Forum as they
% deviate from the general assumptions made in this introduction. 
%They have however been used in previous ATLAS and CMS analyses and discussed thoroughly within the Forum. 
%They are therefore worth considering for early Run-2 searches, since they lead to unique signatures 
%that are not produced by any other of the recommended models. 
%\fi



