\svnidlong
{$HeadURL$}
{$LastChangedDate$}
{$LastChangedRevision$}
{$LastChangedBy$}
\svnid{$Id$}

Many theories of physics beyond the Standard Model predict the existence
of a stable, neutral, weakly-interacting and massive particle that is
a putative dark matter candidate.   In the following, we refer to such
matter as dark matter, even though the observation of such matter at a collider
could only establish that it is neutral, weakly-interactive, massive and stable
on the distance-scales of 10's of meters.
Dark matter has not yet been observed in particle physics experiments, and
there is not yet any evidence for non-gravitational interactions
between dark matter and Standard Model particles.  If such
interactions exist, dark matter particles could be produced
at the LHC. Since dark matter particles of themselves do not produce signals
in the LHC
detectors, one way to observe them is when they are produced in association
with a visible SM particle X(=$g, q, \gamma, Z, W$, or $h$).
%% so simple pair-production of dark matter particles, for example,
%% would
%% not be visible.
%% However, there are different mechanisms by which a
%% could be produced in association with dark matter.
Such reactions, which are
observed as particles or jets recoiling against an invisible state, are
called ``mono-X'' or \MET{}+X reactions.
This type of reaction includes the production of a Higgs boson that
decays to pairs of dark matter particles.


Early Run-1 searches for \MET{}+X signatures at ATLAS and CMS employed
a basis of operators in 
effective field theories (EFTs) \cite{Goodman:2010yf,Goodman:2010ku} to
calculate the possible signals.
\Todo{Check these references here}
These particular EFTs assume that production of dark matter takes place through
a contact interaction involving a quark-antiquark pair or two gluons and
two dark matter particles.  In this case,
the missing
energy distribution of the signal is determined
by the mass of the dark
matter particles and the Lorentz structure of the interaction. Only the
overall production rate is a free parameter to be constrained or measured.
Both experiments studied a variety
of EFTs with different spin structures. Provided that the contact
interaction approximation holds, these EFTs can also provide a
straightforward way to compare the results from different collider
searches with non-collider searches for dark matter.  


The interpretation of some Run-1 LHC data as limits on contact interactions
provided results that were outside the range of applicability of the EFT
description, requiring some thought on how to handle this situation \cite{Bai:2010hh,Kopp:2011eu,Fox:2011fx,Fox:2011pm,Shoemaker:2011vi,Busoni:2013lha}.
Some ``simplified models'' \cite{Alwall:2008ag,Goodman:2011jq,Alves:2011wf}
of dark matter production
were constructed with particles and interactions beyond the SM ones.
These models can be used consistently at LHC energies,
and make some different predictions than the EFTs.
Many proposals for such models have emerged (see, for example
Refs. \cite{An:2012va,An:2012ue,Tait:2013,Buchmueller:2013dya,Bai:2013iqa,Bai:2014osa,An:2013xka,Yavin:14092893,Malik:2014ggr,Harris:2014hga,Buckley:2014fba,Haisch:2015ioa}). 
All of them introduce a new particle -- a mediator -- that allows
interactions between SM and DM particles.
In the limit of large mediator mass, these simplified models map onto
EFT operators.
%% that a contact
%% interaction is often not the correct description for the signals to
%% which the LHC is sensitive.
%Limits from the LHC on particles that are
%QCD singlets and do not decay to hard leptons are relatively
%weak (see, for example, limits on dijet resonances for weakly-coupled
%theories).
%Such particles can mediate reactions that produce dark matter
%particles.
If the mass of the mediator is light, the kinematics of the
dark matter production can differ substantially from that due to a contact
interaction, modifying limits and possibly requiring new search strategies.
Appropriate simplified models can be used both to
interpret mono-X searches and to guide the design of complementary
searches for additional signatures.

At the end of 2014, ATLAS and CMS organized a forum, the ATLAS-CMS Dark
Matter Forum, to form a consensus on the use of simplified models
and EFTs for Run-2 with the participation of experts on
theories of dark matter. This is the final report of that forum.

One of the guiding principles of this report is to channel the efforts
of the ATLAS and CMS Collaboration towards a minimal set of dark
matter models that should influence the design of the early Run-2
searches. At the same time, a thorough survey of realistic collider
signals of dark matter is a crucial input to the overall design of the
search program.

The goal of this report is such a survey, though confined within some
broad assumptions and focused on benchmarks for kinematically-distinct
signals which are most urgently needed. As far as time and resources
have allowed, the assumptions have been carefully motivated by
theoretical consensus and comparisons of simulations. But, to achieve a
true consensus in only a few months before the start of Run-2, it was
important to narrow the scope and timescale to the following:

\begin{enumerate}
\item The forum should propose a prioritized, compact set of benchmark
  simplified models that should be agreed upon by both collaborations for
  Run-2 searches. The values for the scan on the parameters of the models for which
  experimental results are provided should be specified, to facilitate theory reinterpretation 
  beyond the necessary model-independent limits that 
  should be provided by all LHC Dark Matter searches. 
\item The forum should standardize the event generator implementation
  of the simplified models and harmonize other common technical
  details as far as practical. It
  would be desirable to have a common choice of LO/NLO, ME-parton
  shower matching and merging, factorization and renormalization
  scales for each of the simplified models. This will also lead to a
  common set of theory uncertainties, which will facilitate the
  comparison of results between the two collaborations.
\item The forum could also discuss how to apply the
  EFT formalism and present the results of EFT
  interpretations.
\item The forum should prepare a report summarizing these items,
  suitable both as a reference for the internal ATLAS and CMS
  audiences and as an explanation for theory and non-collider
  readers. This report represents the views of the participants of the
  forum.
\end{enumerate}

\section{Grounding Assumptions}

We assume that interactions exist between Standard Model hadrons
and whatever constitutes cosmological dark matter. If this
is not the case, then proton collisions will not produce dark matter
particles, and dark matter will not scatter off nuclei in direct
detection experiments.

The Dark Matter itself is assumed to be a single particle, a Dirac
fermion WIMP, stable on collider timescales and non-interacting with
the detector.  
The former assumption is reductionistic.
The rich
particle content of the Standard Model is circumstantial evidence that
the dark matter sector, which constitutes five times as much of the
mass of the universe, may be more complex than a single particle or a
single interaction. But, as was often the case in the discoveries of
the SM, here only one mediator and one search channel might play a
dominant role in the opening stages of an LHC discovery. The latter
assumption focuses our work on early LHC searches, where small
kinematic differences between models will not matter in a discovery
scenario, and with the imminent re-start of the LHC our report relies
heavily on a large body of existing theoretical work which made this
assumption. Different types of dark matter particles will typically
give similar results.
Some exceptions exist:  the choice of Dirac fermions forbids some
processes that are allowed for Majorana fermions.
Aside from this, the cases
of Dirac or Majorana fermions or scalars produces only minor changes
in the kinematic distributions of the visible particle, especially
when considering cut-and-count\sidenote{Cut-and-count refers to an analysis
that applies a certain event selection and checks the inclusive number of events which pass.   This is to be constrasted with a shape analysis that compares the distribution of events.} analysis. Thus the choice of Dirac
fermion dark matter is deemed sufficient for the benchmarks aiding the
design of the upcoming Run-2 searches. Nevertheless, a more complete
set of models will certainly be required upon a discovery; see
e.g.~\cite{Cotta:2012nj,Haisch:2013fla,Crivellin:2015wva} for some
studies of observables that may distinguish amongst these models.

The weakness of the dark matter--visible matter interactions is explained
by assuming a mediating force or particle that indirectly connects the
two.  In the EFT approach, the mass scale of the mediator is assumed
to be large compared to the typical momentum involved in the production
of dark matter.   Since EFTs have been established for some time,
we focus on 
simplified models that allow the exploration of scenarios where the
mediating scale is not so large.
%because the EFT approach, viable for the
%other types of dark matter searches, breaks down at the LHC. Another
%way of viewing this problem is that any BSM model of dark matter
%requires at least two particles: a dark matter candidate and a
%mediator of the interaction between the dark matter and the Standard
%Model.
One strength of collider experiments lies in the ability to
study the mediator, not just the DM. A discovery of an
anomalous \MET signature
at the LHC would not uniquely imply discovery of dark matter, while at
the same time discovery of an anomalous and annually-modulated signal
in a direct-detection experiment would leave unanswered many questions
about the nature of the interaction. The these two approaches, and
other types of non-collider searches, provide complementary ways to
approach the problem, and it is in this spirit that our focus is on
the mediator. We systematically explore the basic possibilities for
mediators of various possible spins and couplings.
All models are assumed to produce a signature with pairs of dark matter particles.
Though more varied and
interesting possibilities are added to the literature almost daily,
these basic building blocks account for much of the physics studied at
hadron colliders in the past three decades.


We also assume that Minimal Flavor Violation (MFV) \cite{Chivukula:1987py,Hall:1990ac,Buras:2000dm,D'Ambrosio:2002ex} applies to the
recommended models. This means that the flavor structure of the
couplings between dark matter and ordinary particles follows the same
structure as the Standard Model. This choice is simple, since no
additional theory of flavor is required, beyond what is already
present in the SM, and it provides a mechanism to ensure that the
models do not violate flavor constraints.  As a consequence, spin-0
resonances must have couplings to fermions proportional to the SM Higgs couplings. Flavor-safe models can still be constructed beyond the MFV
assumption, for example ~\cite{Agrawal:2014aoa}, and deserve further study.
%cite Dolan:2014ska? Also DM@LHC report, if out in time. 

In the parameter scan for the recommended models, we make the
assumption of a minimal decay width for the particles mediating the
interaction between SM and DM.  This means that only decays
strictly necessary for the self-consistency of the model (e.g.  to DM
and to quarks) are accounted for in the definition of the mediator
width. We forbid any further decays to other invisible particles of
the Dark Sector that may increase the width or produce striking, visible signatures. Studies within the Forum
show that, for cut-and-count analyses, the kinematic distributions of
many models and therefore the sensitivity of the search do not depend
significantly on the mediator width, as long as the width remains smaller
than the mass of the particle.


The particle content of the models chosen as benchmarks is limited to
one single kind of DM whose self-interactions are not relevant for LHC
phenomenology, and to one type of SM/DM interaction at a time. These
assumptions only add a limited number of new particles to the
SM. These simplified models, independently explored by different
experimental analyses, can be used as starting points to build more
complete theories. Even though this factorized picture does not always
lead to full theories and leaves out details that are necessary for
the self-consistency of single models (e.g. the mass generation for
mediator particles), it is a starting point to prepare a set of
distinct but complementary collider searches for Dark Matter, as
it leads to benchmarks that are easily comparable across channels.

Contact interaction operators (EFTs) are also part of this report, 
to be considered whenever neither a simplified model completion 
nor other simplified models 
yielding similar kinematic distributions are available. 
This is the case for dimension 5-7 operators with direct 
Dark Matter-electroweak boson couplings. 
Model-independent results should be emphasized 
in the signal regions that are unique to these models, 
and results should be considered with the necessary caution due to EFT 
validity issues covered in the last chapter in this report. 
Results from these models may nevertheless help validate the 
contact interaction limit of new simplified models developed to complete 
these specific operators. 

The first chapter of the report is dedicated to simplified
models with radiation of a hard object either from the initial state
or from the mediator. These models produce primarily monojet signatures, 
but are recommended for all \MET{}+X searches.
Details of the implementation of these models in
Monte Carlo generators are provided in
Appendix \ref{app:MonojetLikeModels_Appendix}.
%Chapter~\ref{subsec:DMHFModels} describes the specific details of the general models
%for final states including heavy flavor quarks, leaving the implementation
%details to appendix~\ref{app:TTBar_Xsecs}.  
Chapter~\ref{subsec:EWSpecificModels} contains the benchmark model
recommendations and choices for final states specifically containing an electroweak 
boson ($W/Z/\gamma/H$). In this case, both
simplified models leading to mono-boson signatures
and contact interaction operators wherever simplified models are not available are considered. 
The final chapter is devoted to the treatment of the validity of benchmark
models from contact interaction operators.
The technical implementation and further models that can be studied
beyond early searches are in appendix~\ref{app:EWSpecificModels_Appendix}. 
Appendix~\ref{app:Presentation_Of_Experimental_Results} contains the necessary elements that
should be included in the results of experimental searches to allow for further reinterpretation. 
%The Appendices also collect information about the implementation of the
%models in event generator tools, as well as tables of expected cross sections.
%%This could be saved for monotop
%Appendix~\ref{app:ExtraModels} contains a series of models that produce interesting \MET{}+X signatures. 
%These models are not specifically recommended by the Forum as they
% deviate from the general assumptions made in this introduction. 
%They have however been used in previous ATLAS and CMS analyses and discussed thoroughly within the Forum. 
%They are therefore worth considering for early Run-2 searches, since they lead to unique signatures 
%that are not produced by any other of the recommended models. 
%\fi


