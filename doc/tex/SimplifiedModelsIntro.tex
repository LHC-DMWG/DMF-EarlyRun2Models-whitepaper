\svnidlong
{$HeadURL$}
{$LastChangedDate$}
{$LastChangedRevision$}
{$LastChangedBy$}
\svnid{$Id$}

Dark matter has not yet been seen in particle physics experiments, and
there is no evidence for non-gravitational interactions between dark
matter and Standard Model particles.  If such interactions exist,
pairs of dark matter particles could be produced at the LHC, leading
to so-called 'mono-X' signatures.

Run 1 searches for 'mono-X' signatures at ATLAS and CMS employed an
effective field theory \cite{Goodman:2010ku} to quantify the reach of
the analyses and to allow comparison with direct dark matter detection
experiments. There were several reasons for this.  First, the
presentation is model independent.  Secondly, the approach is
economical, requiring only a few simulated datasets.  Finally, at the
time, there was no catalogue of alternative interpretations of the
data. The downside of the EFT approach is that many of the implicit
limitations were not widely understood.

From the Run 1 results, it has become clear \cite{Busoni:2013lha} that
a contact interaction is often not the correct description for such
signals. The LHC is exploring a regime where the mediating process may
play a more interesting role, leading to missing energy signals that
can be kinematically distinct from those predicted by the EFT, as well
as qualitatively different signatures requiring new search
strategies. While the EFT integrates out the degrees of freedom of the
(heavy) intermediate particle, concrete ``simplified models'' with
low-mass mediators \cite{Alves:2011wf} can describe this richer
phenomenology. Such models can be used both to interpret mono-­X
searches and to guide the design of complementary searches for
additional signatures. Many proposals for such models began to emerge;
see for example
Refs. \cite{Tait:2013,Buchmueller:2013dya,Yavin:14092893,Malik:2014ggr,Harris:2014hga,Buckley:2014fba}.

To form a consensus on the simplified models most appropriate to use,
at the end of 2014 ATLAS and CMS organized a forum, the ATLAS-­CMS
Dark Matter Forum, whereby all interested experimenters from both
collaborations would discuss a minimal common set of Run-2 dark matter
interpretations with the participation of experts on theories of dark
matter. This is the final report of that forum.

To achieve a true consenus with only a few months before Run 2 would
bein, it was important to narrow the scope and timescale of this
forum. These were the goals:

\begin{enumerate}
\item A prioritized, small set of benchmark simplified models should
  be agreed upon by both collaborations for Run-2 searches.
\item The matrix element implementation of the simplified models
  should be standardized, and other common technical details (order of
  the calculation, showering) harmonized as much as practical. It
  would be desirable to have a common choice of LO/NLO, ME-­parton
  shower matching and merging, factorization and renormalization
  scales for each of the simplified models. This will also lead to a
  single set of theory uncertainties, which will be easier to deal
  with when comparing results from the two collaborations.
\item The forum could also discuss the conditions under which the EFT
  interpretation may still be desirable.
\item The forum should prepare a report summarising these items,
  suitable both as a reference for the internal ATLAS and CMS
  audiences and as an explanation for theory and non-collider
  readers. This report represents the views of the participants of the
  forum.
\end{enumerate}

There has indeed arisen a consensus on a handful of simple models
of dark matter production that have a hope of mapping into a more
complete theory.  This document describes them.  Also, in some cases,
there is no good alternative to using EFT.  Even in the case of simplified
models, we recommend showing the EFT limit.

One of the guiding principles of this report is to channel the efforts
of the ATLAS and CMS Collaboration towards obtaining experimental
results, rather than on providing elaborate interpretations and
re-interpretations of these results. At the same time, a thorough
survey of realistic collider signals of dark matter is a crucial input
to the design of such searches.

The goal of this report is such a survey, though confined within some
broad assumptions and focused on benchmarks for kinematically-distinct
signals which are most urgently needed for the early Run-2 mono-X searches. As
far as time and resources have allowed, the assumptions have been
carefully motivated by theoretical consensus and comparisons of simulation.

The first choice made for the Run-2 benchmarks is on the nature of the
Dark Matter itself: it is assumed to be a particle, stable on collider
timescales and non-interacting with the detector, and a Dirac fermion.
The latter is a practical choice, given that the Forum relied upon a large body of theoretical work which made the same assumption. Many of the conclusions in this
report should also apply to MET signals of Majorana fermion DM or
scalar DM with some minor modification to account for their different
angular distributions, especially when considering cut-and-count
analysis. Thus the choice of Dirac fermion dark matter is deemed
sufficient for the benchmarks aiding the design of the upcoming Run-2
searches. Nevertheless, a more complete set of models will certainly
be required upon a discovery; see e.g.~\cite{Crivellin:2015wva} for an
example of an observable that may distinguish the spin of the DM
particle.

We also assume that Minimal Flavor Violation (MFV) applies 
to the recommended models. This means that 
the flavor structure of the couplings between dark matter and ordinary particles
follow the CKM matrix. This choice is both natural, as there is no required 
additional theory of flavor beyond what is already present in the SM, and 
and useful to find benchmarks that are not already ruled out by flavor constraints. 
As a consequence, spin-0 resonances exchanged in the s-channel 
must have Yukawa coupings to SM fermions. Flavor-safe models can still 
be constructed beyond the MFV assumption~\ref{Agrawal:2014aoa},
but their phenomenology is left for future studies. 
%cite Dolan:2014ska? Also DM@LHC report, if out in time. 

In the parameter scan for the recommended models, we make the assumption
of minimal width for the particles mediating the interaction between SM and DM. 
This means that only the decays that are strictly necessary for the self-consistency of the model (e.g.
to DM and to quarks) are accounted for in the definition of the mediator width. In turn,
this forbids any further decays to other invisible particles of the Dark Sector that may increase
the width option. However,  studies within the Forum show that, 
for cut and count analyses, the kinematic distributions of many models and therefore the sensitivity of the search
do not depend significantly on the mediator width up to extreme values.  

The particle content of the models chosen as benchmarks is limited to one single kind of DM
whose self-interactions are not relevant for LHC phenomenology, and to one type of 
SM/DM interaction at a time. These assumptions only add a limited number of new particles
to the SM. These simplified models, independently sought by different experimental analyses,
can be used as starting points to build more complete theories. Even though 
this factorized picture does not always lead to full theories and leaves out details that are necessary
for the self-consistency of single models (e.g. the mass generation for mediator particles), 
it is a starting point to prepare a set of distinct but yet complementary collider searches for Dark Matter, 
as it leads to benchmarks that are easily comparable across channels. 
