\svnidlong
{$HeadURL$}
{$LastChangedDate$}
{$LastChangedRevision$}
{$LastChangedBy$}
\svnid{$Id$}

Run 1 searches for 'mono-X' signatures at ATLAS and CMS employed an
effective field theory \cite{Goodman:2010ku} to quantify the reach of
the analyses and to allow comparison with direct dark matter detection
experiments. There were several reasons for this.  First, the
presentation is model independent.  Secondly, the approach is
economical, requiring only a few simulated datasets.  Finally, at the
time, there was no catalogue of alternative interpretations of the
data. The downside of the EFT approach is that many of the implicit
limitations were not widely understood.

From the Run 1 results, it has become clear that in certain cases
\cite{Busoni:2013lha} that the LHC can directly probe additional
mediating particles that the EFT neglects. While the EFT integrates
out the degrees of freedom of the (heavy) intermediate particle,
concrete "simplified" model with low-­mass mediators can provide a
complementary interpretation. Those models can be used both to
interpret mono-­X searches and to guide the design of complementary
searches for additional signatures.

To forum a consensus on the simplified models most appropriate to use,
at the end of 2014 ATLAS and CMS organized a forum, the ATLAS-­CMS
Dark Matter Forum, whereby all interested experimenters from both
collaborations would discuss a minimal common set of Run-2 dark matter
interpretations with the participation of experts on theories of dark
matter. This is the final report of that forum.

To achieve a true consenus with only a few months before Run 2 would
bein, it was important to narrow the scope and timescale of this
forum. These were the goals:

\begin{enumerate}
\item A prioritized, small set of benchmark simplified models should
  be agreed upon by both collaborations for Run-2 searches.
\item The matrix element implementation of the simplified models
  should be standardized, and other common technical details (order of
  the calculation, showering) harmonized as much as practical. It
  would be desirable to have a common choice of LO/NLO, ME-­parton
  shower matching and merging, factorization and renormalization
  scales for each of the simplified models. This will also lead to a
  single set of theory uncertainties, which will be easier to deal
  with when comparing results from the two collaborations.
\item The forum could also discuss the conditions under which the EFT
  interpretation may still be desirable.
\item The forum should prepare a report summarising these items,
  suitable both as a reference for the internal ATLAS and CMS
  audiences and as an explanation for theory and non-collider
  readers. This report represents the views of the participants of the
  forum.
\end{enumerate}

There has indeed arisen a consensus on a handful of simple models
of dark matter production that have a hope of mapping into a more
complete theory.  This document describes them.  Also, in some cases,
there is no good alternative to using EFT.  Even in the case of simplified
models, we recommend showing the EFT limit.

Assumptions and simplifications made in this report:
\begin{itemize}
\item choice of Dark Matter type: Dirac (unless specified otherwise)
  and what we might be missing
\item Minimal Flavor Violation (MFV) -- new physics appears only
  through Yukawa couplings -- and what we might be missing
\item minimal width for mediators
\item minimal particle content
\end{itemize}
\todo{Explain the above}
