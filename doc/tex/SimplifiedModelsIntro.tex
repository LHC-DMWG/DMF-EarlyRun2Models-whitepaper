\svnidlong
{$HeadURL$}
{$LastChangedDate$}
{$LastChangedRevision$}
{$LastChangedBy$}
\svnid{$Id$}

Many Run-1 results of dark matter searches were interpreted in the
context of Effective Field Theory (EFT).  There were several reasons
for this.  First, the presentation is model independent.   Secondly,
the approach is economical, requiring only a few simulated datasets.
Finally, at the time, there was no catalogue of alternative
interpretations of the data.   The downside of the EFT approach is that
many of the implicit limitations were not widely understood.

Since then, there has arisen a consensus on a handful of simple models
of dark matter production that have a hope of mapping into a more
complete theory.  This document describes them.  Also, in some cases,
there is no good alternative to using EFT.  Even in the case of simplified
models, we recommend showing the EFT limit.

Assumptions and simplifications made in this report:
\begin{itemize}
 \item choice of Dark Matter type: Dirac (unless specified otherwise) and what we might be missing
 \item Minimal Flavor Violation (MFV) -- new physics appears only through Yukawa couplings -- and what we might be missing
 \item minimal width for mediators
 \item minimal particle content
\end{itemize}
