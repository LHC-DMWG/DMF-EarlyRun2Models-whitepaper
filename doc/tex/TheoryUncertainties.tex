
\Todo{This section describes the technical details of how to vary
  input parameters, but it does not describe the overall strategy of
  what parameters to vary or by how much.}

{\bf [Comment on proper PDF sets to use, concerns about sea quark PDF in b-initiated diagrams (perhaps the latter belongs in the b-flavored DM section]}

\section{powheg}

A comprehensive and careful assessment of theoretical uncertainties
plays a much more important role for the background estimations
(especially when their evaluation is non-entirely data-driven) than it
does for signal simulations. Nevertheless, when using \powheg it is
possible to study scale and PDF errors for the dark matter signals. A
fast reweighting machinery is available in {\sc POWHEG-BOX} that
allows one to add, after each event, new weights according to
different scale or PDF choices, without the need to regenerate all the
events from scratch. 

To enable this possibilty, the variable \texttt{storeinfo\_rwgt} should be set 
to 1 in the \powheg input file when the events are generated for the 
first time\sidenote{Notice that even if the variable is not present, by 
default it is set to one}. After each event, a line starting with 

\begin{verbatim}
#rwgt 
\end{verbatim}

is appended, containing the necessary information to generate extra 
weights. In order to obtain new weights, corresponding to different 
PDFs or scale choice, after an event file has been generated, a line 

\begin{verbatim}
compute_rwgt 1 
\end{verbatim}
should be added in the input file along with the change in parameters
that is desired. For instance, \texttt{renscfact} and
\texttt{facscfact} allow one to study scale variations on the
renormalisation and factorisation scales around a central value. By
running the program again, a new event file will be generated, named
\texttt{<OriginalName>-rwgt.lhe}, with one more line at the end of each event of the form

\begin{verbatim}
#new weight,renfact,facfact,pdf1,pdf2 
\end{verbatim}

followed by five numbers and a character string. The first of these 
numbers is the weight of that event with the new parameters chosen. By 
running in sequence the program in the reweighting mode, several 
weights can be added on the same file. Two remarks are in order.

\begin{itemize} 

\item The file with new weights is always named 
\begin{verbatim}
<OriginalName>-rwgt.lhe
\end{verbatim}
hence care has to be taken to save it as 
\begin{verbatim}
<OriginalName>.lhe
\end{verbatim}
before each iteration of the reweighting procedure. 

\item Due to the complexity of the environment where the program is 
likely to be run, it is strongly suggested as a self-consistency check 
that the first reweighting is done keeping the initial parameters. If 
the new weights are not exactly the same as the original ones, then 
some inconsistency must have happened, or some file was probably 
corrupted. 

\end{itemize} 

\noindent It is possible to also have weights written in the version 3 Les Houches format. 
To do so, in the original run, at least the token 

\begin{verbatim}
lhrwgt_id 'ID'
\end{verbatim}
must be present. The reweighting procedure is the same as described 
above, but now each new run can be tagged by using a different value 
for the \texttt{lhrwgt\_id} keyword. After each event, the following lines will 
appear: 

\begin{verbatim}
<rwgt> 
<wgt id='ID'>
<wgt id='ID1'>
</rwgt> 
\end{verbatim}

A more detailed explanation of what went into the computation of every 
single weight can be included in the <header> section of the event 
file by adding/changing the line 

\begin{verbatim}
lhrwgt_descr 'some info'
\end{verbatim}

in the input card, before each "reweighting" run is performed. Other 
useful keywords to group together different weights are 
\texttt{lhrwgt\_group\_name} and \texttt{lhrwgt\_group\_combine}. 

More detailed information can be obtained by inspecting the document in 
{\namecaps /Docs/V2-paper.pdf} under the common {\namecaps POWHEG-BOX-V2} directory. 

\section{madgraph}


