
\Todo{This section describes the technical details of how to vary
  input parameters, but it does not describe the overall strategy of
  what parameters to vary or by how much.}

{\bf [Comment on proper PDF sets to use, concerns about sea quark PDF in b-initiated diagrams (perhaps the latter belongs in the b-flavored DM section]}

\section{POWHEG}

A comprehensive and careful assessment of theoretical uncertainties
plays a much more important role for the background estimations
(especially when their evaluation is non-entirely data-driven) than it
does for signal simulations. Nevertheless, when using \powheg it is
possible to study scale and PDF errors for the dark matter signals. A
fast reweighting machinery is available in {\sc POWHEG-BOX} that
allows one to add, after each event, new weights according to
different scale or PDF choices, without the need to regenerate all the
events from scratch. 

To enable this possibility, the variable \texttt{storeinfo\_rwgt} should be set 
to 1 in the \powheg input file when the events are generated for the 
first time\sidenote{Notice that even if the variable is not present, by 
default it is set to 1.}. After each event, a line starting with 

\begin{verbatim}
#rwgt 
\end{verbatim}

is appended, containing the necessary information to generate extra 
weights. In order to obtain new weights, corresponding to different 
PDFs or scale choice, after an event file has been generated, a line 

\begin{verbatim}
compute_rwgt 1 
\end{verbatim}
should be added in the input file along with the change in parameters
that is desired. For instance, \texttt{renscfact} and
\texttt{facscfact} allow one to study scale variations on the
renormalization and factorization scales around a central value. By
running the program again, a new event file will be generated, named
\texttt{<OriginalName>-rwgt.lhe}, with one more line at the end of each event of the form

\begin{verbatim}
#new weight,renfact,facfact,pdf1,pdf2 
\end{verbatim}

followed by five numbers and a character string. The first of these 
numbers is the weight of that event with the new parameters chosen. By 
running in sequence the program in the reweighting mode, several 
weights can be added on the same file. Two remarks are in order.

\begin{itemize} 

\item The file with new weights is always named 
\begin{verbatim}
<OriginalName>-rwgt.lhe
\end{verbatim}
hence care has to be taken to save it as 
\begin{verbatim}
<OriginalName>.lhe
\end{verbatim}
before each iteration of the reweighting procedure. 

\item Due to the complexity of the environment where the program is 
likely to be run, it is strongly suggested as a self-consistency check 
that the first reweighting is done keeping the initial parameters. If 
the new weights are not exactly the same as the original ones, then 
some inconsistency must have happened, or some file was probably 
corrupted. 

\end{itemize} 

\noindent It is possible to also have weights written in the version 3 Les Houches format. 
To do so, in the original run, at least the token 

\begin{verbatim}
lhrwgt_id 'ID'
\end{verbatim}
must be present. The reweighting procedure is the same as described 
above, but now each new run can be tagged by using a different value 
for the \texttt{lhrwgt\_id} keyword. After each event, the following lines will 
appear: 

\begin{verbatim}
<rwgt> 
<wgt id='ID'>
<wgt id='ID1'>
</rwgt> 
\end{verbatim}

A more detailed explanation of what went into the computation of every 
single weight can be included in the <header> section of the event 
file by adding/changing the line 

\begin{verbatim}
lhrwgt_descr 'some info'
\end{verbatim}

in the input card, before each "reweighting" run is performed. Other 
useful keywords to group together different weights are 
\texttt{lhrwgt\_group\_name} and \texttt{lhrwgt\_group\_combine}. 

More detailed information can be obtained by inspecting the document in 
{\namecaps /Docs/V2-paper.pdf} under the common {\namecaps POWHEG-BOX-V2} directory. 

\section{\madgraph and \syscalc}


\syscalc is a package for use with \madgraph that can calculate dedicated event weights for certain theoretical systematic uncertainties. The output is an XML-based file with all relative weights embedded. These weights can then be used to account for the different systematic uncertainties, namely variations on the $\alpha_s$ emission scale factors, factorization ($\mu_F$) and normalization ($\mu_R$) scales,  matching scales, and different parton distribution function (PDF) sets.

The requirements of the package as inputs are : 
\begin{itemize}
\item A systematics file (which can be generated by MadGraph 5 v. 1.6.0 or later) \cite{Alwall:1405.0301,Alwall:2011uj}.
\item The Pythia-PGS package (v. 2.2.0 or later) \cite{Sjostrand:2006za}. This is needed only in the case of matching scales variations.
\item The availability of LHAPDF5 \cite{Whalley:2005nh}.
\item A configuration file (i.e. a text file) specifying the parameters to be varied. 
\end{itemize}

\syscalc supports all leading order computations generated in \madgraph including fixed-order computation and matched-merged computation performed in the MLM scheme~\cite{Mangano:2006rw}.
\madgraph stores additional information inside the event in order to have access to all the information required to compute the convolution of the PDFs with the matrix element for the various supported systematics.

Below follows an example configuration file which could serve as an example:\\[2mm]

{\footnotesize
\texttt{\# Central scale factors \\
scalefact: \\
0.5 1 2\\
\# Scale correlation \\
\# Special value -1: all combination (N**2)\\
\# Special value -2: only correlated variation\\
\# Otherwise list of index N*fac\_index + ren\_index\\
\#\ \ \ \ index starts at 0\\
scalecorrelation: \\
-1 \\
\#   $\alpha_s$ emission scale factors \\
alpsfact:\\
0.5 1 2\\
\#   matching scales\\
matchscale:\\
30 40 80\\
\# PDF sets and number of members (optional)\\
PDF:\\
CT10.LHgrid 53\\
MSTW2008nlo68cl.LHgrid\\[3mm]}
}


Without matching/merging, \syscalc\ is able to compute the variation of renormalisation and factorisation scale (parameter \texttt{scalefact}) and the change of PDFs.
The variation of the scales can be done in a correlated and/or uncorrelated way, basically following the value of the \texttt{scalecorrelation} parameter which can take the following values:
\begin{itemize}
\item  -1 : to account for all $N^2$ combinations.
\item  -2 : to account only for the correlated variations.
\item A set of positive values corresponding to the following entries (assuming \emph{0.5, 1, 2} for the  \texttt{scalefact} entry):
\begin{enumerate}
{\footnotesize
\item[0:] $\mu_F =\mu^{\rm orig}_F /2,\, \mu_R =\mu^{\rm orig}_R /2$
\item[1:] $\mu_F =\mu^{\rm orig}_F /2,\, \mu_R =\mu^{\rm orig}_R $
\item[2:] $\mu_F =\mu^{\rm orig}_F /2,\, \mu_R =\mu^{\rm orig}_R * 2$
\item[3:] $\mu_F =\mu^{\rm orig}_F, \phantom{/2}\, \mu_R =\mu^{\rm orig}_R /2$
\item[4:] $\mu_F =\mu^{\rm orig}_F, \phantom{/2}\, \mu_R =\mu^{\rm orig}_R$
\item[5:] $\mu_F =\mu^{\rm orig}_F, \phantom{/2}\, \mu_R =\mu^{\rm orig}_R * 2$
\item[6:] $\mu_F =\mu^{\rm orig}_F *2, \mu_R =\mu^{\rm orig}_R /2$
\item[7:] $\mu_F =\mu^{\rm orig}_F *2, \mu_R =\mu^{\rm orig}_R$
\item[8:] $\mu_F =\mu^{\rm orig}_F *2, \mu_R =\mu^{\rm orig}_R * 2$
}
\end{enumerate}
\end{itemize}

Without correlation, the weight associated to the renormalisation scale is the following:
\begin{equation}
\mathcal{W}^{\mu_R}_{\rm new} =  \frac{\alpha_S^{N}(\Delta*\mu_ R)}{\alpha_S^{N}(\mu_R)} * \mathcal{W}_{\rm orig}, 
\end{equation}
where $\Delta$ is the scale variation considered, $\mathcal{W_{\rm{orig}}}$ and $\mathcal{W_{\rm{new}}}$ are respectively the original/new weights associated to the event. $N$ is the power in the strong coupling for the associated event (interference is not taken account on an event by event basis).
The weight associated to the scaling of the factorisation scale is:
\begin{equation}
\mathcal{W}^{\mu_F}_{\rm {new}} =   \frac{f_{\rm 1,orig} (x_1, \Delta*\mu_F) * f_{\rm 2,orig} (x_2, \Delta*\mu_F)}{f_{\rm 1,orig}(x_1, \mu_F) * f_{\rm 2,orig}(x_2, \mu_F)} * \mathcal{W}_{\rm orig}, 
\end{equation}
where $f_{\rm i,orig}$ are the probabilities from the original PDF set associated to the incoming partons, which hold a proton momentum fraction $x_1$ and $x_2$ for the first and second beam respectively.

The variations for the PDF are given by the corresponding weights associated to the new PDF sets:
\begin{equation}
\mathcal{W}^{\rm PDF}_{\rm new} =  \frac{f_{\rm 1,new} (x_1, \mu_F) * f_{\rm 2,new} (x_2, \mu_F)}{f_{\rm 1,orig} (x_1, \mu_F) * f_{\rm 2,orig} (x_2, \mu_F)}* \mathcal{W}_{\rm orig},
\end{equation}
where $f_{\rm i,new}$ is the new PDF probability associated to parton $i$.

In presence of matching, \madgraph performs already a scale and PDF rescaling, such that the scale of the strong interaction is set according to a parton shower history (selected via a $k_T$ clustering). \syscalc can perform the associated re-weighting (parameter \texttt{alpsfact}) by dividing and multiplying by the associated factor.

For each vertex of the clustering (associated to a scale $\mu_i$), this corresponds to the following factor for a Final State Radiation :
\begin{equation}
\mathcal{W}^{\rm FSR}_{\rm new} = \frac{ \alpha_s (\Delta* \mu_i)} { \alpha_s (\mu_i)}  * \mathcal{W}_{\rm orig},
\end{equation}
and the following expression for Initial State Radiation (associated to a scale $\mu_i$ and fraction of energy $x_i$):
\begin{equation}
\mathcal{W}^{\rm ISR}_{\rm new} = \frac{ \alpha_s (\Delta* \mu_i)} { \alpha_s (\mu_i)} \frac{ \frac{f_a(x_i,\Delta*\mu_i)}{f_b(x_i,\Delta*\mu_{i+1})}} { \frac{f_a(x_i, \mu_i)}{f_b(x_i, \mu_{i+1)}} }
  *\mathcal{W}_{\rm orig},
\end{equation}
where $\mu_{i+1}$ is the scale of the next vertex in the initial state clustering history.

\syscalc can include the weight associated to different merging scales in the MLM matching/merging mechanism (for output of the \texttt{pythia6} package or \texttt{pythia-pgs} package). In that case, the parton-shower keeps track of the scale of the first emission and applies then a veto to account for the minimal allowed value for the matching scale according to the cut performed at parton-level. \syscalc\ will then test for each of the values specified in the parameter \texttt{matchscale} if the event passes the MLM criteria or not. If it does not, then a zero weight is associated to the event, while if it does, then a weight 1 is kept. As a reminder,  those weights are the equivalent of having a (approximate) Sudakov form-factor and removing at the same time the double counting between the events belonging to different multiplicities.\\

Finally, we give an example of the \syscalc output which follows the LHEF v3 format. The following block appears in the header of the output file:

\footnotesize{
\begin{verbatim}
<header>
  <initrwgt>
    <weightgroup type="Central scale variation" combine="envelope">
      <weight id="1"> mur=0.5 muf=0.5 </weight>
      <weight id="2"> mur=1 muf=0.5 </weight>
      <weight id="3"> mur=2 muf=0.5 </weight>
      <weight id="4"> mur=0.5 muf=1 </weight>
      <weight id="5"> mur=1 muf=1 </weight>
      <weight id="6"> mur=2 muf=1 </weight>
      <weight id="7"> mur=0.5 muf=2 </weight>
      <weight id="8"> mur=1 muf=2 </weight>
      <weight id="9"> mur=2 muf=2 </weight>
    </weightgroup>
    <weightgroup type="Emission scale variation" combine="envelope">
      <weight id="10"> alpsfact=0.5</weight>
      <weight id="11"> alpsfact=1</weight>
      <weight id="12"> alpsfact=2</weight>
    </weightgroup>
    <weightgroup type="CT10nlo.LHgrid" combine="hessian">
      <weight id="13">Member 0</weight>
      <weight id="14">Member 1</weight>
      <weight id="15">Member 2</weight>
      <weight id="16">Member 3</weight>
      ...
      <weight id="65">Member 52</weight>
    </weightgroup>
  </initrwgt>
</header>
\end{verbatim}}

\noindent For each event, the weights are then written as follows:
\footnotesize{
\begin{verbatim}
<rwgt>
  <wgt id="1">83214.7</wgt>
  <wgt id="2">61460</wgt>
  <wgt id="3">47241.9</wgt>
  <wgt id="4">101374</wgt>
  ...
  <wgt id="64">34893.5</wgt>
  <wgt id="65">41277</wgt>
</rwgt>
\end{verbatim}}
\normalsize


%\begin{comment}
%%%%%%%%%%%%%%%%%%%%%%%%%%%%%%%
% 
%Some more comments though :
% 
%\begin{enumerate}
%
% \item The \texttt{alpsfact} stands for the variations of the $\alpha_s$ emission and PDF scale which accounts for the reweighting of the vertex in the presence of matching retained from parton history, which in \texttt{MadGraph} comes from the MLM-\texttt{kT} clustering scheme.}
%\item For the scale correlations, for  $\mu_F$ and $\mu_R$  it gets the following values :
%
%\begin{itemize}
%\item  -1 : to account for all combination (N**2)
%\item  -2 : to account only for correlated variations
%\item  Other values > 0 : to account for a list of indexes $N*(\mu_F_{\rm{index}} + \mu_R_{\rm {index} })$ 
%\end{itemize}
%
%\item The choice of the matching scale and effectively the choice on its variations depends on the clustering scheme one uses ie shower \texttt{kT} or \texttt{kT-MLM}. It has been found that usually qcut $\approx 1~ (1.5-2) \times $ xqcut  \footnote{xqcut defines the minimal distance (in $k_T$ measures) in the phase space between the extra partons (u,d,s,c, and also b if maxjetflavor=5 in the \texttt{MadGraph} run\_card).}for the former (latter) case. 
%
%\end{enumerate}
%   \\
%\newline
%
%
%
%
%
%
%In any case, without going into too much details, \texttt{SysCalc} for instance computes the weights for the renormalization scale systematics using the following ansaltz. \\
%
%Weight for renormalization variation : \\
%- central scale:
%for ISR:
%\begin{equation}
%(\mathcal{W_{\rm{new}}}) = \prod_{i=1}^{N_{QCD}}  \alpha_s(alpsfact*kT_i) * \frac{f_a(x_i,kT_i)}{f_b(x_i,kT_i-1)} / \rm{V_{ME}} *(\mathcal{W_{\rm{orig}}}) 
%\end{equation}
%for FSR:
%\begin{equation}
%(\mathcal{W_{\rm{new}}}) = \prod_{i=1}^{N_{QCD}}  \alpha_s(alpsfact*kT_i) / \rm{V_{ME}} *(\mathcal{W_{\rm{orig}}}) 
%\end{equation}
%\newline
%
%- renormalization scale associate to radiation (in presence of matching)
%\begin{equation}
%(\mathcal{W_{\rm{new}}}) = \prod_i \alpha_s(\rm{alpsfact*scale_i}) / \rm{V_{ME}}*(\mathcal{W_{\rm{orig}}}) 
%
%\end{equation}
%
%Where  \texttt{alpsfact} is the $\alpha_s$ emission scale factor, ${N_{QCD}$ is the number of QCD vertices and $\rm V_{ME}$ stands for the value computed for the central value of the corresponded scale at Matrix Element level.
%
%
%\end{comment}

