% word count contributed by Emanuele Re

A prescription for determining the theoretical uncertainties (PDF and scale) from the \powheg weights is being documented by the \powheg authors.

Here, we document some specific settings needed to run the \powheg 
generation for the Dark Matter models: 

% \begin{itemize} 

% \item \powheg can handle the generation of events (be it at LO or NLO) 
% in the 2 different modes explained in the following. The second one 
% is the recommended one. The relevant keywords in the input card are 
% bornsuppfact and bornktmin. 

% \begin{enumerate} 
% \item unweighted events: keywords 

% bornsuppfact: negative or absent 
% bornktmin <PT> 

% This corresponds to run the program in the more straightforward way, 
% but most likely it is not the more convenient choice, as will be 
% explained below. \powheg will generate unweighted events using a sharp 
% lower cut (with value PT) on the leading-jet pt. Since this is a 
% generation cut, the user should make sure that the value used for 
% bornktmin is lower than the actual analysis cut to be eventually 
% used. It is good practice to use as a value in the input card a 
% transverse momentum 10-20 % smaller than the final analysis cut, and 
% check that the final result is independent, by exploring an even 
% smaller value of bornktmin. The drawback of this running mode is that 
% it's difficult to populate well and in a single run both the low-pt 
% region as well as the high-pt tail. 

% \item weighted events: keywords 

% bornsuppfact <PTS> 
% bornktmin <PT> 

% \powheg will now produce weighted events, thereby allowing to generate 
% a single sample that provides sufficient statistics in all signal 
% regions. Events are still generated with a sharp lower cut set by 
% bornktmin, but the bornsuppfact parameter is used to set the event 
% suppression factor according to 

% --> keep eq A1 as is 

% In this way, the events at, for instance, low Etmiss, are suppressed 
% but receive an higher weight, which ensures at the same time higher 
% statistics at high Etmiss. We recommend to set bornsuppfact to 1000. 

% The bornktmin parameter allows to suppress the low Etmiss region 
% even further by starting the generation at a certain value of kT also 
% in this runnning mode. It is recommended to set this parameter to half 
% the lower analysis Etmiss cut, for the event selection used in the 
% CMS/ATLAS monojet analyses for instance the proposed value for 
% bornktmin is 150. However, this parameter should be set keeping in 
% mind the event selection of all the analyses that will use these 
% signal samples and hence a threshold lower than 150 may be required. 

% \end{enumerate} 

% ======================= 

% --> 
% On the usage of fixed or running width. It is my opinion that one 
% should use fixed widths. Although clearly there are limitations, I 
% think this goes into the direction of making all as simple and 
% transparent as possible. Moreover, all the discussions in the last 
% months aw well as the large majority of runs were done with fixed 
% widths. Since there was some interest over the last couple of weeks, 
% here is my suggestion for the \powheg section. 
% <-- 

% \item Remove the runningwidth keyword, or set it to 0, which is the 
% default value. 

% Although running with fixed widths is the recommended option, a 
% running width for the propagator of the \schannel mediator can be 
% selected by setting runningwidth to 1. In this case the denominator 
% of the mediator's propagator 

% Q^2 - M^2 + complexi M \Gamma 

% is replaced by 

% Q^2 - M^2 + complexi Q^2 \Gamma / M 

% where Q is the virtuality of the mediator, and M and \Gamma are its mass 
% and width respectively. 

% ======================= 

% --> 
% at line 1538-1539 and 1542-1543: 
% <-- 

% "In order to increase speed, set foldsci and foldy to 2 and keep 
% foldphi at 1." 

% should be removed, and, at line 1542-1543, I'd change from 

% "Allow negative weights for the DMV model by setting withnegweights 
% to 1." 

% to 

% "When NLO corrections are included (as for instance in DMV), 
% negative-weighted event could happen and should be kept in the event 
% sample, hence withnegweights should be set to 1. If needed, their 
% fraction can be decreased by setting foldsci and foldy to bigger value 
% (2 for instance). foldphi can be kept to 1." 

% ================= 

% --> 
% on the last point on uncertainties, here is my suggested text. 
% <-- 

% A comprehensive and careful assessment of theoretical uncertainties 
% plays a much more important role for the background estimations 
% (especially when their evaluation is non-entirely data-driven) than it 
% does for signal simulations. Nevertheless, using \powheg it is possible 
% to study scale and PDF errors for the DM signal too. A fast 
% reweighting machinery is available in {\sc POWHEG-BOX}, that allows to add, 
% after each event, new weights according to different scale or PDF 
% choices, without the need to regenerate all the events from scratch. 

% To enable this possibilty, the variable storeinfo_rwgt should be set 
% to 1 in the \powheg input file when the events are generated for the 
% first time (notice that even if the variable is not present, by 
% default it is set to one). After each event, a line starting with 

% #rwgt 

% is appended, containing the necessary information to generate extra 
% weights. In order to obtain new weights, corresponding to different 
% pdf’s or scale choice, after an event file has been generated, a line 

% compute_rwgt 1 

% should be added in the input file, and the parameters that need to be 
% changed have to be changed too. For instance, renscfact and facscfact 
% allow to study scale variations on the renormalisation and 
% factorisation scales around a central value. By running the program 
% again, a new event file will be generated, named 
% <OriginalName>-rwgt.lhe, with one more line at the end of each event, 
% of the form 

% #new weight,renfact,facfact,pdf1,pdf2 

% followed by five numbers and a character string. The first of these 
% numbers is the weight of that event with the new parameters chosen. By 
% running in sequence the program in the reweighting mode, several 
% weights can be added on the same file. Two remarks are in order: 

% \begin{itemize} 

% \item the file with new weights is always named 
% <OriginalName>-rwgt.lhe, hence care has to be taken to save it as 
% <OriginalName>.lhe before each iteration of the reweighting procedure. 

% \item due to the complexity of the environment where the program is 
% likely to be run, it is strongly suggested as a self-consistency check 
% that the first reweighting is done keeping the initial parameters. If 
% the new weights are not exactly the same as the original ones, then 
% some inconsistency must have happened, or some file was probably 
% corrupted. 

% \end{itemize} 

% It is possible to also have weights written in the v3 Les Houches format. 
% To do so, in the original run, at least the token 

% lhrwgt_id ’ID’ 

% must be present. The reweighting procedure is the same as described 
% above, but now each new run can be tagged by using a different value 
% for the lhrwgt_id keyword. After each event, the following lines will 
% appear: 

% <rwgt> 
% <wgt id=’ID’> 
% <wgt id=’ID1’> 
% </rwgt> 

% A more detailed explanation of what went into the computation of every 
% single weight can be included in the <header> section of the event 
% file by adding/changing the line 

% lhrwgt_descr ’some info’ 

% in the input card, before each "reweighting" run is performed. Other 
% useful keywords to group together different weights are 
% lhrwgt_group_name and lhrwgt_group_combine. 

% More detailed information can be obtained by inspecting the document in 
% {\sc /Docs/V2-paper.pdf} under the common {\sc POWHEG-BOX-V2} directory. 

