
% text based on initial versions from
% Amelia Brennan (amelia.jean.brennan@cern.ch),
% http://arxiv.org/abs/arXiv:1402.2285, 
% and http://arxiv.org/abs/arXiv:1409.2893

The preceding sections discussed models containg a Dirac WIMP coupled to the SM through exchange of a neutral vector or scalar mediator in an $s$-channel process. A $t$-channel process may couple the SM and DM directly. Under the assumption that $\chi$ is a Standard Model (SM) singlet, the mediating particle, labeled $\phi$, is necessarily charged and coloured. 

This model is parallel to, and partially motivated by, the squark of the MSSM, but in this case the $\chi$ is chosen to be Dirac as above. 

An important difference with respect to the MSSM is that {\bf [illustrate that diagram is no longer forced to be small. Should forum provide formulae to relate our parameters to MSSM searches?]}. 

Following the example of Ref.~\cite{Papucci:2014}, the interaction Lagrangian is written as

\begin{equation}
\mathcal{L}_{\mathrm{int}} = g \sum_{i=1,2,3} (\phi_L^i \bar{Q}_L^i + \phi_{uR}^i \bar{u}_R^i + \phi_{dR}^i \bar{d}_R^i) \chi
\end{equation}
\todo{(Note: \cite{Papucci:2014} uses only i = 1,2, but I think it's fine to extend this to 3 here.)}

where $Q_L^i$, $u_R^i$ and $d_R^i$ are the SM quarks and $\phi_L^i$, $\phi_{uR}^i$ and $\phi_{dR}^i$ are the corresponding mediators, which (unlike the $s$-channel mediators) must be heavier than $\chi$. These mediators have SM gauge representations under $(SU(3), SU(2))_Y$ of $(3,2)_{-1/6}$, $(3,1)_{2/3}$ and $(3,1)_{-1/3}$ respectively. Variations of the model previously studied include coupling to the left-handed quarks only~\cite{Chang:2014, Busoni:2014haa}, to the $\phi_{uR}^i$ \cite{Tait:2013} or $\phi_{dR}^i$ \cite{Papucci:2014, Yavin:14092893}, or some combination~\cite{Bai:201311171, An:201489115014}.

\begin{figure}
  \unitlength=0.0045\linewidth
  \begin{feynmandiagram}[modelTmonojetA]
    \fmfleft{i1,i2}
    \fmfright{o1,o2}
    \fmftop{isr}
    \fmflabel{\Large ${\bar{q}}$}{i1}
    \fmflabel{\Large ${q}$}{i2}
    \fmf{fermion}{i2,visr,v2}
    \fmf{phantom}{v2,pisr,o2}
    \fmf{phantom,tension=0}{pisr,isr}
    \fmf{fermion,tension=0}{v2,o2}
    \fmf{wiggly,tension=0}{v2,o2}
    \fmf{fermion}{i1,v1,o1}
    \fmf{wiggly,tension=0}{v1,o1}
    \fmf{scalar,label={\Large $\phi_{1,,2,,3}$}}{v1,v2}
    \fmflabel{\Large ${\bar{\chi}}$}{o1}
    \fmflabel{\Large ${\chi}$}{o2}
    \fmfdot{v1,v2,visr}
    \fmf{gluon,tension=0}{visr,isr}
    \fmflabel{\Large ${g}$}{isr}
    \fmfdot{v1,v2,visr}
  \end{feynmandiagram}\quad
  \begin{feynmandiagram}[modelTmonojetC]
    \fmfleft{i1,i2}
    \fmfright{o1,o2,o3}
    \fmf{fermion}{i2,v3}
    \fmf{fermion,tension=0}{v3,o3}
    \fmf{wiggly}{v3,o3}
    \fmf{gluon}{i1,v1}
    \fmf{fermion}{v2,v1,o1}
    \fmf{scalar,label={\Large $\phi_{1,,2,,3}$}}{v3,v2}
    \fmf{fermion,tension=0}{v2,o2}
    \fmf{wiggly}{v2,o2}
    \fmflabel{\Large ${q}$}{i2}
    \fmflabel{\Large ${g}$}{i1}
    \fmflabel{\Large $\chi$}{o3}
    \fmflabel{\Large $q$}{o1}
    \fmflabel{\Large $\bar{\chi}$}{o2}
    \fmfdot{v1,v2,v3}
  \end{feynmandiagram}\\\vspace{\baselineskip}
  \begin{feynmandiagram}[modelTmonojetB]
    \fmfleft{i1,i2}
    \fmfright{o1,o2,o3}
    \fmf{gluon}{i1,v1}
    \fmf{fermion}{i2,v1}
    \fmf{fermion,tension=2}{v1,v2}
    \fmf{scalar,label={\Large $\phi_{1,,2,,3}$}}{v2,v3}
    \fmf{fermion}{v3,o3}
    \fmf{fermion,tension=0}{v2,o1}
    \fmf{wiggly}{v2,o1}
    \fmf{fermion,tension=0}{v3,o2}
    \fmf{wiggly}{v3,o2}
    \fmflabel{\Large ${q}$}{i2}
    \fmflabel{\Large ${g}$}{i1}
    \fmflabel{\Large $\chi$}{o1}
    \fmflabel{\Large $q$}{o3}
    \fmflabel{\Large $\bar{\chi}$}{o2}
    \fmfdot{v1,v2,v3}
  \end{feynmandiagram}\quad
  \begin{feynmandiagram}[modelTmonojetD]
    \fmfleft{i1,i2}
    \fmfright{o1,o2,o3}
    \fmf{gluon}{i1,v1}
    \fmf{fermion}{i2,v2}
    \fmf{scalar,label={\Large $\phi_{1,,2,,3}$}}{v2,v1}
    \fmf{scalar,label={\Large $\phi_{1,,2,,3}$}}{v1,v3}
    \fmf{fermion}{v3,o2}
    \fmf{fermion,tension=0}{v3,o1}
    \fmf{wiggly}{v3,o1}
    \fmf{fermion,tension=0}{v2,o3}
    \fmf{wiggly}{v2,o3}
    \fmflabel{\Large ${q}$}{i2}
    \fmflabel{\Large ${g}$}{i1}
    \fmflabel{\Large $\chi$}{o3}
    \fmflabel{\Large $q$}{o2}
    \fmflabel{\Large $\bar{\chi}$}{o1}
    \fmfdot{v1,v2,v3}
  \end{feynmandiagram}\quad
  \begin{feynmandiagram}[modelTmonojetE]
    \fmfleft{i1,i2}
    \fmfright{o1,o2,o3}
    \fmf{fermion}{i2,v3,o2}
    \fmf{fermion}{o1,v1,i1}
    \fmf{scalar,label={$\phi_{1,,2,,3}$}}{v1,v2}
    \fmf{scalar,label={$\phi_{1,,2,,3}$}}{v2,v3}
    \fmf{gluon}{v2,o2}
    \fmf{wiggly,tension=0}{v3,o2}
    \fmf{wiggly,tension=0}{v1,o1}
    \fmflabel{\Large ${q}$}{i2}
    \fmflabel{\Large ${\bar{q}}$}{i1}
    \fmflabel{\Large $\chi$}{o3}
    \fmflabel{\Large $g$}{o2}
    \fmflabel{\Large $\bar{\chi}$}{o1}
    \fmfdot{v1,v2,v3}
  \end{feynmandiagram}
\caption{This is a caption.}
\end{figure}

\begin{figure}
  \unitlength=0.0045\linewidth
  \begin{feynmandiagram}[modelTDijetA]
    \fmfleft{i1,i2}
    \fmfright{o1,o2,o3,o4}
    \fmf{gluon}{i2,v2}
    \fmf{gluon}{i1,v1}
    \fmf{scalar}{v4,v2,v1,v3}
    \fmf{phantom,tension=0,label={$\phi_{1,,2,,3}$}}{v2,v1}
    \fmf{phantom,tension=0,label={$\phi_{1,,2,,3}$}}{v4,v2}
    \fmf{phantom,tension=0,label={$\phi_{1,,2,,3}$}}{v1,v3}
    \fmf{fermion}{o3,v4,o4}
    \fmf{fermion}{o2,v3,o1}
    \fmf{wiggly,tension=0}{v4,o4}
    \fmf{wiggly,tension=0}{v3,o2}
    \fmflabel{\Large ${g}$}{i2}
    \fmflabel{\Large ${g}$}{i1}
    \fmflabel{\Large $\chi$}{o4}
    \fmflabel{\Large $\bar{\chi}$}{o2}
    \fmflabel{\Large $\bar{q}$}{o3}
    \fmflabel{\Large $q$}{o1}
    \fmfdot{v1,v2,v3,v4}
  \end{feynmandiagram}\quad
  \begin{feynmandiagram}[modelTDijetB]
    \fmfleft{i1,i2}
    \fmfright{o1,o2,o3,o4}
    \fmf{gluon}{i2,v2}
    \fmf{gluon}{i1,v1}
    \fmf{scalar}{v4,v2}
    \fmf{scalar}{v1,v3}
    \fmf{phantom,tension=0,label={$\phi_{1,,2,,3}$}}{v4,v2}
    \fmf{phantom,tension=0,label={$\phi_{1,,2,,3}$}}{v1,v3}
    \fmf{phantom,tension=0,label={$\chi$}}{v2,v1}
    \fmf{fermion}{v2,v1}
    \fmf{wiggly,tension=0}{v2,v1}
    \fmf{fermion}{o3,v4,o4}
    \fmf{fermion}{o2,v3,o1}
    \fmf{wiggly,tension=0}{v4,o4}
    \fmf{wiggly,tension=0}{v3,o2}
    \fmflabel{\Large ${g}$}{i2}
    \fmflabel{\Large ${g}$}{i1}
    \fmflabel{\Large $\chi$}{o4}
    \fmflabel{\Large $\bar{\chi}$}{o2}
    \fmflabel{\Large $\bar{q}$}{o3}
    \fmflabel{\Large $q$}{o1}
    \fmfdot{v1,v2,v3,v4}
  \end{feynmandiagram}\\\vspace{\baselineskip}
  \begin{feynmandiagram}[modelTDijetC]
    \fmfleft{i1,i2}
    \fmftop{o4,o3}
    \fmfright{o1,o2}
    \fmf{fermion}{i2,v4,v3}
    \fmf{gluon}{v4,o4}
    \fmf{fermion}{v3,o3}
    \fmf{wiggly,tension=0}{v3,o3}
    \fmf{scalar,label={$\phi_{1,,2,,3}$}}{v3,v2,v1}
    \fmf{gluon}{i1,v2}
    \fmf{fermion}{v1,o2}
    \fmf{fermion}{o1,v1}
    \fmf{wiggly,tension=0}{o1,v1}
    \fmflabel{\Large ${q}$}{i2}
    \fmflabel{\Large ${g}$}{i1}
    \fmflabel{\Large $g$}{o4}
    \fmflabel{\Large $\chi$}{o3}
    \fmflabel{\Large $q$}{o2}
    \fmflabel{\Large $\bar{\chi}$}{o1}
    \fmfdot{v1,v2,v3,v4}
  \end{feynmandiagram}\quad

  \begin{feynmandiagram}[modelTDijetD]
    \fmfleft{i1,i2}
    \fmftop{o4}
    \fmfbottom{o1}
    \fmfright{o2,o3}
    \fmf{fermion}{i1,v1,v2}
    \fmf{fermion}{v3,v4,i2}
    \fmf{gluon}{v4,o4}
    \fmf{gluon}{v1,o1}
    \fmf{scalar,label={$\phi_{1,,2,,3}$}}{v2,v3}
    \fmf{fermion}{o3,v3}
    \fmf{wiggly,tension=0}{o3,v3}
    \fmf{fermion}{v2,o2}
    \fmf{wiggly,tension=0}{v2,o2}
    \fmflabel{\Large ${\bar{q}}$}{i2}
    \fmflabel{\Large $q$}{i1}
    \fmflabel{\Large ${g}$}{o4}
    \fmflabel{\Large ${g}$}{o1}
    \fmflabel{\Large $\chi$}{o2}
    \fmflabel{\Large $\bar{\chi}$}{o3}
    \fmfdot{v1,v2,v3,v4}
  \end{feynmandiagram}\\\vspace{\baselineskip}
  \begin{feynmandiagram}[modelTDijetE]
    \fmfleft{i1,i2}
    \fmfright{o1,o2,o3,o4}
    \fmf{fermion}{i1,v1,o1}
    \fmf{fermion}{o4,v4,i2}
    \fmf{scalar,label={$\phi_{1,,2,,3}$}}{v1,v2,v3,v4}
    \fmf{fermion}{v2,o2}
    \fmf{fermion}{o3,v3}
    \fmf{wiggly,tension=0}{v4,o4}
    \fmf{wiggly,tension=0}{v1,o1}
    \fmflabel{\Large ${\bar{q}}$}{i2}
    \fmflabel{\Large $q$}{i1}
    \fmflabel{\Large ${\bar{q}}$}{o3}
    \fmflabel{\Large ${q}$}{o2}
    \fmflabel{\Large $\chi$}{o1}
    \fmflabel{\Large $\bar{\chi}$}{o4}
    \fmfdot{v1,v2,v3,v4}
  \end{feynmandiagram}\\\vspace{\baselineskip}

\caption{This is a caption.}
\end{figure}

\vspace{5mm}

As for the $s$-channel models, we assume Minimal Flavour Violation (MFV), setting the mediator masses for each flavour; the same logic also applies to the couplings $g$. The available parameters are then

\begin{equation}
\{ m_{\chi},\, M_{\phi},\, g\}.
\end{equation}

In practice, the third mediator mass and coupling could be separated from the other two, if higher order corrections to the MFV prediction arise due to the large top Yukawa coupling -- a common variation is then to define this split between the first two generations and the third, so the parameters are extended to

\begin{equation}
\{ m_{\chi},\, M_{\phi_{1,2}},\,M_{\phi_3},\, g_{1,2},\, g_3\}.
\end{equation}

\vspace{5mm}

The width of each mediator is expressed, using the example of decay to an up quark, as

\begin{equation}
\begin{split}
\Gamma (\phi_i \rightarrow \bar{u}_i \chi) &= \frac{g_i^2}{16 \pi M_{\phi_i}^3}(M_{\phi_i}^2 - m_{u_i}^2 - m_{\chi}^2) 		\\
					   & \times \sqrt{M_{\phi_i}^4 + m_{u_i}^4 + m_{\chi}^4 - 2M_{\phi_i}^2m_{u_i}^2 - 2M_{\phi_i}^2m_{\chi}^2 - 2m_{u_i}^2m_{\chi}^2} \, ,
\end{split}
\end{equation}
this reduces to 

\begin{equation}
\frac{g_i^2 M_{\phi_i}}{16 \pi} \left(1 - \frac{m_{\chi}^2}{M_{\phi_i}^2} \right)^2
\end{equation}
in the limit $M_{\phi_i}, m_{\chi} \gg m_{u_i}$.

\vspace{5mm}

An interesting point of difference with the $s$-channel simplified models is that the mediator can radiate a SM object, such as a jet or gauge boson, thus providing three separate mono-X diagrams which must be considered together in calculations. This model can also give a signal in the di-jet + MET channel when, for example, the $\chi$ is exchanged in the $t$-channel and the resulting $\phi$ pair each decay to a jet + $\chi$. 
