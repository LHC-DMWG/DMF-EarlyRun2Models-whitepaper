
% text initially from Amelia Brennan (amelia.jean.brennan@cern.ch)

An alternative set of simplified models exist where the mediator is exchanged in the $t$-channel, thereby coupling the quark and dark matter particle directly. Under the assumption that $\chi$ is a Standard Model (SM) singlet, the mediating particle, labeled $\phi$, is necessarily charged and coloured. This model is parallel to, and partially motivated by, the squark of the MSSM, but in this case the $\chi$ is chosen to be Dirac. Following the example of Ref.~\cite{Papucci:2014}, the interaction Lagrangian is written as

\begin{equation}
\mathcal{L}_{\mathrm{int}} = g \sum_{i=1,2,3} (\phi_L^i \bar{Q}_L^i + \phi_{uR}^i \bar{u}_R^i + \phi_{dR}^i \bar{d}_R^i) \chi
\end{equation}
\textcolor{red}{(Note: \cite{Papucci:2014} uses only i = 1,2, but I think it's fine to extend this to 3 here.)}
where $Q_L^i$, $u_R^i$ and $d_R^i$ are the SM quarks and $\phi_L^i$, $\phi_{uR}^i$ and $\phi_{dR}^i$ are the corresponding mediators, which (unlike the $s$-channel mediators) must be heavier than $\chi$. These mediators have SM gauge representations under $(SU(3), SU(2))_Y$ of $(3,2)_{-1/6}$, $(3,1)_{2/3}$ and $(3,1)_{-1/3}$ respectively. Variations of the model previously studied include coupling to the left-handed quarks only~\cite{Chang:2014, Busoni:2014haa}, to the $\phi_{uR}^i$ \cite{Tait:2013} or $\phi_{dR}^i$ \cite{Papucci:2014, Yavin:14092893}, or some combination~\cite{Bai:201311171, An:201489115014}.

\vspace{5mm}

Minimal Flavour Violation (MFV) requires that the mediator masses for each flavour be equal; the same logic also applies to the couplings $g$. The available parameters are then

\begin{equation}
\{ m_{\chi},\, M_{\phi},\, g\}.
\end{equation}

In practice, the third mediator mass and coupling could be separated from the other two, if higher order corrections to the MFV prediction arise due to the large top Yukawa coupling -- a common variation is then to define this split between the first two generations and the third, so the parameters are extended to

\begin{equation}
\{ m_{\chi},\, M_{\phi_{1,2}},\,M_{\phi_3},\, g_{1,2},\, g_3\}.
\end{equation}

\vspace{5mm}

The width of each mediator is expressed, using the example of decay to an up quark, as

\begin{equation}
\begin{split}
\Gamma (\phi_i \rightarrow \bar{u}_i \chi) &= \frac{g_i^2}{16 \pi M_{\phi_i}^3}(M_{\phi_i}^2 - m_{u_i}^2 - m_{\chi}^2) 		\\
					   & \times \sqrt{M_{\phi_i}^4 + m_{u_i}^4 + m_{\chi}^4 - 2M_{\phi_i}^2m_{u_i}^2 - 2M_{\phi_i}^2m_{\chi}^2 - 2m_{u_i}^2m_{\chi}^2} \, ,
\end{split}
\end{equation}
this reduces to 

\begin{equation}
\frac{g_i^2 M_{\phi_i}}{16 \pi} \left(1 - \frac{m_{\chi}^2}{M_{\phi_i}^2} \right)^2
\end{equation}
in the limit $M_{\phi_i}, m_{\chi} \gg m_{u_i}$.

\vspace{5mm}

An interesting point of difference with the $s$-channel simplified models is that the mediator can radiate a SM object, such as a jet or gauge boson, thus providing three separate mono-X diagrams which must be considered together in calculations. This model can also give a signal in the di-jet + MET channel when, for example, the $\chi$ is exchanged in the $t$-channel and the resulting $\phi$ pair each decay to a jet + $\chi$. 
