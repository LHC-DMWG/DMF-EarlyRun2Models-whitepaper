\svnidlong
{$HeadURL$}
{$LastChangedDate$}
{$LastChangedRevision$}
{$LastChangedBy$}
\svnid{$Id$}

% text based on initial versions from
% Amelia Brennan (amelia.jean.brennan@cern.ch),
% http://arxiv.org/abs/arXiv:1402.2285, 
% and http://arxiv.org/abs/arXiv:1409.2893

The preceding sections address models with a Dirac fermion coupled to
the SM through exchange of a neutral spin-0 or spin-1 in an
\schannel process.  A \tchannel process may couple the SM and DM
directly, leading to a different phenomenology.  Here, we examine a
model where $\chiDM$ is a Standard Model (SM) singlet, a Dirac
fermion% dropped after comment from Tom Rizzo
%\footnote{Though outside the scope of this report, a colored
%  vector \tchannel mediator with vector dark matter may also be
%  worthy of study. The UV completion of such models would look like an
%  RS KK gluon,for example. In these case, it is expcted that the most
%  interesting mediator masses will be of order several TeV.}
; the
mediating particle, labeled $\phi$, is charged and coloured; and the
SM particle is a quark. Such models have been studied in
Refs.~\cite{An:2013xka,Papucci:2014iwa}.

Following the example of Ref.~\cite{Papucci:2014iwa}, the interaction Lagrangian is written as

\begin{equation}
\mathcal{L}_{\mathrm{int}} = g \sum_{i=1,2,3} (\phi_L^i \bar{Q}_L^i + \phi_{uR}^i \bar{u}_R^i + \phi_{dR}^i \bar{d}_R^i) \chiDM
\end{equation}

where $Q_L^i$, $u_R^i$ and $d_R^i$ are the SM quarks and $\phi_L^i$, $\phi_{uR}^i$ and $\phi_{dR}^i$ are the corresponding mediators, which 
(unlike the \schannel mediators) must be heavier than $\chiDM$. \Todo{Note: \cite{Papucci:2014iwa} uses only i = 1,2, but I think it's fine to extend this to 3 here.}
These mediators have SM gauge representations under $(SU(3), SU(2))_Y$ of $(3,2)_{-1/6}$, $(3,1)_{2/3}$ and $(3,1)_{-1/3}$ respectively. Variations of the model previously studied include coupling to the left-handed quarks only~\cite{Chang:2014, Busoni:2014haa}, to the $\phi_{uR}^i$ \cite{Tait:2013} or $\phi_{dR}^i$ \cite{Papucci:2014iwa, Yavin:14092893}, or some combination~\cite{Bai:201311171, An:201489115014}.

\vspace{5mm}

As for the \schannel models, we assume Minimal Flavour Violation (MFV), setting the mediator masses for each flavour equal; the same logic also applies to the couplings $g$. The available parameters are then

\begin{equation}
\{ \mDM,\, \Mphi,\, g\}.
\end{equation}

\vspace{5mm}

In practice, the third mediator mass and coupling could be separated from the other two, if higher order corrections to the MFV prediction arise due to the large top Yukawa coupling --- a common variation is then to define this split between the first two generations and the third, so the parameters are extended to

\begin{equation}
\{ \mDM,\, M_{\phi_{1,2}},\,M_{\phi_3},\, g_{1,2},\, g_3\}.
\end{equation}

For the purposes of the rest of this section, we will assume the universal variant $M_{\phi_{1,2}} = M_{\phi_3}$, $g_{1,2} = g_3$.

\vspace{5mm}

The minimal width of each mediator is expressed, using the example of decay to an up quark, as

\begin{equation}
\begin{split}
\Gamma (\phi_i \rightarrow \bar{u}_i \chiDM) &= \frac{g_i^2}{16 \pi M_{\phi_i}^3}(M_{\phi_i}^2 - m_{u_i}^2 - \mDM^2) 		\\
					   & \times
%\sqrt{M_{\phi_i}^4 + m_{u_i}^4 + \mDM^4 - 2M_{\phi_i}^2m_{u_i}^2 - 2M_{\phi_i}^2\mDM^2 - 2m_{u_i}^2\mDM^2} \, ,
\sqrt{(M_{\phi_i}^2 - (m_{u_i} + \mDM)^2)(M_{\phi_i}^2 - (m_{u_i}-\mDM)^2)} \, ,
\end{split}
\end{equation}
which reduces to 

\begin{equation}
\frac{g_i^2 M_{\phi_i}}{16 \pi} \left(1 - \frac{\mDM^2}{M_{\phi_i}^2} \right)^2
\end{equation}
in the limit $M_{\phi_i}, \mDM \gg m_{u_i}$.

\vspace{5mm}

The leading-order processes involved in MET+jet production are shown
in Fig. \ref{fig:tchannelMonojet}.
Note that the generation index for $\phi$ is linked to the incoming
fermion(s).   Thus, mono-jet production via $\phi^3_u$ is not possible at
this order, while production through $\phi^3_d$ is suppressed by
the $b$ parton PDF.
This model can also give a signal in the di-jet +
MET channel when, for example, the $\chiDM$ is exchanged in the
\tchannel and the resulting $\phi$ pair each decay to a jet +
$\chiDM$. Fig.~\ref{fig:tchannelDijet} shows the leading order diagrams.
Except for the $gg$ induced process, di-jet production
through $\phi^3_u$ is not possible, and production through $\phi^3_d$ is
again suppressed.   The diagram involving the \tchannel exchange
of $\chiDM$ is strongly dependent upon the Dirac fermion assumption.
For a Majorana fermion, $q\bar q,\bar q\bar q,$ and $qq$ production
would be possible with the latter having a pronounced enhancement
at the LHC.


This model is similar to the MSSM with only light squarks and
a neutralino, except for two distinct points:  the $\chiDM$ is
a Dirac fermion and the coupling $g$ is not limited to be
weak scale.
In the MSSM, most of these processes are sub-dominant, even
if resonantly enhanced, because the production is proportional
to weak couplings.
%SM: not true.  Little explicit optimization.
%% A second, crucial different with respect to the MSSM is the
%% contribution from diagrams involving the exchange or production of a
%% single squark $\phi$ and the importance of off-shell $\phi$. Searches for
%% supersymmetry signals are typically optimised for squark pair
%% production, since the equvialent of the DM coupling $g$ is constrained
%% to be very small within the MSSM.
In the more general theories
considered here, $g$ is free to take on large values of order 1 or
more, and thus diagrams neglected in MSSM simulation can occur at a
much higher rate here. While constraints from SUSY jets+MET analyses
on MSSM models should be translated to the specific model in this report, 
DM searches should also test their sensitivity to the MSSM benchmark models.
%study analyses directly optimized for the diagrams which
%are relevant for large $g$.

\begin{figure}
  \unitlength=0.0043\linewidth
  \begin{feynmandiagram}[modelTmonojetA]
    \fmfleft{i1,i2}
    \fmfright{o1,o2}
    \fmftop{isr}
    \fmflabel{\Large ${\bar{q}}$}{i1}
    \fmflabel{\Large ${q}$}{i2}
    \fmf{fermion}{i2,visr,v2}
    \fmf{phantom}{v2,pisr,o2}
    \fmf{phantom,tension=0}{pisr,isr}
    \fmf{fermion,tension=0}{v2,o2}
    \fmf{wiggly,tension=0}{v2,o2}
    \fmf{fermion}{i1,v1,o1}
    \fmf{wiggly,tension=0}{v1,o1}
    \fmf{scalar,label={\Large $\phi_{1,,2,,3}$}}{v1,v2}
    \fmflabel{\Large ${\bar{\chiDM}}$}{o1}
    \fmflabel{\Large ${\chiDM}$}{o2}
    \fmfdot{v1,v2,visr}
    \fmf{gluon,tension=0}{visr,isr}
    \fmflabel{\Large ${g}$}{isr}
    \fmfdot{v1,v2,visr}
  \end{feynmandiagram}\quad
  \begin{feynmandiagram}[modelTmonojetC]
    \fmfleft{i1,i2}
    \fmfright{o1,o2,o3}
    \fmf{fermion}{i2,v3}
    \fmf{fermion,tension=0}{v3,o3}
    \fmf{wiggly}{v3,o3}
    \fmf{gluon}{i1,v1}
    \fmf{fermion}{v2,v1,o1}
    \fmf{scalar,label={\Large $\phi_{1,,2,,3}$}}{v3,v2}
    \fmf{fermion,tension=0}{v2,o2}
    \fmf{wiggly}{v2,o2}
    \fmflabel{\Large ${q}$}{i2}
    \fmflabel{\Large ${g}$}{i1}
    \fmflabel{\Large $\chiDM$}{o3}
    \fmflabel{\Large $q$}{o1}
    \fmflabel{\Large $\bar{\chiDM}$}{o2}
    \fmfdot{v1,v2,v3}
  \end{feynmandiagram}\\\vspace{3\baselineskip}
  \begin{feynmandiagram}[modelTmonojetB]
    \fmfleft{i1,i2}
    \fmfright{o1,o2,o3}
    \fmf{gluon}{i1,v1}
    \fmf{fermion}{i2,v1}
    \fmf{fermion,tension=2}{v1,v2}
    \fmf{scalar,label={\Large $\phi_{1,,2,,3}$}}{v2,v3}
    \fmf{fermion}{v3,o3}
    \fmf{fermion,tension=0}{v2,o1}
    \fmf{wiggly}{v2,o1}
    \fmf{fermion,tension=0}{v3,o2}
    \fmf{wiggly}{v3,o2}
    \fmflabel{\Large ${q}$}{i2}
    \fmflabel{\Large ${g}$}{i1}
    \fmflabel{\Large $\chiDM$}{o1}
    \fmflabel{\Large $q$}{o3}
    \fmflabel{\Large $\bar{\chiDM}$}{o2}
    \fmfdot{v1,v2,v3}
  \end{feynmandiagram}\quad
  \begin{feynmandiagram}[modelTmonojetD]
    \fmfleft{i1,i2}
    \fmfright{o1,o2,o3}
    \fmf{gluon}{i1,v1}
    \fmf{fermion}{i2,v2}
    \fmf{scalar,label={\Large $\phi_{1,,2,,3}$}}{v2,v1}
    \fmf{scalar,label={\Large $\phi_{1,,2,,3}$}}{v1,v3}
    \fmf{fermion}{v3,o2}
    \fmf{fermion,tension=0}{v3,o1}
    \fmf{wiggly}{v3,o1}
    \fmf{fermion,tension=0}{v2,o3}
    \fmf{wiggly}{v2,o3}
    \fmflabel{\Large ${q}$}{i2}
    \fmflabel{\Large ${g}$}{i1}
    \fmflabel{\Large $\chiDM$}{o3}
    \fmflabel{\Large $q$}{o2}
    \fmflabel{\Large $\bar{\chiDM}$}{o1}
    \fmfdot{v1,v2,v3}
  \end{feynmandiagram}\\\vspace{3\baselineskip}
  \begin{feynmandiagram}[modelTmonojetE]
    \fmfleft{i1,i2}
    \fmfright{o1,o2,o3}
    \fmf{fermion}{i2,v3,o3}
    \fmf{fermion}{o1,v1,i1}
    \fmf{scalar,label={$\phi_{1,,2,,3}$}}{v1,v2}
    \fmf{scalar,label={$\phi_{1,,2,,3}$}}{v2,v3}
    \fmf{gluon}{v2,o2}
    \fmf{wiggly,tension=0}{v3,o3}
    \fmf{wiggly,tension=0}{v1,o1}
    \fmflabel{\Large ${q}$}{i2}
    \fmflabel{\Large ${\bar{q}}$}{i1}
    \fmflabel{\Large $\chiDM$}{o3}
    \fmflabel{\Large $g$}{o2}
    \fmflabel{\Large $\bar{\chiDM}$}{o1}
    \fmfdot{v1,v2,v3}
  \end{feynmandiagram}
\caption{Leading order mono-jet \tchannel processes, adapted from \cite{Papucci:2014iwa}.}\label{fig:tchannelMonojet}
\end{figure}

\begin{figure}
  \unitlength=0.0043\linewidth
  \begin{feynmandiagram}[modelTDijetA]
    \fmfleft{i1,i2}
    \fmfright{o1,o2,o3,o4}
    \fmf{gluon}{i2,v2}
    \fmf{gluon}{i1,v1}
    \fmf{scalar}{v4,v2,v1,v3}
    \fmf{phantom,tension=0,label={$\phi_{1,,2,,3}$}}{v2,v1}
    \fmf{phantom,tension=0,label={$\phi_{1,,2,,3}$}}{v4,v2}
    \fmf{phantom,tension=0,label={$\phi_{1,,2,,3}$}}{v1,v3}
    \fmf{fermion}{o3,v4,o4}
    \fmf{fermion}{o2,v3,o1}
    \fmf{wiggly,tension=0}{v4,o4}
    \fmf{wiggly,tension=0}{v3,o2}
    \fmflabel{\Large ${g}$}{i2}
    \fmflabel{\Large ${g}$}{i1}
    \fmflabel{\Large $\chiDM$}{o4}
    \fmflabel{\Large $\bar{\chiDM}$}{o2}
    \fmflabel{\Large $\bar{q}$}{o3}
    \fmflabel{\Large $q$}{o1}
    \fmfdot{v1,v2,v3,v4}
  \end{feynmandiagram}\quad
  \begin{feynmandiagram}[modelTDijetB]
    \fmfleft{i1,i2}
    \fmfright{o1,o2,o3,o4}
    \fmf{fermion}{v2,i2}
    \fmf{fermion}{i1,v1}
    \fmf{scalar}{v4,v2}
    \fmf{scalar}{v1,v3}
    \fmf{phantom,tension=0,label={$\phi_{1,,2,,3}$}}{v4,v2}
    \fmf{phantom,tension=0,label={$\phi_{1,,2,,3}$}}{v1,v3}
    \fmf{phantom,tension=0,label={$\chiDM$}}{v2,v1}
    \fmf{fermion}{v2,v1}
    \fmf{wiggly,tension=0}{v2,v1}
    \fmf{fermion}{o3,v4,o4}
    \fmf{fermion}{o2,v3,o1}
    \fmf{wiggly,tension=0}{v4,o4}
    \fmf{wiggly,tension=0}{v3,o2}
    \fmflabel{\Large ${\bar{q}}$}{i2}
    \fmflabel{\Large ${q}$}{i1}
    \fmflabel{\Large $\chiDM$}{o4}
    \fmflabel{\Large $\bar{\chiDM}$}{o2}
    \fmflabel{\Large $\bar{q}$}{o3}
    \fmflabel{\Large $q$}{o1}
    \fmfdot{v1,v2,v3,v4}
  \end{feynmandiagram}\\\vspace{3\baselineskip}
  \begin{feynmandiagram}[modelTDijetC]
    \fmfleft{i1,i2}
    \fmfright{o1,o2,o3,o4}
    \fmf{fermion}{i2,v4,v3}
    \fmf{gluon}{v4,o4}
    \fmf{fermion}{v3,o3}
    \fmf{wiggly,tension=0}{v3,o3}
    \fmf{scalar,label={$\phi_{1,,2,,3}$}}{v3,v2,v1}
    \fmf{gluon}{i1,v2}
    \fmf{fermion}{v1,o2}
    \fmf{fermion}{o1,v1}
    \fmf{wiggly,tension=0}{o1,v1}
    \fmflabel{\Large ${q}$}{i2}
    \fmflabel{\Large ${g}$}{i1}
    \fmflabel{\Large $g$}{o4}
    \fmflabel{\Large $\chiDM$}{o3}
    \fmflabel{\Large $q$}{o2}
    \fmflabel{\Large $\bar{\chiDM}$}{o1}
    \fmfdot{v1,v2,v3,v4}
  \end{feynmandiagram}\quad
  \begin{feynmandiagram}[modelTDijetD]
    \fmfleft{i1,i2}
    \fmfright{o1,o2,o3,o4}
    \fmf{fermion}{i1,v1,v2}
    \fmf{fermion}{v3,v4,i2}
    \fmf{gluon}{v4,o4}
    \fmf{gluon}{v1,o1}
    \fmf{scalar,label={$\phi_{1,,2,,3}$}}{v2,v3}
    \fmf{fermion}{o3,v3}
    \fmf{wiggly,tension=0}{o3,v3}
    \fmf{fermion}{v2,o2}
    \fmf{wiggly,tension=0}{v2,o2}
    \fmflabel{\Large ${\bar{q}}$}{i2}
    \fmflabel{\Large $q$}{i1}
    \fmflabel{\Large ${g}$}{o4}
    \fmflabel{\Large ${g}$}{o1}
    \fmflabel{\Large $\chiDM$}{o2}
    \fmflabel{\Large $\bar{\chiDM}$}{o3}
    \fmfdot{v1,v2,v3,v4}
  \end{feynmandiagram}\\\vspace{3\baselineskip}
  \begin{feynmandiagram}[modelTDijetE]
    \fmfleft{i1,i2}
    \fmfright{o1,o2,o3,o4}
    \fmf{fermion}{i1,v1,o1}
    \fmf{fermion}{o4,v4,i2}
    \fmf{scalar,label={$\phi_{1,,2,,3}$}}{v1,v2,v3,v4}
    \fmf{fermion}{v2,o2}
    \fmf{fermion}{o3,v3}
    \fmf{wiggly,tension=0}{v4,o4}
    \fmf{wiggly,tension=0}{v1,o1}
    \fmflabel{\Large ${\bar{q}}$}{i2}
    \fmflabel{\Large $q$}{i1}
    \fmflabel{\Large ${\bar{q}}$}{o3}
    \fmflabel{\Large ${q}$}{o2}
    \fmflabel{\Large $\chiDM$}{o1}
    \fmflabel{\Large $\bar{\chiDM}$}{o4}
    \fmfdot{v1,v2,v3,v4}
  \end{feynmandiagram}
\caption{Leading order two-jet \tchannel processes, adapted from \cite{Papucci:2014iwa}.}\label{fig:tchannelDijet}
\end{figure}

\newthought{Parameter scans}

Ref.~\cite{Papucci:2014iwa} studies the parameter space and obtains
bounds on this model from LHC Run 1 mono-jet and dijets+MET data.

As in the \schannel models, scans should be performed over
$\mDM$ and $\Mphi$. The viable ranges of both parameters nearly
coincide with the scan proposed for the \schannel; for simplicity we
recommend adopting the \schannel mono-jet grid.

The rates of the first three diagrams of
Fig.~\ref{fig:tchannelMonojet} scale as one with the coupling $g$. In
the heavy mediator limit, then, the kinematic distributions depend
only indirectly on the coupling through the effect on the minimal
mediator width. In contrast with the \schannel case, however, the
bounds one obtains from X+MET searches depends strongly on the width
of the mediator, as is visible in Figs.~5 and 6 of
Ref.~\cite{Papucci:2014iwa}, except in the heavy mediator limit ($\Mphi \approx 2$\,TeV). A scan over the width was not available for this report; thus we recommend scanning a range of possible widths as discussed in a more-limited way for the \schannel mono-jet, spanning from the minimal width to a value approaching the particle limit (for example, $\Gamma \approx \Mphi/3$ in Ref.~\cite{Papucci:2014iwa}).

\newthought{Implementation}

The MadGraph implementation of the Ref.~\cite{Papucci:2014iwa} is
available from the Forum repository~\cite{ForumSVN_TChannel}, following the
matching and merging prescription described in its Section II and
Appendix A.

