\svnidlong
{$HeadURL$}
{$LastChangedDate$}
{$LastChangedRevision$}
{$LastChangedBy$}
\svnid{$Id: $}

\section{Specific models involving $b$-quarks in the final state}

In some theoretically motivated scenario (e.g. for high $tan\beta$ in 2HDM in the pMSSM), 
\spinzero mediators might couple more strongly to down generation quarks.
 This assumption motivates the study of final states involving $b$-quarks 
 as a complementary search to the $t\bar
t$+DM models presented in the previous section, to directly probe the $b$-quark coupling. 
An example of such a model can be found in Ref.~\cite{Buckley:2014fba}

As in the $t\bar t$ case, detailed studies were performed, analyzing 
these models and their key features. It was found that they show the
same weak dependence of the kinematics of the event on mediator width
variation, except in few corner cases. In addition, it was found that
the same benchmark parameters of the $t\bar t$ case could be chosen.

\newthought{Implementation}
There are some subtleties to the Monte Carlo simulation relevant for
this case that are discussed in Appendix~\ref{app:MonojetLikeModels_Appendix}.

% Removing these plots as they won't make it last-minute. 
%\Todo{[TODO: The following figures are placeholders for now and will be added later].  If these are supporting material for the MC generation, put in the appendix}.
%
%\begin{figure}
%    \vbox{\hfill}
%    \caption{\label{fig:bbscanPhi} Example of the dependence of the kinematics on the scalar mediator mass. 
%    	The Dark Matter mass is fixed to be $1 {\rm GeV}$.}
%\end{figure}
%
%\begin{figure}[!ht]
%    \vbox{\hfill}
%    \caption{\label{fig:bbscanPhiPseudo} Example of the dependence of the kinematics on the pseudoscalar mediator mass. 
%    	The Dark Matter mass is fixed to be $1 {\rm GeV}$.}
%\end{figure}
