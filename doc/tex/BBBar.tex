\svnidlong
{$HeadURL$}
{$LastChangedDate$}
{$LastChangedRevision$}
{$LastChangedBy$}
\svnid{$Id: $}

\label{sec:ttdm}


In some theoretically motivated scenario (e.g. for high $tan\beta$ in 2HDM in the pMSSM), 
spin-0 mediators might couple more strongly to down generation quarks.
 This assumption motivates the study of final states involving $b$-quarks 
 as a complementary search to the $t\bar
t$+DM models presented in the previous section, to directly probe the $b-$ quark coupling. 
An example of such a model can be found in Ref.~\cite{Buckley:2014fba}

As in the $t\bar t$ case, detailed studies were performed, analyzing 
these models and their key features. It was found that they show the
same weak dependence of the kinematics of the event on width
variation, except in few corner cases. In addition, it was found that
the same benchmark parameters of the $t\bar t$ case could be chosen.

In this particular model we recommend an additional care when choosing
the flavor scheme generation. 
It is found that the best modeling of two $b$-quarks final states is
achieved using a 4-flavor scheme and a massive treatment of the
$b$-quarks \Todo{Add reference}.
% We recommend to use in the generation NNPDF3.0 set (lhaid
%263400).
In addition, we recommend to calculate the cross sections of these
models in the 5-flavour scheme, and as in the $t bar t$ case we provide 
values for the suggested coupling scan in the appendix. 
The PDF used to calculate these cross section is NNPDF3.0 (lhaid 263000). 

\Todo{The following figures are placeholders for now and will be added once ready}.

\begin{figure}
  \vbox{\hfill}
  \caption{Comparison of the subleading jet $p_T$ and $b$-jet multiplicity
    for $bb$+DM scalar model generated in the 4-flavour (left) and 5-flavour (right)
    schemes, respectively}
\end{figure}

\begin{figure}
    \vbox{\hfill}
    \caption{\label{fig:bbscanPhi} Example of the dependence of the kinematics on the scalar mediator mass. 
    	The Dark Matter mass is fixed to be $1 {\rm GeV}$.}
\end{figure}

\begin{figure}[!ht]
    \vbox{\hfill}
    \caption{\label{fig:bbscanPhiPseudo} Example of the dependence of the kinematics on the pseudoscalar mediator mass. 
    	The Dark Matter mass is fixed to be $1 {\rm GeV}$.}
\end{figure}
