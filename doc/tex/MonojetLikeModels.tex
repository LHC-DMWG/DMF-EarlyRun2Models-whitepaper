\newcommand{\bra}[1]{\langle #1|}
\newcommand{\ket}[1]{|#1\rangle}
\newcommand{\MET}{\slashed{E}_T}
\newcommand{\mDM}{m_{\rm{DM}}}
\newcommand{\mMed}{M_{\rm{med}}}
\newcommand{\gDM}{g_{\rm{DM}}}
\newcommand{\gq}{g_q}
\paragraph{Vector and axial vector mediator, s-channel exchange}

\begin{itemize}
\item Matrix Element implementations (with references)
\begin{itemize}
 \item Production mechanism
\begin{figure}[t!]
\centering
\includegraphics[width=0.5\columnwidth]{figures/monoLHC.pdf}
\caption{The diagram shows the pair production of dark matter particles in association with a parton from the initial state via an s-channel vector or axial-vector mediator. The process if specified by ($\mMed ,\, \mDM ,\, \gDM ,\, \gq)$, the mediator and dark matter masses, and the mediator couplings to dark matter and quarks respectively.}
\label{fig:OP}
\end{figure}


 \item Lagrangian
We consider the case of a dark matter particle that is a Dirac fermion and where the production proceeds via the exhange of a spin-1 $s$-channel mediator. We consider the following interactions between the DM and SM fields including a vector mediator with:\\
(a) vector couplings to DM and SM.\\
(b) axial-vector couplings to DM and SM.\\

\begin{align}
\label{eq:AV} 
\mathcal{L}_{\mathrm{vector}} &= \sum_q \gq Z'_{\mu} \bar{q}\gamma^{\mu}q + \gDM Z'_{\mu} \bar{\chi}\gamma^{\mu}\chi \\
\mathcal{L}_{\rm{axial}} &= \sum_q \gq Z'_{\mu} \bar{q}\gamma^{\mu}\gamma^5q + \gDM Z'_{\mu} \bar{\chi}\gamma^{\mu}\gamma^5\chi
\end{align}
where the coupling extends over all the quarks and universal couplings are assumed for all the quarks. 
It is also possible to consider another model in which mixed vector and axial-vector couplings are considered, for instance the couplings to the quarks are vector whereas those to DM are axial-vector. As a starting point, we consider only the models with the vector couplings only and axial vector couplings only. Studies have been performed to see if the case of a mixed coupling can be simply extracted from the other models by some reweighting procedure to take account of the difference in cross section. This would assume that the difference between the pure and mixed couplings case does not affect the kinematics of the event. 


 \item Definition of minimal width
We assume that no additional visible or invisible decays contribute to the width of the mediator, this is referred to as the minimal width and it is defined as follows for the vector and axial-vector models.

\begin{equation}
\Gamma_{\rm{min}}=\Gamma_{\bar{\chi}\chi} + \sum_{q}N_{c}\Gamma_{\bar{q}q}
\end{equation}

where the individual contributions to this from the partial width are from,
\begin{align}
\Gamma_{\bar{\chi}\chi}^{\rm{vector}}&=\frac{\gDM^2 \mMed}{12\pi}\left(1+\frac{2 \mDM^2}{\mMed^2} \right)\sqrt{1-\frac{4 \mDM^2}{\mMed^2}}\\
\Gamma_{\bar{q}q}^{\rm{vector}}&= \frac{3 \gq^2 \mMed}{12\pi}\left(1+\frac{2 m_q^2}{\mMed^2} \right)\sqrt{1-\frac{4 m_q^2}{\mMed^2}}\\
\Gamma_{\bar{\chi}\chi}^{\rm{axial}}&=\frac{\gDM^2 \mMed}{12\pi} \left(1-\frac{4 \mDM^2}{\mMed^2}\right)^{3/2}\\
\Gamma_{\bar{q}q}^{\rm{axial}}&= \frac{3 \gq^2 \mMed}{12\pi}\left(1-\frac{4 m_q^2}{\mMed^2}\right)^{3/2}\label{eq:Gamma4}\;.
\end{align}

\end{itemize}
\item Couplings
\item Parameter choices (for scan)
Vary mediator mass and DM mass 
\item Generator implementation
\end{itemize}

\paragraph{Scalar and pseudoscalar mediator, s-channel exchange}

\paragraph{Colored scalar mediator, t-channel exchange}

