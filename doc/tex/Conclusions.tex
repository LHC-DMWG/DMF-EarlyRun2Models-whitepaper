\svnidlong
{$HeadURL: svn+ssh://mrenna@svn.cern.ch/reps/LHCDMF/trunk/doc/tex/EFTValidity.tex $}
{$LastChangedDate: 2015-05-22 01:10:53 -0500 (Fri, 22 May 2015) $}
{$LastChangedRevision: 214 $}
{$LastChangedBy: doglioni $}
\svnid{$Id: $}

%Points to be developed further:
%\begin{itemize}
%
%\item In the case of s-channel simplified models, in particular, there is complementarity between searches for dark matter and visible particles.   Thus, limits on invisible decays must be consistent with dijet and dilepton searches.  In the case of the mono-top simplified model, limits on visible single-top final states must be considered.
%
%\item There are many implicit assumptions that have not been laid out entirely in our presentation.   As stated earlier, the term \textit{dark matter} in this report refers to a putative dark matter candidate.
%  The details of a particular mono-X analysis rely on the fact that a
%  WIMP exists, and that it is collider-stable.   The observation of a
%  signal consistent with WIMP production does can only provide indirect or confirming evidence of a dark matter particle.
%
%\item The presentation of results comparing different experimental frontiers has to be done carefully and clearly.  We see the need for broader discussions on this topic.
%
%\item The experiments should aim to present limits on production cross sections corrected for acceptance when this is viable.  
%
%\item The Appendix contains a presentation of some models that came out in our discussions that were not deemed a priority for early Run2 analyses.  However, these should be considered in the future.
%\end{itemize}

The ATLAS/CMS Dark Matter Forum has concluded its works in June 2015. Its mandate, 
limited in scope and time, was focused on the proposal of a prioritized, compact set of benchmarks
to be used for the design of the early Run-2 searches. 
This report document the studies by Forum contributors from experiments and theory that allowed
to identify a set of kinematically distinct models and model points. 
Table 6.1 summarizes the state of the art calculations and tools that are available to the LHC collaborations 
at the start of Run-2 and that are under development on a similar timescale. 
The ATLAS and CMS collaborations have agreed to follow the recommendations of the Forum on the use of
the state-of-the-art calculations for Dark Matter models. 
ATLAS and CMS will harmonize the Monte Carlo signal benchmark generation for 
early Run-2 searches, by simulating the central values for the model points detailed in this document,
using the tools employed for studies within the Forum and marked in bold in Table 6.1. 
 
%\clearpage 

\begin{footnotesize}
 
\begin{table}
	\centering\scriptsize
\begin{tabular}{llrr} \toprule \multicolumn{4}{c}{\textbf{Benchmark models for ATLAS and CMS Run-2 DM searches}}\\
	
	
	\cmidrule(r){2-4} 
	\multicolumn{4}{c}{vector/axial vector mediator, \schannel}\\
	\cmidrule(r){2-4} 

	Signature & State of the art calculation & Link to implementation & References \\ 
	\cmidrule(r){1-4} 
    jet + \MET{} & \textbf{NLO+PS (\powheg, r3059)} & \cite{ForumSVN_DMA, ForumSVN_DMV} & \cite{Haisch:2013ata,Alioli:2010xd,Nason:2004rx,Frixione:2007vw,Haisch:2015ioa} \\ 
	& NLO (\mcfm) & TBC & \cite{Fox:2012ru} \\ 
	& LO+PS (\madgraph, v2.2.3) &TBC & \cite{Alwall:2014hca,Alloul:2013bka,Degrande:2011ua} \\ 		
    $W/Z/\gamma$ + \MET{} & \textbf{LO+PS (\madgraph), v2.2.3} & \cite{ForumSVN_EW_DMV} & \cite{Alwall:2014hca,Alloul:2013bka,Degrande:2011ua}  \\ 


	\cmidrule(r){2-4} 
	\multicolumn{4}{c}{scalar/pseudoscalar mediator, \schannel}\\
	\cmidrule(r){2-4} 
	Signature & State of the art calculation & Link to implementation & References \\ 
		\cmidrule(r){1-4} 
		
	jet + \MET{} & \textbf{LO+PS, top loop (\powheg, r3059)} &  \cite{ForumSVN_DMS_tloop, ForumSVN_DMP_tloop} &  \cite{Haisch:2013ata,Alioli:2010xd,Nason:2004rx,Frixione:2007vw,Haisch:2015ioa} \\ 
	& NLO (\mcfm) & TBC & \cite{Fox:2012ru} \\ 
	& NLO+PS (\madgraph, v2.3.0) & TBC & \cite{Alwall:2014hca,Hirschi:2011pa,Alloul:2013bka,Degrande:2011ua} \\ 		
	$W/Z/\gamma$ + \MET{} & \textbf{LO+PS (\madgraph), v2.2.3} & TBC  & \cite{Alwall:2014hca,Alloul:2013bka,Degrande:2011ua} \\ 
    $t\bar{t},b\bar{b}$+ \MET{} & \textbf{LO+PS (\madgraph), v2.2.3} & TBC & \cite{Alwall:2014hca,Alloul:2013bka,Degrande:2011ua}\\ 
    & NLO+PS (\powheg, r3059) & TBC & \cite{Haisch:2013ata,Alioli:2010xd,Nason:2004rx,Frixione:2007vw,Haisch:2015ioa} \\ 
%    DMS contains S and P mediators that couple to the quark in a Yukawa-like way (at NLO). 
%    Since top quarks are not present in the proton, here the processes that dominate at tree-level are
%    b bbar -> chi chibar g (via S or P)
%    g b -> chi chibar b (via S or P)
%    and also bottom <-> charm processes.
%    Since the b and c yukawa are very small, cross sections here are extremelly small, and hence this process is less interesting than DMS_tloop in the monojet context.

	\cmidrule(r){2-4} 
	\multicolumn{4}{c}{scalar mediator, \tchannel}\\
	\cmidrule(r){2-4} 
	Signature & State of the art calculation & Link to implementation & References \\ 
		\cmidrule(r){1-4} 
		
	jet(s) + \MET{} & \textbf{LO+PS (\madgraph), v2.2.3} & \cite{ForumSVN_TChannel}& \cite{Papucci:2014iwa} \\ 
	$W/Z/\gamma$ + \MET{} & \textbf{LO+PS (\madgraph), v2.2.3} & TBC  & \cite{Alwall:2014hca,Alloul:2013bka,Degrande:2011ua}\\ 
		
%	
%	\midrule  scalar mediator, \tchannel &  jet(s) + \MET{} & \textbf{LO+PS (\madgraph)} & link &Ref \\ 
%	 $W/Z/\gamma$ + \MET{} & \textbf{LO+PS (\madgraph)} & link &Ref \\ 
	
	\bottomrule 
	\end{tabular}
	\label{tab:summaryModels}
\end{table}

\end{footnotesize}

\begin{center}
	Table 6.1: Summary table for available benchmark models considered within the works of this Forum. Experimental collaborations will use the implementation in bold for Run-2 searches. 
\end{center}
. 

%This document primarily presents studies related to simplified models. 
%The presentation of results for EFT benchmark models is also discussed. 
%The studies contained in this report are meant to highlight the use of
%EFTs as a benchmark that is complementary to simplified models, 
%and to demonstrate how that collider results could be presented a function of the 
%fraction of events that are valid within the contact interaction approximation.  
%
%A number of points remain to be developed beyond the scope of this Forum, 
%in order to fully benefit from LHC searches in the global quest for Dark Matter.
%First and foremost, many implicit assumptions have not been fully explored in this work. 
%As a consequence, the list of models employed for early LHC Run-2 searches is not meant to be 
%complete or exhaustive, nor cover the entire range of possibility even for the limited scope of the 
%putative collider-stable dark matter candidate considered. 
%The limitations of the list of benchmarks considered should encourage the theoretical community to continue developing
%models and calculations, and guide the experimental community to test and use those new ideas in future searches. 
%Furthermore, we see the need for broader discussion on the comparison 
%of experimental results from different energy frontiers. The role of constraints on the mediator particles 
%from direct past and present collider searches should be developed further. 
%The uncertainties in these comparisons should be discussed and conveyed, so that the different results 
%can be placed in the correct context to enhance their complementarity.
%Work beyond this Forum will be needed to provide guidance on the identification of a possible signal, 
%if hints are found either during the early LHC Run-2 or by direct and indirect detection experiments. Conversely,
%if nothing is found, novel collider search strategies that explore new corners will be required. 
%\Todo{These points will be discussed at the meeting that marks the conclusion of the works of the Forum, to be held during the week of June 10th.} 

